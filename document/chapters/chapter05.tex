\chapter*{Pushforward}
\addcontentsline{toc}{chapter}{Pushforward}

One may ask whether cocartesian fibrations in $\sSet$ model Pi types, which is a
piece needed to understand a novel model of dependent type theory provided by
$\sSet$. To answer this question, an explicit description of the right adjoint
$p_*$ of the pullback functor $p^*\colon\sSet/Y\rightarrow\sSet/X$ induced by a
morphism $p\colon X\rightarrow Y$ is needed.

Consider an object $f\colon T\rightarrow X$ in $\sSet/X$. What is $p_*(f)\colon
T'\rightarrow Y$? We know that a $n$-simplex $t'$ of $T'$ corresponds
bijectively to a map $t'\colon\Delta^n\rightarrow T'$, which in turn corresponds
bijectively to a commutative diagram
\[\begin{tikzcd}
	{\Delta^n} & {} & {T'} \\
	& Y
	\arrow["{t'}", from=1-1, to=1-3]
	\arrow["y"', from=1-1, to=2-2]
	\arrow["{p_*(f)}", from=1-3, to=2-2]
\end{tikzcd}\]
and, under the adjunction $p^*\dashv p_*$, we get bijectively another
commutative diagram
\[\begin{tikzcd}
	U & {} & T \\
	& X
	\arrow["t", from=1-1, to=1-3]
	\arrow["{p^*(y)}"', from=1-1, to=2-2]
	\arrow["f", from=1-3, to=2-2]
\end{tikzcd},\]
from which follows that
\[T'_n\cong\{(y,t)\ |\ y\in Y_n,\ t\in\sSet/X(p^*(y),f)\}\]
and the map $p_*(f)$ then sends $(y,t)\in T'_n$ to $y\in Y_n$.

The same method can be extended to give us the pushforward along a map of marked
simplicial sets $p\colon (X,E_X)\rightarrow (Y,E_Y)$ in $\mSet$. Specifically,
our previous construction can be adapted to give us the $n$-simplices by
starting from maps $(\Delta^n)_\flat\rightarrow p_*(T,E_T)=(T',E_{T'})$, telling
us again that
\[T'_n\cong\{(y,t)\ |\ y\in Y_n,\ t\in\mSet/X(p^*(y),f)\},\]
while to get the markings we notice that every marked edge in $p_*(T,E_T)$
corresponds to a unique map $(\Delta^1)_\natural\rightarrow p_*(T,E_T)$ and the
same procedure allows us to write
\[E_{T'}=\{(y,t)\ |\ y\in E_Y,\ t\in\mSet/X(p^*(y),f)\},\]
which fully specifies the needed data.

Now, under which conditions on $p$ does this specify a Quillen adjunction when
the slices of $\mSet$ are equipped with the contravariant model structure? If
it is a coCartesian fibration, it generally doesn't, but it does when it is a
Cartesian fibration. How can we specify an approximation $q$ of $p_*$ such that,
after localizing in the infinity-sense, we get an adjunction $p^*\dashv q$?

Idea: use the theory of bifibrations. From a coCartesian fibration $\phi\colon
X\rightarrow Y$ we can construct a bifibration $E\rightarrow X\times Y$ by
constructing the maps $p\colon E\rightarrow X$, $q\colon E\rightarrow Y$ by
first taking the pullback of $\phi$ along $ev_0\colon Y^{\Delta^1}\rightarrow Y$
and then composing the map $E\rightarrow Y^{\Delta^1}$ with $ev_1$.
\[\begin{tikzcd}
	E & X \\
	{Y^{\Delta^1}} & Y \\
	Y
	\arrow["\phi", from=1-2, to=2-2]
	\arrow["{ev_0}", from=2-1, to=2-2]
	\arrow["p", from=1-1, to=1-2]
	\arrow[from=1-1, to=2-1]
	\arrow["{ev_1}", from=2-1, to=3-1]
	\arrow["\lrcorner"{anchor=center, pos=0.125}, draw=none, from=1-1, to=2-2]
	\arrow["q"', curve={height=20pt}, from=1-1, to=3-1]
\end{tikzcd}\]

We want to show that, for any coCartesian morphisms $f\colon A\rightarrow Y$,
$g\colon A\rightarrow X$, we have an equivalence between $\Map_(\phi^*f,g)$ and
$\Map(f,q_*p^*(g))$.

Canonically, we have
\[E_n=\{(x,\Delta^n\times\Delta^1\xrightarrow{g}Y)\ |\ x\in X_n,\
\phi(x)=g|_{\Delta^n\times\{0\}}\}\]
and, by pasting the pullback squares, we also get
\[(\dom(p^*f))_n=\{(a,\Delta^n\times\Delta^1\xrightarrow{g} Y)\ |\ a\in A_n,\
f(a)=g|_{\Delta^n\times\{0\}}\}\]
and therefore
\begin{align*}
  (\dom(q_*p^*(f)))_n &=\{(y,q^*(y)\xrightarrow{g} p^*(f))\ |\ y\in Y_n\} \\
                      &=\{(y,p_!q^*(y)\xrightarrow{g} f)\ |\ y\in Y_n\},
\end{align*}
which we want to relate to $\phi_*(f)$.

To do this, we want to understand the maps $p_!q^*(y)\xrightarrow{g} f$ and
somehow relate them to $\phi^*(y)\rightarrow f$. By definition,
\begin{align*}
  \dom(p_!q^*(y))_k&=\dom(q^*(y))_k \\
                   &=\{(x,\Delta^k\times\Delta^1\xrightarrow{h}Y,t)\ |\ x\in
                     X_k,\ \phi(x)=h|_{\Delta^k\times\{0\}},\ t\in(\Delta^n)_k,\
                   y(t)=h|_{\Delta^k\times\{1\}}\},
\end{align*}
with $q^*(y)(x,h,t)=(x,h)$, thus $p_!q^*(y)(x,h,t)=x$.

On the other hand, we have
\[\dom(\phi^*(y))_k=\{(x,t)\ |\ x\in X_k,\ t\in(\Delta^n)_k,\ \phi(x)=y(t)\}\]
and $\phi^*(y)(t,x)=x$.

If we can create a bijection between morphisms of the form $p_!q^*(y)\rightarrow
f$ and $\phi^*(y)\rightarrow f$ we are done. Unfortunately, I do not see how we
can do this: any morphism $p_!q^*(y)\rightarrow f$ induces a
morphism $\phi^*(y)\rightarrow f$ by precomposing with the inclusion
$\phi^*(y)\rightarrow p_!q^*(y)$, $(x,t)\mapsto(x,h_{\phi(x)},t)$, where
$h_{\phi(x)}$ is obtained by precomposing $\phi(x)\colon\Delta^k\rightarrow Y$
with $p_{\Delta^k}\colon\Delta^k\times\Delta^1\rightarrow\Delta^k$, but this
association is only injective, not surjective, and I have no good idea about how
to construct others.

To construct the bijection I may start from a morphism $p_!q^*(y)->f$ and
construct another one with $\phi^*(y)$ as domain by lifting morphisms
$\Delta^k\times\Delta^1\rightarrow Y$ to decide where to map
$(x,h,t)\in\dom(p_!q^*(y))_k$, however this involves solving a coherence problem
and I would have to do so coherently to define a morphism of simplicial sets as
desired. Perhaps these restrictions actually allow a solution, but I do not
believe so.

It may also be possible that the injective morphism we mentioned earlier is a
weak equivalence with respect to our model structure, which may be enough.

We provide a counterexample to the previous claim in the context of right
fibrations. Consider
$\phi\colon\partial\Delta^1\rightarrow\Delta^1$, $i\mapsto 0$, which is a right
fibration. We have that $\phi^*(1)=0$, the empty simplicial set, thus
$\phi_*(f)^{-1}(1)\cong\Delta^0$. On the other hand, $(p_!q^*)(1)=U\amalg V$,
thus $q_*p^*(f)^{-1}(1)$ can be a disjoint union of non-zero simplicial sets,
which would then not be an equivalent $\infty$-groupoid. It follows that our map
is not, in general, a weak equivalence in the model structure of right
fibrations on slices of $\sSet$. \wfd{(MAYBE WRONG: YOU CAN'T CHECK THIS ON
FIBERS BECAUSE THESE ARE NOT FIBRANT OBJECTS IN $\sSet/Y$! NEED TO USE THE
DEFINITION CONCERNING HOMOTOPY CLASSES OF MAPS INTO FIBRANT OBJECTS IN THE
SLICE)}

If instead from a right fibration $\phi$ we first take the opposites of $\phi$
and $f$, then do the pushforward and finally we take again the opposites we get
$\phi_*(f)$, which is encoded in the following commutative diagram where the
vertical maps are isomorphisms.
\[\begin{tikzcd}
	{f\colon Z\rightarrow X} & {\sSet/X} & {\sSet/Y} & {f\colon Z\rightarrow Y} \\
	{f^{\op}\colon Z^{\op}\rightarrow X^{\op}} & {\sSet/X^{\op}} & {\sSet/Y^{\op}} & {f^{\op}\colon Z^{\op}\rightarrow Y^{\op}}
	\arrow["\op", from=1-3, to=2-3]
	\arrow["\op"', from=1-2, to=2-2]
	\arrow["{\phi^{\op}_*}"', from=2-2, to=2-3]
	\arrow["{\phi_*}", from=1-2, to=1-3]
	\arrow[maps to, from=1-1, to=2-1]
	\arrow[maps to, from=1-4, to=2-4]
\end{tikzcd}\]

The same argument extends to show that any map $\phi_*(f)\rightarrow q_*p^*(f)$
or in the other direction is not a weak equivalence in general. A concrete
example can be given by taking $f=\phi$.


\printbibliography

\end{document}

