\chapter{Background}

We begin by presenting our dependent type theory and then introduce a class of
algebraic models.

\section{Logical Setting}

We assume some familiarity with some kind of dependent type theory, preferably
Martin-L\"{o}f Type Theory \cite{ML84} or an extension, which
should be enough to understand our setting. This has become a popular formal
theory of logic to start from when creating an alternative foundation of
mathematics since it provides a more nuanced concept of equality, the
possibility of creating powerful proof assistants and, thanks to its relation to
homotopy theory, cleaner and more general proofs of some theorems, like the
Black-Massey theorem, presented in the language of Homotopy Type Theory
\cite{Uni13}.

\begin{notation}
  A context $\Gamma$ is a sequence $a_1:A_1,\ldots,a_n:A_n$ of types $A_i$ with
  choices of terms
  $a_i:A_i$, each of them derivable from the previous ones and possibly
  dependent on them.
  \begin{align*}
    a_1:A_1,\ldots,a_{i-1}:A_{i-1} &\vdash A_i \text{ type} \\
    x_1:A_1,\ldots,x_{i-1}:A_{i-1} &\vdash x_i:A_i.
  \end{align*}
  We shall hide the
  dependency on previous terms in our notation (as we have just done) unless
  it is needed for clarity, in which case we
  may write $B(a_1,\ldots,a_n)$ for a type $B$ dependent on
  $a_1:A_1,\ldots,a_n:A_n$ and similarly $b(a_1,\ldots,a_n):B(a_1,\ldots,a_n)$
  for a dependent term or suppress individual dependencies.
\end{notation}

\noindent
We will be considering the structural rules from \cite[App.\ A.1]{KL12} and some
logical ones as presented in \cite[App.\ A.2]{KL12}. Informally, we deal with:
\begin{enumerate}
  \item \emph{dependent sums} $\Sigmas(A,B)$, also called
    \emph{$\Sigmas$-types}, and the associated introduction, elimination and
    computation rules;
  \item \emph{dependent products} $\Pis(A,B)$, also called \emph{$\Pis$-types},
    and the associated introduction, elimination and computation rules;
  \item \emph{(intensional) identity types} $\Ids_A$, also called
    \emph{$\Ids$-types}, and the associated introduction, elimination and
    computation rules.
\end{enumerate}

\noindent
Furthermore, we consider the following two rules:
\begin{enumerate}[resume]
  \item \emph{$\eta$-conversion for functions}: for any type $A$ and any type $B$
    dependent on $x:A$, given a function $f:\Pis(A,B)$, we have
    $\lambda(x:A).\app(f,x)\equiv f$;
  \item \emph{function extensionality}: for any type $A$ and any type $B$ dependent on
    $x:A$, given functions $f,g:\Pis(A,B)$, if $\app(f,x)=\app(g,x)$ for every
    $x:A$, then $f=g$.
\end{enumerate}

\begin{defn}
  We say that a dependent type theory $\T$ has \emph{$\Pies$-types} if its
  $\Pi$-types satisfy the $\eta$-conversion for functions.
\end{defn}

\begin{rmk}
  Observe that when specifying $\eta$-conversion we used a \emph{judgemental
  equality}, while for function extensionality we used a simple equality. The
  latter means that, given $h:\Pis(x:A,\Id_B(\app(f,x),\app(g,x)))$, we can
  present a term $\ext(f,g,h)$ of type $\Ids_{\Pis(A,B)}(f,g)$.
\end{rmk}

\begin{rmk}
It should be noted that function extensionality is a common assumption, however
it is sometimes derived instead from other principles, like in Homotopy Type
Theory, where it is a consequence of univalence \cite[Ch.\ 4.9]{Uni13}.
Also, the other principles we are considering are fairly general and very
desirable for any dependent type theory meant as a candidate foundation of
mathematics, however they are still not enough for such a role: indeed, we lack
introduction rules to construct any type or term starting from the empty
context. This is fine, since it means that our results will apply to virtually
any type-theoretic foundation of mathematics.
\end{rmk}

\section{Contextual Categories}

A problem of working with dependent type theory is
the handling of the structural rules, namely context substitution, variable
binding, variable substitution and so on. To avoid this issue, it is convenient
to work within a \emph{semantic model}, which we now present.

\noindent
There are many algebraic models of dependent type theory in the literature meant
to only satisfy these rules, like
\emph{category with attributes}, \emph{categories with families} and
\emph{comprehension categories}, which are
fairly similar among them.
Here we shall work with \emph{contextual
categories}, which were first explored by Cartmell and Streicher
\cite{Car78,Car86,Str91} and later by Voevodsky, under the name
$C$\emph{-systems} \cite{Voe14a}. For informations on the
others we refer to \cite{nlab:categorical_model_of_dependent_types}.

\begin{defn}
  A \emph{contextual category} $\sfC$ is a small category, also denoted $\sfC$,
  with the following data:
  \begin{enumerate}
    \item a length function $l\colon\Ob\sfC\rightarrow\bbN$ or, equivalently, a
      grading on objects $\Ob\sfC=\coprod_{n\in\bbN}\Ob_n\sfC$, also called
      \emph{contexts};
    \item an object $*\in\Ob_0\sfC$, the \emph{empty context};
    \item for each $n\in\bbN$, a map
      $\cft_n\colon\Ob_{n+1}\sfC\rightarrow\Ob_n\sfC$, often simply denoted
      $\cft$;
    \item for each $n\in\bbN$ and $X\in\Ob_{n+1}\sfC$, a map $p_X\colon
      X\rightarrow\cft X$;
    \item for each $n\in\bbN$, $X\in\Ob_{n+1}\sfC$ and $f\colon Y\rightarrow
      \cft X$, an object $f^*X$ and a map $q(f,X)\colon f^*X\rightarrow X$;
  \end{enumerate}
  such that:
  \begin{enumerate}
    \item $*$ is the unique element of $\Ob_0\sfC$;
    \item $*$ is terminal;
    \item for each $n$, $X\in\Ob_{n+1}\sfC$ and $f\colon Y\rightarrow \cft X$, we
      have $\cft(f^*X)=Y$ and the square
      \[\begin{tikzcd}
        {f^*X} & X \\
        Y & \cft X
        \arrow["f"', from=2-1, to=2-2]
        \arrow["{p_X}", from=1-2, to=2-2]
        \arrow["{q(f,X)}", from=1-1, to=1-2]
        \arrow["{p_{f^*X}}"', from=1-1, to=2-1]
      \end{tikzcd}\]
      is a pullback;
    \item for each $n\in\bbN$, $X\in\Ob_{n+1}\sfC$ and pair of maps $f\colon
      Y\rightarrow \cft X$, $g\colon Z\rightarrow Y$, we have $(fg)^*X=g^*f^*X$,
      $1^*_{\cft X}X=X$, $q(fg,X)=q(f,X)\cdot q(g,f^*X)$ and $q(1_{\cft X},X)=1_X$.
  \end{enumerate}
\end{defn}

\begin{rmk}
  The last condition in the definition means that our choice of pullbacks is
  functorial, which allows us to see contextual categories as a strict model of
  dependent type theory, not requiring to keep track of coherency maps.
  Other models, like comprehension categories, do not
  have such requirements, which makes them more general and
  interpretation harder, unless they are first strictified, that is replaced by
  an equivalent model of the same kind but with functorial pullbacks.
\end{rmk}

\noindent
A motivating example for contextual categories is the \emph{syntactic category
of a dependent type theory}, which is constructed as follows.

\begin{construction}\cite[2.6]{Car78}
  Given a dependent type theory $\T$, its syntactic category $\Syn{\T}$ has:
  \begin{enumerate}
    \item $\Ob_n\Syn{\T}$ given by contexts $[x_1:A_1,\ldots,x_n:A_n]$ of length
      $n$, modulo judgemental equality and renaming of free variables;
    \item maps are \emph{context morphisms}, or \emph{substitutions}, modulo
      judgemental equalities and renaming of free variables. This means that a
      map
      \[f\colon[x_1:A_1,\ldots,x_n:A_n]\rightarrow[y_1:B_1,\ldots,y_m:B_m(y_1,\ldots,y_{m-1})]\]
      is an equivalence class of tuples of terms $(f_1,\ldots,f_m)$ such that
      \begin{align*}
        x_1:A_1,\ldots,x_n:A_n &\vdash f_1:B_1, \\
        \vdots & \\
        x_1:A_1,\ldots,x_n:A_n &\vdash f_m:B_m(f_1,\ldots,f_{m-1}),
      \end{align*}
      are all derivable judgements and two such tuples
      $(f_1,\ldots,f_n),(g_1,\ldots g_n)$ are equivalent if we have
      \[x_1:A_1,\ldots,x_n:A_n\vdash f_i\equiv g_i:B_i(f_1,\ldots,f_{i-1})\]
      for every $i$; we shall henceforth also write $[f_i]$ for $f$;
    \item composition is given by substitution;
    \item the identity $\Gamma\rightarrow\Gamma$ is given by the variables of
      $\Gamma$, considered as
      terms, that is we take the sequence $[x_i]$ given by
      \begin{align*}
        x_1:A_1,\ldots,x_n:A_n &\vdash x_1:A_1, \\
        \vdots & \\
        x_1:A_1,\ldots,x_n:A_n &\vdash x_n:A_n(x_1,\ldots,x_{n-1});
      \end{align*}
    \item the terminal object is the empty context $[]$;
    \item
      $\cft([x_1:A_1,\ldots,x_{n+1}:A_{n+1}])=[x_1:A_1,\ldots,x_n:A_n]$;
    \item for $\Gamma=[x_1:A_1,\ldots,x_n:A_{n+1}]$,
      $p_{\Gamma}\colon\Gamma\rightarrow \cft\Gamma$ is the \emph{dependent
      projection context morphism} $[x_1,\ldots,x_n]$, defined by
      \begin{align*}
        x_1:A_1,\ldots,x_{n+1}:A_{n+1} &\vdash x_1:A_1, \\
        \vdots & \\
        x_1:A_1,\ldots,x_{n+1}:A_{n+1} &\vdash x_n:A_n(x_1,\ldots,x_n)
      \end{align*}
      and thereby simply forgetting the last variable of $\Gamma$;
    \item given contexts
      \begin{align*}
        \Gamma &=[x_1:A_1,\ldots,x_{n+1}:A_{n+1}(x_1,\ldots,x_n)], \\
        \Delta &=[y_1:B_1,\ldots,y_m:B_m(y_1,\ldots,y_{m-1})]
      \end{align*}
      and a map $f=[f_i(y)]\colon\Delta\rightarrow\cft\Gamma$ (where $y$ is a
      vector of variables of length $m$), the pullback $f^*\Gamma$ is the
      context
      \[[y_1:B_1,\ldots,y_m:B_m(y_1,\ldots,y_{m-1}),y_{m+1}:A_{n+1}(f_1(y),\ldots,f_n(y))]\]
      for some new variable $y_{m+1}$, while $q(\Gamma,f)\colon
      f^*\Gamma\rightarrow\Gamma'$ is specified by $[f_1,\ldots,f_n,y_{m+1}]$.
  \end{enumerate}
\end{construction}

\begin{rmk}
  Given a dependent type theory $\T$, the terms $a:A$ of a type over a context
  $\Gamma$ can be recovered (up to definitional equality) from the syntactic
  category $\Syn{\T}$ by looking at sections of the basic dependent projection
  $p_{[\Gamma,x:A]}\colon[\Gamma,x:A]\rightarrow[\Gamma]$, which indeed act as
  identities over $\Gamma$ and furthermore specify a term $\Gamma\vdash a:A$.
  Given the importance of such maps, we shall often simply write ``sections'' to
  refer to sections of basic dependent projections, without specifying which
  projections unless it creates ambiguity.

\noindent
  The above construction also tells us how to think about the other elements in
  the definition of contextual categories: basic dependent
  projections $p_{\Gamma.A.B}\colon\Gamma.A.B\rightarrow\Gamma.A$ represent
  dependent types $B$ over $A$ in the context $\Gamma$, while pulling back along
  a dependent projection corresponds to switching context
  and so on for the other objects in the definition.
\end{rmk}

\begin{notation}
  We shall also make use of some conventions inspired by this construction.
  Namely, given a contextual category $\sfC$ and an object $\Gamma\in\Ob_n\sfC$,
  we shall write $\Gamma.A_1.\ldots.A_k$ and $\Gamma.\Delta$ interchangeably
  for an object $X$ in $\Ob_{n+k}\sfC$ with $\cft^kX=\Gamma$ and call it a
  \emph{context extension of $\Gamma$ of length $k$}. We shall
  also write $p_{A_1.\ldots.A_k}$ and $p_\Delta$ for the
  composition of the basic dependent projections
  $p_{\Gamma.A_1.\ldots.A_i}$, with $i$
  ranging from $1$ to $k$, and the resulting map will be called a
  \emph{dependent projection}. In the case where $k=1$, $p_{A_1}$ corresponds to
  a basic dependent projection and the context extension of $\Gamma$ will be
  \emph{simple}, while if $k=0$ we have $p=1_{\Gamma}$ and then the context
  extension will be \emph{trivial}. We shall also write $1_\Delta,
  1_{A_1.\ldots.A_k},\ldots,1_{A_k}$ for $1_{\Gamma.\Delta}$, depending on what
  we want to emphasize. Greek letters
  shall be used to indicate context extensions of arbitrary length, while Latin
  ones will be reserved to simple extensions. \\
  Continuing, given a dependent projection $p_{A_1.\ldots.A_k}=p_\Theta$ as
  above and a context morphism $f\colon\Delta\rightarrow\Gamma$, we define
  inductively $f^*(\Gamma.\Theta)=\Delta.f^*\Theta=\Delta.f^*A_1.\ldots.f^*A_k$
  and $q(f,\Gamma.\Theta)=q(f,\Theta)=q(f,A_1.\ldots.A_k)$ by looking at the
  pasting of pullback squares
  \[\begin{tikzcd}
    {\Delta.f^*A_1.\ldots.f^*A_k} & {\Delta.f^*A_1.\ldots.f^*A_{k-1}} & \cdots & {\Delta.f^*A_1} & \Delta \\
    {\Gamma.A_1.\ldots.A_k} & {\Gamma.A_1.\ldots.A_{k-1}} & \cdots & {\Gamma.A_1} & \Gamma
    \arrow["f", from=1-5, to=2-5]
    \arrow["{p_{A_1}}", from=2-4, to=2-5]
    \arrow["{p_{f^*A_1}}"', from=1-4, to=1-5]
    \arrow["{q(f,A_1)}"', from=1-4, to=2-4]
    \arrow[from=2-3, to=2-4]
    \arrow[from=1-3, to=1-4]
    \arrow[from=1-2, to=1-3]
    \arrow[from=2-2, to=2-3]
    \arrow["{q(f,A_1.\ldots.A_{k-1})}", from=1-2, to=2-2]
    \arrow["{q(f,A_1.\ldots.A_k)}"', from=1-1, to=2-1]
    \arrow["{p_{f^*A_k}}"', from=1-1, to=1-2]
    \arrow["{p_{A_k}}", from=2-1, to=2-2]
    \arrow["{p_{f^*A_1.\ldots.f^*A_k}}", curve={height=-24pt}, from=1-1, to=1-5]
    \arrow["{p_{A_1.\ldots.A_k}}"', curve={height=24pt}, from=2-1, to=2-5]
  \end{tikzcd},\]
  which also shows that
  $$q(f,A_1.\ldots.A_k)=q(q(f,A_1),A_2.\ldots.A_k)=q(q(f,A_1.\ldots.A_{k-1}),A_k).$$
  As usual, if $k=0$ we have 
  \[q(f,\Theta)=f,\quad f^*(\Gamma.\Theta)=\Gamma,\]
  while for $k=1$
  \[q(f,A_1)=q(f,\Gamma.A_1),\quad \Delta.f^*A_1=f^*(\Gamma.A_1),\]
  therefore agreeing with the base structure of $\sfC$.

\noindent
  Furthermore, given a section $a\colon\Gamma\rightarrow\Gamma.A$ and a context
  morphism $f\colon\Delta\rightarrow\Gamma$, we also want to specify $f^*a$,
  that is the section which we get by switching context. This is given by
  the map $(1_{\Delta},a\cdot f)$ specified by the pullback square
  \[\begin{tikzcd}
    \Delta \\
    & {\Delta.f^*A} & {\Gamma.A} \\
    & \Delta & \Gamma
    \arrow["f"', from=3-2, to=3-3]
    \arrow["{p_{f^*A}}", from=2-2, to=3-2]
    \arrow["{p_A}", from=2-3, to=3-3]
    \arrow["{q(f,A)}"', from=2-2, to=2-3]
    \arrow[curve={height=18pt}, Rightarrow, no head, from=1-1, to=3-2]
    \arrow["{a\cdot f}", curve={height=-18pt}, from=1-1, to=2-3]
    \arrow["{(1_\Delta,a\cdot f)}"{description}, dotted, from=1-1, to=2-2]
  \end{tikzcd},\]
  and, as shown by the commutative diagram
  \[\begin{tikzcd}
    \Delta & \Gamma \\
    {\Delta.f^*A} & {\Gamma.A} \\
    \Delta & \Gamma
    \arrow["f"', from=3-1, to=3-2]
    \arrow["{p_{f^*A}}", from=2-1, to=3-1]
    \arrow["{p_A}"', from=2-2, to=3-2]
    \arrow["{q(f,A)}"', from=2-1, to=2-2]
    \arrow["a"', from=1-2, to=2-2]
    \arrow["{f^*a}", from=1-1, to=2-1]
    \arrow["f", from=1-1, to=1-2]
    \arrow[curve={height=-24pt}, Rightarrow, no head, from=1-2, to=3-2]
    \arrow[curve={height=24pt}, Rightarrow, no head, from=1-1, to=3-1]
  \end{tikzcd},\]
  it corresponds to the pullback of $a$ along $q(f,A)$. By the techniques we
  provided earlier, we extend this construction to contexts extensions of
  arbitrary length.

\noindent
  Finally, when reasoning in a syntactic category, to specify a map
  $$[f]\colon[\Gamma,a_1:A_1,\ldots,a_n:A_n]\rightarrow[\Gamma,b_1:B_1,\ldots,b_m:B_m]$$
  stable over $\Gamma$ (that is acting like the identity on the corresponding
  variables) we can suppress the corresponding terms in the tuple. To
  produce such a map, we shall write then
  $$(a_1,\ldots,a_n)\mapsto(f_1,\ldots,f_m)$$ and clarify how to construct the
  $f_i$ from the variables in the domain context. We will also write
  \[(f_1,\ldots,f_n)\equiv(g_1,\ldots,g_n)\]
  if we have
  $$\Gamma\vdash f_i\equiv g_i : B_i(f_1,\ldots,f_{i-1})$$
  for every $i$.
\end{notation}

\begin{rmk}
  When we pull back a dependent projection
  $p_\Delta\colon\Gamma.\Delta\rightarrow\Gamma$ along another one
  $p_\Psi$, we may also choose to do it the other way around, which induces an
  isomorphism
  $\Gamma.\Delta.p^*_\Delta\Psi\rightarrow\Gamma.\Psi.p^*_\Psi\Delta$
  denoted by $\exch_{\Delta,\Psi}$. Syntactically, this amounts to swapping two
  context extensions not dependent on one another.
\end{rmk}

\begin{defn}
  A \emph{contextual functor} between contextual categories
  $F\colon\sfC\rightarrow\sfD$ is a functor on the underlying categories which
  preserves the grading, basic dependent projections and such that
  $q(Ff,FX)=F(q(f,X))$.
\end{defn}

\begin{rmk}
  Our definition allows us to see contextual categories as models for a (small)
  essentially algebraic theory \cite{AR94} with sorts
  indexed by $\bbN+\bbN\times\bbN$. In that context, we get a notion of
  morphism between models of this theory,
  which coincides with the one we have just provided. Its category of models
  will be the category of contextual categories, denoted by $\Cxl$,
  which is complete and cocomplete as the category of models of an
  essentially algebraic theory. The same can be done for the other algebraic
  models we mentioned.
\end{rmk}

\noindent
As we stated, the defining properties of contextual
categories are meant to model structural rules, so that we can reason about
dependent type theories in terms of pullbacks, sections and so on while
eliminating a lot of the bureaucracy needed to provide a model, which can now be
done by constructing such a category.

\begin{exmp}
  A constructive and formalization-ready approach to producing contextual
categories was developed by Voevodsky in \cite{Voe14b,Voe14b,Voe15a,Voe15b},
  where he starts from a \emph{universe category}.
\end{exmp}

\noindent
On the other hand, as noted in \cite[par.\ 1.2]{KL12}, interpreting complicated
structures in
contextual categories quickly becomes unreadable, meaning that ideally we would
like to be able to switch from a syntactic to a semantic presentation
and viceversa freely, so that we may work with the most
convenient one for a given situation. This would amount to some \emph{soundness}
and \emph{correctness} theorems reminiscent of the ones of \emph{first order
predicate logic}. In categorical semantics, this corresponds to
\emph{initiality}.

\begin{named}[Initiality]\label{initconj}
  Given a dependent type theory $\T$, its syntactic category $\Syn{\T}$ is
  initial in the category of contextual categories with the appropriate
  structure.
\end{named}

\begin{rmk}
Here by ``appropriate structure'' we mean extra algebraic structures meant to
model the
logical rules of the type theory, like $\Sigmas$-types, $\Pis$-types and
$\Ids$-types. Indeed, the definition of contextual category is not meant to deal
with them. We will introduce the such structures in the next section.
\end{rmk}

\begin{rmk}
This conjecture has been partially proven for a simple variant of dependent type
theory in \cite{Str91} and a proof formalized in Agda
for Martin-L\"{o}f Type Theory was provided by de Boer and Brunerie
\cite{Boe20}, however we still do not have a fully
rigorous general statement. Such a statement would first require a general
notion of dependent type theory,
which has been worked on in \cite{Isa17, Uem19, Bru20, BHL20, NU22}. \wfd{(ARE
YOU SURE ABOUT THESE CITATIONS?)}
\end{rmk}

\noindent
If we could prove it, then, under the proper conditions,
for any contextual category $\sfC$ we would have a
contextual functor $\Syn{\T}\rightarrow\sfC$ explaining how to interpret $\T$ in
$\sfC$. This is essentially an algorithmic problem: it reduces to explaining
inductively to a computer how to construct the aforementioned functor.

\begin{rmk}\label{internal}
  From the Initiality Conjecture researchers have also derived some homotopical
  models, which are the great drivers
  of research in the field. Specifically, it was shown
  that \emph{$\infty$-toposes} can model Homotopy Type Theory \cite{Shu19} and,
  as a concrete example, in \cite{KL12}
  one can find a model of Univalent Type Theory produced using Voevodsky's
  previously mentioned methods. More recent results concern the \emph{Internal
  Languages Conjecture} \ref{intlang}.
  \wfd{(UNIVALENT = HOMOTOPY???)}
\end{rmk}

\begin{rmk}
Another reason researchers often assume this conjecture to be true is that,
thanks to what is known in folklore as the \emph{syntax-semantic adjunction}, it
provides \emph{internal languages}, an intuitive but not yet formalized notation
to reason about contextual categories. We shall not use such instruments
and instead restrict ourselves to working with syntactic categories, with the
idea that a generalization of our results to arbitrary contextual categories
should be straightforward once we have a proof of the conjecture and all of the
tools mentioned above. Unfortunately,
this will also mean that we will not be able to rely on some results and
definitions as originally stated.
\end{rmk}

\begin{rmk}
It should be noted that, since $\Cxl$ and its variants are complete, they do
have initial objects.
\end{rmk}

\section{Logical Structures}

We now define the extra structures on contextual categories we mentioned
earlier. Our definitions shall be taken from \cite{KL12,KL18}.

\begin{defn}
  A \emph{$\Sigmas$-structure} on a contextual category $\sfC$ consists of:
  \begin{enumerate}
    \item for each $\Gamma.A.B\in\Ob_{n+2}\sfC$, an object
      $\Gamma.\Sigmas(A,B)\in\Ob_{n+1}\sfC$;
    \item for each $\Gamma.A.B\in\Ob_{n+2}\sfC$, a morphism $\pair_{A,B}\colon
      \Gamma.A.B\rightarrow\Gamma.\Sigmas(A,B)$ over $\Gamma$;
    \item for each $\Gamma.A.B,\Gamma.\Sigmas(A,B).C\in\Ob_{n+2}\sfC$, and
      $d\colon\Gamma.A.B\rightarrow\Gamma.\Sigmas(A,B).C$ with $p_C\cdot
      d=\pair_{A,B}$, a section
      $\csplit_d\colon\Gamma.\Sigmas(A,B)\rightarrow\Gamma.\Sigma(A,B).C$ such
      that $\csplit_d\cdot\pair_{A,B}=d$;
    \item where all of the above is compatible with context substitution, that
      is given a map $f\colon\Delta\rightarrow\Gamma$ we have
      \begin{align*}
        f^*(\Gamma.\Sigmas(A,B)) &=\Delta.\Sigma(f^*A,f^*B), \\
        f^*\pair_{A,B} &=\pair_{f^*A,f^*B}, \\
        f^*\csplit_d &=\csplit_{f^*d}.
      \end{align*}
  \end{enumerate}
\end{defn}

\begin{defn}
  A \emph{$\Ids$-structure} on a contextual category $\sfC$ consists of:
  \begin{enumerate}
    \item for each $\Gamma.A\in\Ob_{n+1}\sfC$, an object
      $\Gamma.A.p^*_AA.\Ids_A\in\Ob_{n+3}\sfC$;
    \item for each $\Gamma.A\in\Ob_{n+1}\sfC$, a morphism
      $\refl_A\colon\Gamma.A\rightarrow\Gamma.A.p^*_AA.\Ids_A$ such that
      $p_{\Ids_A}\cdot\refl_A=(1_A,1_A)\colon\Gamma.A\rightarrow\Gamma.A.p^*_AA$;
    \item for each $\Gamma.A.p^*_AA.\Ids_A.C$ and
      $d\colon\Gamma.A\rightarrow\Gamma.A.p^*_AA.\Ids_A.C$ with $p_C\cdot
      d=\refl_A$, a section
      $\sfJ_{C,d}\colon\Gamma.A.p^*_AA.\Ids_A\rightarrow\Gamma.A.p^*_AA.\Ids_A.C$
      such that $\sfJ_{C,d}\cdot\refl_A=d$;
    \item where all of the above is compatible with context substitution, that
      is given a map $f\colon\Delta\rightarrow\Gamma$ we have
      \begin{align*}
        f^*(\Gamma.A.p^*_AA.\Ids_A) &=\Delta.f^*A.p^*_{f^*A}(f^*A).\Ids_{f^*A} \\
        f^*\refl_A &=\refl_{f^*A} \\
        f^*\sfJ_{C,d} &=\sfJ_{f^*C,f^*d}.
      \end{align*}
  \end{enumerate}
\end{defn}

\begin{defn}\label{appcomposed}
  A \emph{$\Pis$-structure} on a contextual category $\sfC$ consists of:
  \begin{enumerate}
    \item for each $\Gamma.A.B\in\Ob_{n+2}\sfC$, an object
      $\Gamma.\Pis(A,B)\in\Ob_{n+1}\sfC$;
    \item for each $\Gamma.A.B\in\Ob_{n+2}\sfC$, a map
      $\app_{A,B}\colon\Gamma.\Pis(A,B).p^*_{\Pis(A,B)}A\rightarrow\Gamma.A.B$
      over $\Gamma$, that is such that $p_B\cdot\app_{A,B}=q(\Pis(A,B),A)$;
    \item for each $\Gamma.A.B\in\Ob_{n+2}\sfC$ and section
      $b\colon\Gamma.A\rightarrow\Gamma.A.B$, a section
      $\lambda_{A,B}(b)\colon\Gamma\rightarrow\Gamma.\Pis(A,B)$;
    \item such that for any sections
      $k\colon\Gamma\rightarrow\Gamma.\Pis(A,B)$,
      $a\colon\Gamma\rightarrow\Gamma.A$
      the map $\app_{A,B}(k,a)$ defined as the composition of $\app_{A,B}$ with
      $(k,a)$ specified by the factorization through the pullback
      \[\begin{tikzcd}[column sep=huge]
        \Gamma \\
        & {\Gamma.\Pis(A,B).p^*_{\Pis(A,B)}A} & {\Gamma.A} \\
        & {\Gamma.\Pis(A,B)} & \Gamma
        \arrow["{p_{\Pis(A,B)}}"', from=3-2, to=3-3]
        \arrow["{p_A}", from=2-3, to=3-3]
        \arrow["{q(p_{\Pis(A,B)},A)}"', from=2-2, to=2-3]
        \arrow["{p_{p^*_{\Pis(A,B)}A}}", from=2-2, to=3-2]
        \arrow["a", curve={height=-12pt}, from=1-1, to=2-3]
        \arrow["k"', curve={height=12pt}, from=1-1, to=3-2]
        \arrow["{(k,a)}"{description}, dotted, from=1-1, to=2-2]
      \end{tikzcd},\]
      we have $p_B\cdot\app_{A,B}(k,a)=a$;
    \item such that for any $\Gamma.A.B$ and sections
      $a\colon\Gamma\rightarrow\Gamma.A$,
      $b\colon\Gamma.A\rightarrow\Gamma.A.B$ we have
      \[\app(\lambda_{A,B}(b),a)=b\cdot a;\]
    \item all of the above is compatible with context substitution, that is for
      any $f\colon\Delta\rightarrow\Gamma$ we have
      \begin{align*}
        f^*(\Gamma,\Pis(A,B)) &=(\Delta,\Pis(f^*A,f^*B)), \\
        f^*\lambda_{A,B}(b) &=\lambda_{f^*A,f^*B}(f^*b), \\
        f^*(\app_{A,B}(k,a)) &=\app_{f^*A,f^*B}(f^*k,f^*a).
      \end{align*}
  \end{enumerate}

\noindent
  We shall say that the $\Pis$-structure satisfies the
  \emph{$\Pies$-rule} if the equation
  \[q(p_{\Pis(A,B)},\Pis(A,B))\cdot\lambda(1_{p^*_{\Pis(A,B)}A},\app_{A,B})=1_{\Gamma.\Pis(A,B)}\]
  is satisfied, in which case the structure will be called a
  \emph{$\Pies$-structure}. 

\noindent
  A \emph{$\Piext$-structure} on a contextual category with a $\Pis$-structure
  $\sfC$ is an operation giving for each $\Gamma.A.B\in\Ob_{n+2}\sfC$ a map
  \begin{align*}
    \ext_{A,B} &\colon
    \Gamma.\Pis(A,B).p^*_{\Pis(A,B)}\Pis(A,B).\Htp_{A,B}\rightarrow \\
    &\quad\quad\quad\quad\quad\quad\quad\quad
    \Gamma.\Pis(A,B).p^*_{\Pis(A,B)}\Pis(A,B).\Ids_{\Pis(A,B)}
  \end{align*}
  over $\Gamma.\Pis(A,B).p^*_{\Pis(A,B)}\Pis(A,B)$, stably in $\Gamma$, where
  $\Htp_{A,B}$ is a shorthand for the extension
  \begin{align*}
    \Pis\Big((p_{\Pis(A,B)}\cdot & p_{p^*_{\Pis(A,B)}\Pis(A,B)})^*A, \\
    &(\app_{A,B}\cdot q(p_{\Pis(A,B)},A),
    \app_{A,B}\cdot q(q(p_{\Pis(A,B)},\Pis(A,B)),A))^*\Ids_B\Big),
  \end{align*}
  representing the type of homotopies from between two terms of $\Pis(A,B)$. The
  map then models function extensionality.
\end{defn}

\begin{rmk}
  The definition we provided matches \cite[Def.~2.5]{KL18}, while the one in
  \cite[App.~B.1.1]{KL12} is mildly different as it starts from
  $\app(f,a)$ instead of $\app_{A,B}$ and constructs the latter from the
  former. We prefer ours because it will allow us to more easily extend the
  $\Pis$-structure to arbitrary context extensions and check the necessary
  properties in Chapter \ref{second}.
\end{rmk}

\begin{exmp}
  Given a dependent type theory $\T$, its syntactic category $\Syn{T}$ has all
  of the logical structures corresponding to its logical rules. We now proceed
  to show what some of the above maps correspond to in this category.
\end{exmp}

\begin{rmk}
  In $\Syn{T}$, the map $\app_{A,B}$
  corresponds to \[(f,a)\mapsto(a,\app(f,a)).\] Also, 
  the association $b\mapsto\lambda_{A,B}(b)$ for sections
  $\Gamma.A\rightarrow\Gamma.A.B$ corresponds to \emph{lambda abstraction},
  hence $\lambda_{A,B}(b)$ specifies the term $\lambda(a:A).b(a):\Pis(A,B)$.

  \noindent
  The map on the left of the definition of the $\Pies$-rule is the
  \emph{$\eta$-expansion map}, that is it corresponds syntactically to the map
  \[f\mapsto\lambda(a:A).\app(f,a)\] over $\Gamma$, thus
  the property means that we have a judgement
  \[\Gamma\vdash f\equiv\lambda(a:A).\app(f,a):\Pis(A,B)\]
  which corresponds to $\eta$-conversion.

  \noindent
  The map $(1_{p^*_{\Pis(A,B)}A},\app_{A,B})$ is specified by the following
  factorization through the pullback.
  \[\begin{tikzcd}[column sep=huge]
    {\Gamma.\Pis(A,B).p^*_{\Pis(A,B)}A} \\
    & {\Gamma.\Pis(A,B).p^*_{\Pis(A,B)}A.p^*_{\Pis(A,B)}B} & {\Gamma.A.B} \\
    & {\Gamma.\Pis(A,B).p^*_{\Pis(A,B)}A} & {\Gamma.A}
    \arrow["{q(p_{\Pis(A,B)},A)}"', from=3-2, to=3-3]
    \arrow["{p_A}", from=2-3, to=3-3]
    \arrow["{q(p_{\Pis(A,B)},A.B)}"', from=2-2, to=2-3]
    \arrow["{p_{p^*_{\Pis(A,B)}A}}", from=2-2, to=3-2]
    \arrow["{\app_{A,B}}", curve={height=-12pt}, from=1-1, to=2-3]
    \arrow[curve={height=12pt}, Rightarrow, no head, from=1-1, to=3-2]
    \arrow["{(1_{p^*_{\Pis(A,B)}A},\app_{A,B})}"{description}, dotted, from=1-1, to=2-2]
  \end{tikzcd}\]
\end{rmk}

\begin{construction}\label{idterm}
  Let $\sfC$ be a contextual category with a $\Pis$-structure. We define for any
  object $\Gamma.A$ a map $\id_A\colon\Gamma\rightarrow\Gamma.\Pis(A,p^*_A)$ as
  $\lambda_{A,p^*_AA}(1_A,1_A)$.
\end{construction}

\begin{construction}\label{applyf}
  Let $\sfC$ be a contextual category with a $\Pis$-structure,
  $f\colon\Gamma.A\rightarrow\Gamma.B$ a map over $\Gamma$. We want to provide
  a section $\hat{f}\colon\Gamma\rightarrow\Gamma.\Pis(A,B)$ corresponding to
  $f$. We do so by looking at the commutative diagram
  \[\begin{tikzcd}
    {\Gamma.A} \\
    & {\Gamma.A.p^*_AB} & {\Gamma.B} \\
    & {\Gamma.A} & \Gamma
    \arrow["f", curve={height=-12pt}, from=1-1, to=2-3]
    \arrow[curve={height=12pt}, Rightarrow, no head, from=1-1, to=3-2]
    \arrow["{(1_A,f)}"{description}, dotted, from=1-1, to=2-2]
    \arrow["{p_B}", from=2-3, to=3-3]
    \arrow["{p_A}"', from=3-2, to=3-3]
    \arrow["{q(p_A,B)}"', from=2-2, to=2-3]
    \arrow["{p_{p^*_AB}}", from=2-2, to=3-2]
  \end{tikzcd},\]
  and then taking $\lambda_{A,p^*_AB}(1_A,f)$.

\noindent
  In $\Syn{T}$, using
  \begin{align*}
    q(p_A,B)\cdot
    \app_{A,B}\cdot
    q(p_{p^*_AB},\lambda_{A,p^*_AB}(1_A,f))\cdot
    a
    &=q(p_A,B)\cdot
    \app_{A,B}(\lambda_{A,p^*_AB}(1_A,f),a) \\
    &=q(p_A,B)\cdot
    (1_A,f)\cdot
    a \\
    &=f\cdot
    a,
  \end{align*}
  we see $f$ acting as
  \begin{align*}
    a \\
    ^{q(p_{p^*_AB},\lambda_{A,p^*_AB}(1_A,f))}&\mapsto(a,\hat{f}) \\
    ^{\app_{A,B}}&\mapsto(a,\app(\hat{f},a)) \\
    ^{q(p_A,B)}&\mapsto\app(\hat{f},a).
  \end{align*}
  This justifies abusing the notation to write $f$ for $\hat{f}$ and $(1_A,f)$
  then acts as $a\mapsto(a,\app(f,a))$, meaning that $\lambda_{A,p^*_AB}(1_A,f)$
  corresponds to $\lambda(a:A).\app(f,a):\Pis(A,B)$ in $\Syn{T}$.
\end{construction}

\begin{construction}\label{postcomp}
  Let $\sfC$ be a contextual category with a $\Pis$-structure,
  $f\colon\Gamma.A.B\rightarrow\Gamma.A.B'$ a morphism over
  $\Gamma.A$, meaning that in $\Syn{T}$ it is a map $b\mapsto\app(f,b)$ for the
  term $f:\Pis(B,B')$ it induces. We want to construct a morphism
  $$\Gamma.\Pis(A,f)\colon\Gamma.\Pis(A,B)\rightarrow\Gamma.\Pis(A,B')$$
  modeling the postcomposition by $f$, that is
  $g\mapsto\lambda(a:A).\app(f,\app(g,a))$.

\noindent
  We do so by looking at the commutative diagram
  \[\begin{tikzcd}[column sep=huge]
    {\Gamma.\Pis(A,B).p^*_{\Pis(A,B)}A} && {\Gamma.A.B} \\
                                          &
    {\Gamma.\Pis(A,B).p^*_{\Pis(A,B)}A.p^*_{\Pis(A,B)}B'} & {\Gamma.A.B'} \\
    & {\Gamma.\Pis(A,B).p^*_{\Pis(A,B)}A} & {\Gamma.A}
    \arrow["{q(p_{\Pis(A,B)},A)}"', from=3-2, to=3-3]
    \arrow["{p_{B'}}"', from=2-3, to=3-3]
    \arrow["{q(p_{\Pis(A,B)},A.B')}"', from=2-2, to=2-3]
    \arrow["{p_{p^*_{\Pis(A,B)}B'}}", from=2-2, to=3-2]
    \arrow[curve={height=12pt}, Rightarrow, no head, from=1-1, to=3-2]
    \arrow["f"', from=1-3, to=2-3]
    \arrow["{p_B}", curve={height=-36pt}, from=1-3, to=3-3]
    \arrow["{\app_{A,B}}", from=1-1, to=1-3]
    \arrow["{(1_{p^*_{\Pis(A,B)}A},f\cdot\app_{A,B})}"{pos=1}, from=1-1, to=2-2, dotted]
  \end{tikzcd}\]
  and then applying $\lambda_{p^*_{\Pis(A,B)}A,p^*_{\Pis(A,B)}B'}$ to the map
  given by the universal property of the pullback, which provides us with the
  section
  \[\lambda_{p^*_{\Pis(A,B)}A,p^*_{\Pis(A,B)}B'}
  (1_{p^*_{\Pis(A,B)}A},
  f\cdot\app_{A,B})\colon
  \Gamma.\Pis(A,B)\rightarrow
  \Gamma.\Pis(A,B).\Pis(A,B').\]
  All we have to do now is postcompose with
  $q(p_{\Pis(A,B)},\Pis(A,B'))$. Verifying that in $\Syn{T}$ the action on
  sections is the one we want is straightforward as we have done so far.
\end{construction}

\begin{rmk}
  The previous construction is such that $\Gamma.\Pis(A,1_B)=1_{\Pis(A,B)}$ for
  every context $\Gamma.A.B$ in $\sfC$ if and only if the $\Pis$-structure
  satisfies the $\Pies$-rule.
\end{rmk}

\begin{notation}
  As noted before, contextual categories with some extra logical structures
  still are models of small essentially algebraic theory with sorts indexed by
  $\bbN+\bbN\times\bbN$. Their categories of models are then specified in the
  literature by writing as subscript of $\Cxl$ the corresponding structures,
  that is we write $\Cxl_{\Ids}$, $\Cxl_{\Sigmas,\Ids}$ and so on.
\end{notation}
