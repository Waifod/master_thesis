\chapter{Localizations of \texorpdfstring{$\infty$}{∞}-Categories}

To prove that by localizing a syntactic category of a dependent type theory with
some logical rules we get a locally
cartesian closed $\infty$-category we need a theory of localizations
of $\infty$-categories, which in our case will be quasi-categories. We shall
provide then such a theory as
developed by Cisinski \cite{Cis19} with the aim of
proving Theorem \ref{7616}, which will do the heavy lifting in showing
the desired result. Those familiar with the theory may skip the entire chapter
while keeping in mind the aforementioned theorem, while those who do not know it
may read it for a quick tour. We recommend however to read the original source,
from which come all of the materials here presented.

Our exposition is divided in two parts: first we define localizations and show
some results derived directly from it, while later we present some results which
allow us to give a better condition of the localization when the starting
$\infty$-category has a fibrational structure. The latter generalizes
previous work by Szumilo and Kapulkin \cite{Szu14,KS15,Kap17}, who proved
similar results for the simplicial localizations of \emph{fibration categories}
by studying the associated \emph{$\infty$-categories of frames}}, which provide
concrete models, thereby avoiding fibrant replacements.

\section{Universal Property}

Localizations are defined in any context by a universal property of a certain
form. Here we present ours and see what we can derive from it without extra
assumptions.

\begin{defn}
  Let $C$ be a simplicial set and $W\subset C$ a simplicial subset. Given an
  $\infty$-category $\cD$, we define $\home_W(C,\cD)$ to be the full simplicial
  subset of $\home(C,\cD)$ whose objects are the morphisms $f\colon
  C\rightarrow\cD$ sending the 1-simplices in $W$ to isomorphisms.
\end{defn}

\begin{rmk}
  The above definition induces a canonical pullback square
  \[\begin{tikzcd}
    {\home_W(C,\cD)} & {\home(C,\cD)} \\
    {\home_W(W,\cD)} & {\home(W,\cD)}
    \arrow[from=1-1, to=2-1]
    \arrow[from=2-1, to=2-2]
    \arrow[from=1-2, to=2-2]
    \arrow[from=1-1, to=1-2]
    \arrow["\lrcorner"{anchor=center, pos=0.125}, draw=none, from=1-1, to=2-2]
  \end{tikzcd}\]
  given by the inclusion $W\rightarrow C$.
\end{rmk}

\begin{defn}
  Given a simplicial set $C$ and a simplicial subset $W$, a \emph{localization
  of $C$ by $W$} is a morphism $\gamma\colon C\rightarrow L(C)$ in $\sSet$ such
  that:
  \begin{enumerate}
    \item $L(C)$ is an $\infty$-category;
    \item $\gamma$ sends the 1-simplices of $W$ to isomorphisms in $L(C)$;
    \item for any $\infty$-category $\cD$ there is an equivalence of
      $\infty$-categories
      \[\home(L(C),\cD)\rightarrow\home_W(C,\cD)\]
      given by precomposing with $\gamma$.
  \end{enumerate}
\end{defn}

\begin{prop}\label{exuniq}
  The localization of $C$ by $W$ always exists and is essentially unique.
\end{prop}
\begin{proof}
  We begin by proving that a localization exists in the case where $W=C$.

  \noindent
  In this context, $\home_W(C,\cD)\cong\home(C,\cD^{\cong})$ canonically,
  where $\cD^{\cong}$ is the maximal subgroupoid of $\cD$.
  Factoring $C\rightarrow\Delta^0$ in the Kan model structure, we find an
  anodyne map $C\rightarrow C'$. We then choose this map for $\gamma$ and set
  $L(C)=C'$.

  \noindent
  Remember that for any anodyne map $A\rightarrow B$ we get a trivial fibration
  $\home(B,\cD^{\cong})\rightarrow\home(A,\cD^{\cong})$. Looking then at the
  commutative diagram
  \[\begin{tikzcd}
    {\Hom(L(C),\cD^{\cong})} & {\Hom_W(C,\cD^{\cong})} \\
    {\home(L(C),\cD)} & {\home_W(C,\cD)}
    \arrow["\gamma^*", "\sim"', two heads, from=1-1, to=1-2]
    \arrow[from=1-1, to=2-1]
    \arrow["{\gamma^*}"', from=2-1, to=2-2]
    \arrow["\cong", from=1-2, to=2-2]
    \arrow["\cong"', from=1-1, to=2-1]
  \end{tikzcd},\]
  we see that the lower $\gamma^*$ is a trivial fibration, thus we have
  constructed a valid localization of $C$ by $W$.
  
  We now move on to the general case. First of all, notice that as a particular
  case of the previous one we get that localizing $\Delta^1$ at its non-trivial
  morphism our construction provides $L(\Delta^1)=J\sim\Delta^0$, while $\gamma$
  is the inclusion $\Delta^1\rightarrow J$. Taking then $W\subset C$, we
  consider the commutative diagram
  \[\begin{tikzcd}
    {\coprod_{f\in W_1}\Delta^1} & C \\
    {\coprod_{f\in W_1}J} & {C'} \\
    && {L(C)}
    \arrow[from=1-1, to=1-2]
    \arrow[from=1-1, to=2-1]
    \arrow[from=2-1, to=2-2]
    \arrow[from=1-2, to=2-2]
    \arrow[from=1-1, to=2-1]
    \arrow["\sim"', from=2-2, to=3-3]
    \arrow["\gamma", curve={height=-12pt}, from=1-2, to=3-3]
    \arrow["\lrcorner"{anchor=center, pos=0.125, rotate=180}, draw=none, from=2-2, to=1-1]
  \end{tikzcd},\]
  where $C'\xrightarrow{\sim} L(C)$ is an inner anodyne map obtained by taking the
  fibrant replacement of $C'$ in the Joyal model structure. This can be done
  functorially via the small object argument.

  \noindent
  For any $\infty$-category $\cD$, we get a trivial fibration
  $\home(L(C),\cD)\dhxrightarrow{\sim}\home(C',\cD)$ and a pullback square
  \[\begin{tikzcd}
    {\home(C',\cD)} & {\home(C,\cD)} \\
    {\Pi_{f\in W_1}\home(J,\cD)} & {\Pi_{f\in W_1}\home(\Delta^1,\cD)}
    \arrow[from=1-2, to=2-2]
    \arrow[from=2-1, to=2-2]
    \arrow[from=1-1, to=2-1]
    \arrow[from=1-1, to=1-2]
  \end{tikzcd},\]
  which together with the pullback
  \[\begin{tikzcd}
    {\home_W(C,\cD)} & {\home(C,\cD)} \\
    {\Pi_{f\in W_1}\home(\Delta^1,\cD^{\cong})} & {\Pi_{f\in W_1}\home(\Delta^1,\cD)}
    \arrow[from=1-2, to=2-2]
    \arrow[from=2-1, to=2-2]
    \arrow[from=1-1, to=2-1]
    \arrow[from=1-1, to=1-2]
  \end{tikzcd}\]
  implies by pasting that
  \[\begin{tikzcd}
    {\home(C',\cD)} & {\home_W(C,\cD)} \\
    {\Pi_{f\in W_1}\home(J,\cD)} & {\Pi_{f\in W_1}\home(\Delta^1,\cD^{\cong})}
    \arrow[from=1-2, to=2-2]
    \arrow[from=1-1, to=1-2]
    \arrow[from=1-1, to=2-1]
    \arrow["\sim", two heads, from=2-1, to=2-2]
    \arrow[from=1-2, to=2-2]
  \end{tikzcd}\]
  is also a pullback and therefore the upper arrow is a trivial fibration.
  Composing it with the other one we get
  $\gamma^*\colon\home(L(C),\cD)\rightarrow\home_W(C,\cD)$, which is then a
  trivial fibration and therefore an equivalence of $\infty$-categories.

  We now move on to proving that the localization is essentially unique. For
  this, we notice that $\gamma^*$ establishes then an isomorphism between
  $\pi_0(k(\home_W(C,-)))$ and $\pi_0(\home(L(C),-))=ho(\sSet)(L(C),-)$
  with respect to the Joyal model structure, thus by Yoneda the pair
  $(L(\cC),\gamma)$ satisfying the universal property is
  unique up to unique isomorphism in $ho(\sSet)$ and up to a contractible space
  of equivalences in $\sSet$.
\end{proof}

\begin{rmk}\label{714}
  Through our construction above, one can always choose $L(C)$ so that $\gamma$
  is a bijection on objects because $C'\rightarrow
  L(C)$ is an inner anodyne extension and therefore a retract of a countable
  composition of sums of pushouts of maps which are the identity on objects,
  that is the inner horn inclusions.
\end{rmk}

\begin{rmk}\label{715}
  7.1.5

  In this context, we may define $\overline{W}$, the saturation of $W$ in $C$,
  as the maximal simplicial subset of $C$
  whose morphisms are the ones which become invertible in $L(C)$, thus we have
  $\overline{W}\cong k(L(C))\times_{L(C)}C$ canonically, meaning that we have
  the following pullback square.
  \[\begin{tikzcd}
    {\overline{W}} & C \\
    {k(L(C))} & {L(C)}
    \arrow[hook, from=2-1, to=2-2]
    \arrow["\gamma", from=1-2, to=2-2]
    \arrow[hook, from=1-1, to=1-2]
    \arrow[from=1-1, to=2-1]
  \end{tikzcd}\]

  \noindent
  We have then inclusions $Sk_1(W)\subset W\subset\overline{W}$ and, for any
  $\infty$-category $\cD$, this induces equalities
  \[\home_{Sk_1(W)}(C,\cD)=\home_{W}(C,\cD)=\home_{\overline{W}}(C,\cD),\]
  implying that $(L(C),\gamma)$ is also the localization of $C$ by $Sk_1(W)$
  and the one by $\overline{W}$. It can be noted that the inclusion
  $\overline{W}\rightarrow C$ is a fibration with respect to the Joyal model
  structure as it is the pullback of one, implying that if $C$ is an is an
  $\infty$-category then so is $\overline{W}$.

  \noindent
  We shall say that $W$ is \emph{saturated} if $W=\overline{W}$.
\end{rmk}

\begin{rmk}
  %7.1.6

  Given an $\infty$-category $\cC$ and a simplicial subset $W$, the functor
  $ho(\cC)\rightarrow ho(L(\cC))$ exhibits $ho(L(\cC))$ as the
  1-categorical localization of $\cC$ at $\Arr(\tau(W))$, as can be seen by
  using the universal property.

  \noindent
  On the other hand, given a 1-category $\cC$ and considered a set of
  morphisms $W$, not necessarily the induced map $L(N(\cC))\rightarrow
  N(L(\cC))$ is an isomorphism. Indeed, $L(N(\cC))$ can have much better
  properties, as can be seen for example from \ref{756}, and in fact localizing
  1-categories after taking their nerves gives every $\infty$-category as shown
  in \cite[Prop.\ 7.3.15]{Cis19}.
\end{rmk}

\begin{rmk}
  %7.1.9

  Given a universe $\bfU$ and a simplicial subset $W$ of a $\bfU$-small
  simplicial set $C$, let $\gamma\colon C\rightarrow L(C)$ be the
  associated localization. Then the functor
  $$\gamma^*\colon\home(L(C)^{\op},\cS)\rightarrow\home(C^{\op},\cS)$$ is
  fully faithful because it is obtained as
  \[\home(L(C)^{\op},\cS)\simeq\home_{W^{\op}}(C^{\op},\cS)\rightarrow\home(C^{\op},\cS).\]
  It also has a left adjoint $\gamma_!$ and a right adjoint $\gamma_*$ as
  shown in \cite[Ch.\ 6]{Cis19}. Full faithfulness then implies that for any
  presheaf $F\colon L(C)^{\op}\rightarrow\cS$ the unit map
  $F\rightarrow\gamma_*\gamma^*(F)$ is invertible and, by adjunction, the same
  goes for the counit map $\gamma_!\gamma^*(F)\rightarrow F$.

  \noindent
  Furthermore, its essential image consists exactly of those presheafs $F\colon
  C^{\op}\rightarrow\cS$ such that, for any morphism $u\colon x\rightarrow y$ in
  $W$, the map $Fu\colon Fy\rightarrow Fx$ is invertible in $\cS$. Indeed,
  restricting
  $\gamma_!$ and $\gamma_*$ to $\home_{W^{\op}}(C^{\op},\cS)$, we see that they
  are left and right adjoint to the equivalence of $\infty$-categories induced
  by $\gamma^*$, meaning that the counit map $\gamma^*\gamma_*(F)\rightarrow F$
  and the unit map $F\rightarrow\gamma^*\gamma_!(F)$ are invertible.
\end{rmk}

\begin{prop}\label{7110}
  Given a simplicial set $C$ and a simplicial subset $W$, the localization
  functor $\gamma\colon C\rightarrow L(C)$ is final and cofinal. In
  particular, if $e\colon\Delta^0\rightarrow C$ encodes a final or a cofinal
  object, so does $\gamma(e)$.
\end{prop}
\begin{proof}
  First of all, the functor $\gamma^{\op}$ is also a localization, so it
  suffices to prove that $\gamma$ is final. To do this, first we fix a universe
  $\bfU$ such that $C$ is $\bfU$-small and then we remember that there
  is an adjunction
  $\adjunction{\gamma^*}{\home(L(C),\cS)}{\home(C,\cS)}{\gamma_*}$ and, by the
  above remark, the unit of the
  adjunction is invertible, hence $1\cong\gamma_*\gamma^*$. This gives us that
  \[\lim_{C}F\cong\lim_{C}\gamma_*\gamma^*(F)\cong\lim_{L(C)}\gamma^*(F),\]
  for any presheaf $F\colon C\rightarrow\cS$, which is enough to prove that
  $\gamma$ is final by \cite[Thm.\ 6.4.5]{Cis19}.
\end{proof}

\begin{prop}
  %7.1.11

  Let's fix a universe $\bfU$, a $\bfU$-small simplicial set $C$ and a
  simplicial subset $W$. Consider a morphism $f\colon C\rightarrow\cD$,
  where $\cD$ is a small $\infty$-category. Then $f$ exhibits $\cD$ as the
  localization of $C$ by $W$ if and only if the following conditions hold:
  \begin{enumerate}
    \item the morphism $f$ sends the 1-simplices of $W$ to invertible maps of
      $\cD$;
    \item the morphism $f$ is essentially surjective;
    \item the morphism $f^*$ induces an equivalence of $\infty$-categories
      \[f^*\colon\home(\cD^{\op},\cS)\rightarrow\home_{W^{\op}}(C^{\op},\cS).\]
  \end{enumerate}
\end{prop}
\begin{proof}
  If $f$ is a localization then the conditions are satisfied (for (2) look at
  the construction in Prop.\ \ref{exuniq}).
  For the converse, let's pick a localization $\gamma\colon C\rightarrow
  L(C)$ and, through condition (1), we get a factorization $g\colon
  L(C)\rightarrow\cD$ such that $g\cdot\gamma\cong f$, giving us a triangle
  \[\begin{tikzcd}[column sep=tiny]
    {\home(\cD^{\op},\cE)} && {\home(L(C)^{\op},\cE)} \\
                           & {\home_{W^{\op}}(C^{\op},\cE)}
    \arrow["{g^*}", from=1-1, to=1-3]
    \arrow["{f^*}"', from=1-1, to=2-2]
    \arrow["{\gamma^*}", from=1-3, to=2-2]
  \end{tikzcd}\]
  commuting up to $J$-homotopy for any $\infty$-category $\cE$. Picking
  $\cE=\cS$, $\gamma^*$ and $f^*$ are equivalences
  of $\infty$-categories, the latter by (3). It follows by 2-out-of-3 that $g^*$
  is one too, therefore the same applies to its left adjoint $g_!$, which is
  then fully faithful. This is equivalent to $g$ being fully
  faithful by \cite[Prop.\ 6.1.15]{Cis19} and, since
  $f$ is essentially surjective by (2), the same goes for $g$. It follows that
  $g$ is an equivalence of $\infty$-categories. In the above triangle $g^*$ is
  then an equivalence for any choice of $\cE$ and the same applies to $f^*$ by
  2-out-of-3. We conclude by using $(-)^{\op}$.
\end{proof}

\begin{prop}\label{7114}
  %7.1.14

  Let $f\colon\cC\rightarrow\cD$ be a functor between $\infty$-categories with
  a right adjoint $g\colon\cD\rightarrow\cC$ and suppose that we are given
  simplicial subsets $V\subset\cC$, $W\subset\cD$ such that $f(V)\subset W$,
  $g(W)\subset V$. We can lift them to an adjunction
  $\adjunction{\overline{f}}{L(\cC)}{L(\cD)}{\overline{g}}$ such that the
  diagrams
  \[\begin{tikzcd}
    \cC & \cD \\
    {L(\cC)} & {L(\cD)}
    \arrow["{\gamma_\cD}", from=1-2, to=2-2]
    \arrow["{\gamma_\cC}"', from=1-1, to=2-1]
    \arrow["{\overline{f}}"', from=2-1, to=2-2]
    \arrow["f", from=1-1, to=1-2]
  \end{tikzcd},
  \quad
  \begin{tikzcd}
    \cD & \cC \\
    {L(\cD)} & {L(\cC)}
    \arrow["{\gamma_\cC}", from=1-2, to=2-2]
    \arrow["{\gamma_\cD}"', from=1-1, to=2-1]
    \arrow["{\overline{g}}"', from=2-1, to=2-2]
    \arrow["g", from=1-1, to=1-2]
  \end{tikzcd}
  \]
  commute up to $J$-homotopy.
\end{prop}
\begin{proof}
  Let's write $\home_V^W(\cC,\cD)$ for the full subcategory of $\home(\cC,\cD)$
  whose objects are functors $\phi$ such that $\phi(V)\subset W$. The
  equivalence
  $\gamma^*_\cC\colon\Hom(L(\cC),L(\cD))\rightarrow\home_V(\cC,L(\cD))$ allows
  us to construct a functor
  $\home^W_V(\cC,\cD)\rightarrow\home_V(\cC,L(\cD))\rightarrow\home(L(\cC),L(\cD))$
  which associates to any $\phi$ as above a functor $\overline{\phi}$ making the
  square
  \[\begin{tikzcd}
    \cD & \cC \\
    {L(\cD)} & {L(\cC)}
    \arrow["{\gamma_\cC}", from=1-2, to=2-2]
    \arrow["{\gamma_\cD}"', from=1-1, to=2-1]
    \arrow["{\overline{\phi}}"', from=2-1, to=2-2]
    \arrow["\phi", from=1-1, to=1-2]
  \end{tikzcd}\]
  commute up to $J$-homotopy.

  \noindent
  The proof works by observing that our map also lifts natural transformations
  functorially, which allows us to show the triangle identities for the lifted
  unit and counit.
\end{proof}

\begin{prop}
  %7.1.18

  Let $u\colon\cC\rightarrow\cD$ be a functor between $\infty$-categories with a
  fully faithful right adjoint $v$ and consider $W=k(\cD)\times_\cD\cC$, the
  subcategory of maps of $\cC$ which become invertible in $\cD$. Then $u$
  exhibits $\cD$ as the localization of $\cC$ by $W$.
\end{prop}
\begin{proof}
  Given a localization $\gamma\colon\cC\rightarrow L(\cC)$ by $W$, we get a
  functor $\gamma\cdot v\colon\cD\rightarrow L(\cC)$ which, paired with the
  $\overline{u}$ obtained from the construction in the previous proof, lifts the
  adjunction $u\dashv v$ to the localizations (where $L(\cD)=\cD$ as we
  localize at the identities). Lifting maintains the counit invertible, which
  allows us to conclude that $\gamma\cdot v$ is fully faithful.

  Essential surjectivity
  follows from the fact that, for any object $c$ in $\cC$, the unit $\eta_c$ is
  such that $\epsilon_{u(c)}\cdot u(\eta_c)=1_{u(c)}$ and, since $\epsilon$ is
  invertible, so is $u(\eta_c)$, thus $\eta_c$ becomes invertible in $L(\cC)$
  and shows that $(\gamma_{\cC}\cdot v)(u(c))=\gamma_{\cC}(vu(c))\cong c$.
  Notice that here we used that $L(\cC)_0=\cC_0$, which is permissible up to
  equivalence as noted in Remark \ref{714}.

  \wfd{(PLEASE CHECK PROOF)}
\end{proof}

%\begin{cor}
%  7.2.5
%\end{cor}
%\begin{proof}
%  7.1.5
%\end{proof}

%\begin{cor}
%  7.2.8
%\end{cor}
%\begin{proof}
%  7.1.5, 7.1.18, 7.2.5
%\end{proof}

%\begin{cor}
%  7.2.10
%\end{cor}

%\begin{cor}
%  7.2.15
%\end{cor}

%\begin{cor}
%  7.2.16
%\end{cor}

%\begin{cor}
%  7.2.18
%\end{cor}
%\begin{proof}
%  7.2.16
%\end{proof}

%\begin{cor}
%  7.2.25
%\end{cor}
%\begin{proof}
%  7.2.8, 7.2.10, 7.2.15
%\end{proof}

%\begin{cor}
%  7.3.2
%\end{cor}

%\begin{cor}
%  7.3.4
%\end{cor}

%\begin{cor}
%  7.3.5
%\end{cor}
%\begin{proof}
%  7.1.10, 7.3.2, 7.3.4
%\end{proof}

%\begin{cor}
%  7.3.8
%\end{cor}
%\begin{proof}
%  7.1.10, 7.1.12
%\end{proof}

%\begin{cor}
%  7.3.9
%\end{cor}
%\begin{proof}
%  7.3.8
%\end{proof}

%\begin{cor}
%  7.3.10
%\end{cor}
%\begin{proof}
%  7.3.9
%\end{proof}

%\begin{prop}
%  7.3.11
%\end{prop}

%\begin{cor}
%  7.3.15
%\end{cor}
%\begin{proof}
%  7.1.12, 7.3.9, 7.3.10
%\end{proof}

%\begin{cor}
%  7.3.16
%\end{cor}
%\begin{proof}
%  7.3.5, 7.3.15
%\end{proof}

%\begin{cor}
%  7.3.26
%\end{cor}

%\begin{cor}
%  7.3.27
%\end{cor}
%\begin{proof}
%  7.3.16, 7.3.26
%\end{proof}

%\begin{cor}
%  7.3.28
%\end{cor}

%\begin{cor}
%  7.3.29
%\end{cor}
%\begin{proof}
%  7.3.27, 7.3.28
%\end{proof}

%\begin{prop}
%  7.4.2
%\end{prop}

%\begin{prop}
%  7.4.11
%\end{prop}
%\begin{proof}
%  7.3.11, 7.4.2
%\end{proof}

\section{Localizations of Fibrational Structures}

\begin{defn}
  An \emph{$\infty$-category with weak equivalences and fibrations} is a triple
  $(\cC,W,\fib)$ where $\cC$ is an $\infty$-category with a final object,
  $W\subset\cC$ is a subcategory with the 2-out-of-3 property and
  $\fib\subset\cC$ a subsimplicial set such that:
  \begin{enumerate}
    \item for any morphism $p\colon x\rightarrow y$ in $\fib$ (and $W$) with $y$
      fibrant, there is in $\cC$ a pullback square
      \[\begin{tikzcd}
        {x'} & x \\
        {y'} & y
        \arrow["p", from=1-2, to=2-2]
        \arrow["{p'}"', from=1-1, to=2-1]
        \arrow["v"', from=2-1, to=2-2]
        \arrow["u", from=1-1, to=1-2]
      \end{tikzcd}\]
      where $p'$ also lies in $\fib$ (and $W$);
    \item for any map $f\colon x\rightarrow y$ with fibrant codomain can be
      factored as a map in $W$ followed by one in $\fib$.
  \end{enumerate}
  By \emph{fibrant object} we mean an object whose map to the terminal one is in
  $\fib$.

  We shall call \emph{weak equivalences} the maps in $W$ and \emph{fibrations}
  the ones in $\fib$. Maps which are both shall be referred to as \emph{trivial
  fibrations}.
\end{defn}

\begin{construction}\label{fincomplarefib}
  Any finitely complete $\infty$-category $\cC$ can be given the structure of an
  $\infty$-category with weak equivalences and fibrations by setting $W=k(\cC)$,
  $\fib=\cC$, which we will be doing henceforth.
\end{construction}

\begin{construction}
  For any $\infty$-category with weak equivalences and fibrations $\cC$ and a
  fibrant object $c$, we can give to the slice category $\cC/c$ the structure of an
  $\infty$-category with weak equivalences and fibrations by specifying as weak
  equivalences the morphisms which are mapped to weak equivalences of $\cC$ by
  the projection $\cC/c\rightarrow\cC$ and similarly for the fibrations.
\end{construction}

\begin{defn}
  An $\infty$-category of fibrant objects is an $\infty$-category with weak
  equivalences and fibrations $\cC$ in which all objects are fibrant.
\end{defn}

\begin{construction}
  For any $\infty$-category with weak equivalences and fibrations $\cC$, its
  full subcategory given by fibrant objects is canonically an $\infty$-category
  of fibrant objects. We shall denote it by $\cC_f$ and its weak equivalences
  are given by $W_f=W\cap\cC_f$, its fibrations by $\fib_f=\fib\cap\cC_f$.
\end{construction}

\begin{lem}[Brown's Lemma]\label{7413}
  7.4.13

  For any map $f\colon x\rightarrow y$ between fibrant objects in an
  $\infty$-category with weak equivalences and fibrations $\cC$, there exists a
  commutative diagram of the form
  \[\begin{tikzcd}
    & x \\
    x & z & y
    \arrow[Rightarrow, no head, from=1-2, to=2-1]
    \arrow["s", from=1-2, to=2-2]
    \arrow["p", "\sim"', two heads, from=2-2, to=2-1]
    \arrow["q"', two heads, from=2-2, to=2-3]
    \arrow["f", from=1-2, to=2-3]
  \end{tikzcd},\]
  where $s$ is a weak equivalence, $p$ a trivial fibration and $q$ a fibration.
\end{lem}
\begin{proof}
  Since $x$ and $y$ are fibrant, the pullback of $x\rightarrow e$ and
  $y\rightarrow e$ exists and it corresponds to $x\times y$. The maps $\id_x$,
  $f$ define a cone over our cospan which induces a map $g\colon x\rightarrow
  x\times y$ and we then
  factor the latter as a weak equivalence $s\colon x\rightarrow z$ followed by a
  fibration $\pi\colon z\rightarrow x\times y$. We get then the desired maps
  $p=p_x\cdot\pi$, $q=p_y\cdot\pi$, where $p_x$, $p_y$ denote the projections
  $x\times y\rightarrow x$, $x\times y\rightarrow y$ respectively.
\end{proof}

\begin{cor}\label{7414}
  Let $\cC$ be an $\infty$-category with weak equivalences and fibrations, $\cD$
  an $\infty$-category and $V\subset \cD$ a subcategory with the 2-out-of-3
  property. If $F$ sends trivial fibrations between fibrant objects into $V$,
  then it also sends weak equivalences between fibrant objects into $V$.
\end{cor}
\begin{proof}
  Looking at the commutative diagram
  \[\begin{tikzcd}
    x \\
    {} & z & y \\
    x
    \arrow["s"', from=1-1, to=2-2]
    \arrow["p"', two heads, from=2-2, to=3-1]
    \arrow[Rightarrow, no head, from=1-1, to=3-1]
    \arrow["f"', curve={height=12pt}, from=3-1, to=2-3]
    \arrow["f", curve={height=-12pt}, from=1-1, to=2-3]
    \arrow["q", from=2-2, to=2-3]
  \end{tikzcd},\]
  given by Brown's Lemma \ref{7413} we see that $Fp$ lies in $V$ and therefore
  the same goes for $Fs$. Also, since $f$ and $s$ are weak equivalences we know
  that $q$ is too, hence it is a trivial fibration. It follows that $Fq$ is in
  $V$ and the same goes for $Ff=Fq\cdot Fs$.
\end{proof}

\begin{construction}
  Given an $\infty$-category with weak equivalences and fibrations $\cC$ and a
  fibrant object $z$ in it, we write $\cC(z)$ for $(\cC/z)_f$, that is the
  full subcategory $\cC/z$ given by the fibrations $x\rightarrow z$ of $\cC$; we
  shall refer to $\cC(z)$ as the \emph{fibrant slice of $\cC$ over $z$}.
  For any morphism $f\colon x\rightarrow y$ between fibrant objects, we have a
  left exact functor
  $f^*\colon\cC(y)\rightarrow\cC(x)$ induced pulling back along $f$
  \wfd{(CAREFUL WITH THIS! YOU ARE USING IT LATER ON; CHECK \cite[Prop.\
  7.4.15]{Cis19})}. The existence follows from the fact that pullbacks along
  fibrations with fibrant codomain exist, while left exactness comes from limits
  commuting and weak equivalences being preserved as a consequence of
  \ref{7414}. \wfd{(PERHAPS MORE DETAIL?)}
\end{construction}


%\begin{cor}
%  7.4.14
%\end{cor}
%\begin{proof}
%  7.4.13
%\end{proof}

%\begin{prop}
%  7.4.16
%\end{prop}
%\begin{proof}
%  7.4.13
%\end{proof}

%\begin{prop}
%  7.4.19
%\end{prop}
%\begin{proof}
%  7.4.11
%\end{proof}

\begin{defn}
  Let $\cC$, $\cD$ be $\infty$-categories with weak equivalences and fibrations.
  A functor $F\colon\cC\rightarrow\cD$ is \emph{left exact} if it has the
  following properties:
  \begin{enumerate}
    \item the functor $F$ preserves final objects;
    \item the functor $F$ sends (trivial) fibrations between fibrant objects to
      (trivial) fibrations;
    \item the functor $F$ preserves any pullback square in $\cC$
      \[\begin{tikzcd}
        {x'} & x \\
        {y'} & y
        \arrow["{p'}"', from=1-1, to=2-1]
        \arrow["p", from=1-2, to=2-2]
        \arrow["u", from=1-1, to=1-2]
        \arrow["v"', from=2-1, to=2-2]
      \end{tikzcd}\]
      where $p$ is a fibration and $y$, $y'$ are fibrant objects.
  \end{enumerate}
\end{defn}

\begin{rmk}
  By Brown's Lemma, a left exact functor preserves weak equivalences between
  fibrant objects.
\end{rmk}

\begin{rmk}
  When considering a functor $F$ between finitely complete $\infty$-categories,
  left exactness is equivalent to preserving finite limits.
\end{rmk}

\begin{prop}\label{756}
  7.5.6

  Given an $\infty$-category with weak equivalences and fibrations $\cC$, the
  localization $L(\cC_f)$ has finite limits and the localization functor
  $\cC_f\rightarrow L(\cC_f)$ is left exact. Moreover, for any $\infty$-category
  $\cD$ with finite limits and any left exact functor
  $f\colon\cC_f\rightarrow\cD$, the induced functor $\overline{F}\colon
  L(\cC_f)\rightarrow\cD$ is left exact.
\end{prop}
\begin{proof}
  Maybe do not prove it? It relies on a bunch of results from ch. 7.2, 7.3, 7.4
  which we do not really want to prove.

  7.1.10, 7.2.18, 7.2.25, 7.3.27, 7.4.13, 7.4.16

  We know by \ref{7110} that $L(\cC_f)$ has a final object, hence to show
  completeness it is enough to prove that it also has pullbacks (\cite[Thm.\
  7.3.27]{Cis19}). This can be done using the fact that any morphism in
  $L(\cC_f)$ can be seen as a composition $\gamma(p)\cdot\gamma(s)^{-1}$, where
  $s$ is a trivial fibration, for which Cisinski uses the theory of the
  \emph{right calculus of fractions}, and the fact that $\gamma_f$ preserves
  pullbacks along fibrations. The proof also shows us that all pullback squares
  in $L(\cC_f)$ are isomorphic to images of pullback squares in $\cC_f$ in which
  all maps are fibrations.
\end{proof}

\begin{prop}\label{7516}
  7.5.16

  Let $x$ be a fibrant object in an $\infty$-category with weak equivalences and
  fibrations $\cC$. The induced functor
  $\cC_f/\gamma_f(x)\rightarrow\cC/\gamma(x)$ is final.
\end{prop}
\begin{proof}
  We have that $\cC_f/\gamma_f(x)=L(\cC_f)/\gamma_f(x)\times_{L(\cC_f)}\cC_f$
  and $\cC/\gamma(x)=L(\cC)/\gamma(x)\times_{L(\cC)}\cC$ and the functor we are
  considering is induced by $\overline{\iota}\colon L(\cC_f)\rightarrow L(\cC)$.
  
  To prove that it is final, it is sufficient to show that for any object
  $(c,u)$ of $L(\cC)/\gamma(x)$ the coslice $(c,u)\backslash(\cC_f/\gamma_f(x))$ is
  weakly contractible and, to do this, by \cite[Lem.\ 4.3.15]{Cis19} we can show
  that any functor $F\colon E\rightarrow (c,u)\backslash(\cC_f/\gamma_f(x))$, where $E$
  is the nerve of a finite partially ordered set, is $\Delta^1$-homotopic to a
  constant functor. This can be done through the theory of Reedy fibrant
  diagrams developed in \cite[Ch.\ 7.4]{Cis19}.
\end{proof}

\begin{prop}\label{7517}
  7.5.17

  Let $\bfU$ be a universe and $\cC$ a $\bfU$-small $\infty$-category with weak
  equivalences and fibrations. For any $\infty$-category $\cD$ with $\bfU$-small
  colimits and any functor $F\colon\cC\rightarrow\cD$, we have an isomorphism
  \[(\gamma_f)_!\iota^*(F)\cong\overline{\iota}^*\gamma_!(F)\]
  induced by the square
  \[\begin{tikzcd}
    {\cC_f} & \cC \\
    {L(\cC_f)} & {L(\cC)}
    \arrow["\gamma", from=1-2, to=2-2]
    \arrow["{\gamma_f}"', from=1-1, to=2-1]
    \arrow["\iota", from=1-1, to=1-2]
    \arrow["{\overline{\iota}}"', from=2-1, to=2-2]
  \end{tikzcd},\]
  which commutes up to $J$-homotopy.
\end{prop}
\begin{proof}
  We only need to prove that the evaluation of the canonical map
  $(\gamma_f)_!\iota^*(F)\cong\overline{\iota}^*\gamma_!(F)$ at any object $x$
  of $\cC_f$ is invertible. This evaluation is equivalent by \cite[Prop.\
  6.4.9]{Cis19} to the map
  \[\colim_{\cC_f/\gamma_f(x)}i^*(F)/\gamma_f(x)\rightarrow\colim_{\cC/\gamma(x)}F/\gamma(x),\]
  where $F/\gamma(x)$ is define by composing $F$ with the canonical projection
  $\cC/\gamma(x)\rightarrow\cC$ and similarly for $i^*(F)/\gamma_f(x)$. Using
  \ref{7516} and the commutativity of the square above, we get that the desired
  map is indeed invertible for all $x$.
\end{proof}

\begin{prop}\label{7518}
  7.5.18

  Let $\cC$ be an $\infty$-category with weak equivalences and fibrations. The
  canonical functor $\overline{\iota}\colon L(\cC_f)\rightarrow L(\cC)$ is an
  equivalence of $\infty$-categories, hence the $\infty$-category $L(\cC)$ is
  finitely complete and the localization functor $\gamma\colon\cC\rightarrow
  L(\cC)$ is left exact.
\end{prop}
\begin{proof}
  7.5.6, 7.5.17

  We already know that $\overline{\iota}$ is essentially surjective as every
  object in $\cC$ is weakly equivalent to one in $\cC_f$ and the localization
  functors are essentially surjective themselves, thus it is enough to prove
  that it is fully faithful. To do this, we may fix a universe $\bfU$ such that
  $\cC$ is $\bfU$-small and prove that the functor
  \[\overline{\iota}_!\colon\home(L(\cC_f),\cS)\rightarrow\home(L(\cC),\cS)\]
  is fully faithful and use \cite[Prop.\ 6.1.15]{Cis19}. Remember that this full
  faithfulness condition is equivalent to the unit map
  $1\rightarrow\overline{\iota}^*\overline{\iota}_!$ of the adjunction
  $\overline{\iota}_!\dashv\overline{\iota}^*$ being invertible.

  We know that $\overline{\iota}_*$ and $\overline{\iota}^*$ both have right
  adjoints, thus they preserve colimits. Also, every $\cS$-valued functor
  indexed by a $\bfU$-small $\infty$-category can be obtained as a colimit of
  representable ones, hence it is enough to check that the condition holds for
  any representable functor $F$. Also, $\gamma_f$ is essentially surjective,
  which means that it is sufficient to check that map
  $(\gamma_f)_!\rightarrow\overline{\iota}^*\overline{\iota}_!(\gamma_f)_!$
  which we get by precomposing the unit with $(\gamma_f)_!$ is invertible.

  We have then the chain of isomorphisms
  \begin{align*}
    (\gamma_f)_! &\cong(\gamma_f)_!\overline{\iota}^*\overline{\iota}_! \\
                 &\cong\overline{\iota}^*\gamma_f\iota_! \\
                 &\cong\overline{\iota}^*\overline{\iota}_!(\gamma_f)_!,
  \end{align*}
  where the first isomorphism comes from the full faithfulness of $\iota$, the
  second one from \ref{7517} and the last one the fact that
  $\overline{\iota}\cdot\gamma_f\cong\gamma\cdot\iota$, as noted in \ref{7517}.

  The second claim follows directly from the first one and \ref{756}.
\end{proof}

\begin{rmk}
  Here we see that the theory of localizations of $\infty$-categories with weak
  equivalences and fibrations provides much better results the 1-categorical
  equivalent, embodied by the homotopy theory of model categories and fibration
  categories (which we will define in the next chapter): indeed, these are
  particular cases of the $\infty$-analogue, however their homotopy categories,
  i.e.\ their 1-categorical localizations by weak equivalences, are almost never
  finitely complete.
\end{rmk}

\begin{cor}\label{7519}
  7.5.19

  Let $\cC$ be an $\infty$-category with weak equivalences and fibrations. For a
  morphism between fibrant objects $p\colon x\rightarrow y$, the following
  conditions are equivalent:
  \begin{enumerate}
    \item the morphism $p$ has a section in $ho(L(\cC))$;
    \item there exists a morphism $p'\colon x'\rightarrow x$ s.t.\ the
      composition of $p'$ and $p$ is a weak equivalence;
    \item there exists a fibration $p'\colon x'\rightarrow x$ s.t.\ the
      composition of $p'$ and $p$ is a weak equivalence.
  \end{enumerate}
\end{cor}
\begin{proof}
  7.5.18

  We see that (iii) trivially implies (ii), therefore we shall focus on the
  other implication. Given then such a morphism $p'$, we factor it as $qi=p'$, a
  weak equivalence followed by a fibration. Since $p\cdot p'=p\cdot(q\cdot
  i)=(p\cdot q)\cdot i$, by 2-out-of-3 $p\cdot q$ is a weak equivalence, giving
  us what we wanted.

  Should we prove (i)? Uses right calculus of fractions, but it's rather simple.
\end{proof}

\begin{construction}\label{7522}
  7.5.22
  
  Given an $\infty$-category $\cC$ with weak equivalences and fibrations, we can
  get another one $\overline{\cC}$ with the same underlying $\infty$-category
  and class of fibrations, but where the weak equivalences are given by the
  saturation $\overline{W}$ as described in \ref{715}. We have that $L(\cC)\cong
  L(\overline{\cC})$, hence in general we can substitute $\cC$ by
  $\overline{\cC}$ with no issues. Also, the substitution commutes with the
  formation of slices over fibrant objects, that is, for any fibrant object $x$
  of $\cC$, a map in $\cC/x$ induces an invertible map in $L(\cC/x)$ if and only
  if its image becomes invertible in $L(\cC)$, which can be seen as a
  consequence of \ref{7519}.
\end{construction}

\begin{rmk}
  Let $\cC$ be an $\infty$-category with weak equivalences $W$ and
  $F\colon\cC\rightarrow\cD$ be a functor. The precomposition functor
  $\gamma^*\colon\home(L(\cC),\cD)\rightarrow\home(\cC,\cD)$ does not have a
  left adjoint in general, but we may ask whether $\Hom(F,\gamma^*(-))$ is
  representable in $\home(L(\cC),\cD)$. If it is, a representative is denoted by
  $\bfR F\colon L(\cC)\rightarrow\cD$ and is called the \emph{right derived
  functor of $F$}. Beware that to be precise one would have to specify the
  natural transformation $F\rightarrow\bfR F\cdot\gamma$ exhibiting it as such.
  Dually, a representative of $\Hom(\gamma^*(-),F)$ is the \emph{left derived
  functor of $F$}.
\end{rmk}

\begin{prop}\label{7524}
  7.5.24

  If $F\colon\cC\rightarrow\cD$ sends weak equivalences to isomorphisms, then
  the functor $\overline{F}\colon L(\cC)\rightarrow\cD$, associated to $F$ by
  the universal property of $L(\cC)$, is the right derived functor of $F$.
\end{prop}
\begin{proof}
  Let's fix a universe $\bfU$ such that $\cC$ and $\cD$ are $\bfU$-small and let
  $G\colon L(\cC)\rightarrow\cD$ be any functor. Then the invertible map
  $\overline{F}\cdot\gamma\cong F$ and the equivalence of $\infty$-categories
  $\home(L(\cC),\cD)\simeq\home_W(\cC,\cD)$ induce invertible maps
  $\Hom(\overline{F},G)\simeq\Hom(\overline{F}\cdot\gamma,G\cdot\gamma)\simeq\Hom(F,G\cdot\gamma)$
  in $\cS$, functorially in $G$.
\end{proof}

\begin{construction}\label{7525}
  \wfd{(NOT COMPLETE, ONE MAY SHOW THAT OUR CONSTRUCTION DOES GIVE THE RIGHT
  DERIVED FUNCTOR)}

  Let $\cC$ be an $\infty$-category with weak equivalences and fibrations. Any
  functor $F\colon\cC\rightarrow\cD$ sending weak equivalences between fibrant
  objects to invertible maps then has a right derived functor $\bfR F$, which
  may be constructed as follows.

  First we choose a quasi-inverse $R\colon L(\cC)\rightarrow L(\cC_f)$ of the
  equivalence of $\infty$-categories specified in \ref{7518}, then we pick a
  functor $\overline{F}\colon L(\cC_f)\rightarrow\cD$ and a natural isomorphism
  $j\colon\overline{F}\cdot\gamma_f\rightarrow F\cdot\iota$. We set then $\bfR
  F=\overline{F}\cdot R$.

  What we are doing in this construction is selecting
  for every object in $\cC$ a fibrant replacement, exactly like when we talk
  about right derived functors in the context of model categories. This is
  necessary because, a priori, we are not sending all weak equivalences to
  invertible maps in $\cD$, hence we would have to show that before applying the
  universal property of localizations. Also, for any
  other functor $G\colon\cD\rightarrow\cE$, we have that $G\cdot\bfR
  F=\bfR(G\cdot F)$.
\end{construction}

\begin{defn}
  Given an $\infty$-category with weak equivalences and fibrations $\cC$ and an
  $\infty$-category with weak equivalences $\cD$, let's consider a functor
  $F\colon\cC\rightarrow\cD$ preserving weak equivalences between fibrant
  objects of $\cC$. We call the \emph{right derived functor of} $F$ the right
  derived functor of the composition
  \[\cC\xrightarrow{F}\cD\xrightarrow{\gamma_{\cD}}L(\cD),\]
  where $\gamma_\cD$ is the localization functor of of $\cD$ at its weak
  equivalences. This right derived functor of $F$ is denoted by $\bfR F$, that
  is $\bfR F=\bfR(\gamma_{\cD}\cdot F)\colon L(\cC)\rightarrow L(\cD)$, which
  makes sense since we can apply the construction \ref{7525}.
\end{defn}

There are some interesting remarks which may be included!!!!

\begin{prop}\label{7528}
  7.5.28

  For any left exact functor $F\colon\cC\rightarrow\cD$ between
  $\infty$-categories with weak equivalences and fibrations, the right derived
  functor $\bfR F\colon L(\cC)\rightarrow L(\cD)$ is left exact.
\end{prop}
\begin{proof}
  7.5.6

  We have a square
  \[\begin{tikzcd}
    {L(\cC_f)} & {L(\cD_f)} \\
    {L(\cC)} & {L(\cD)}
    \arrow["{\overline{F}}", from=1-1, to=1-2]
    \arrow[from=1-2, to=2-2]
    \arrow[from=1-1, to=2-1]
    \arrow["{\bfR F}"', from=2-1, to=2-2]
  \end{tikzcd}\]
  commuting up to $J$-homotopy, where the vertical maps are equivalences of
  $\infty$-categories and $\overline{F}$ is the functor obtained by restricting
  $F$ to the subcategories of fibrant objects $\cC_f$ and $\cD_f$. It therefore
  suffices to show that $\overline{F}$ is left exact, but this follows from
  \ref{756}.
\end{proof}

\begin{rmk}
  \wfd{(WHY DO WE NEED TO SPECIFY THIS?)}

  For the remainder of this chapter, given an $\infty$-category $\cC$,
  subcategories of weak equivalences $W\subset\cC$ are such that the inclusion
  $W\rightarrow\cC$ is an inner fibration. This means that a simplex
  $x\colon\Delta^n\rightarrow\cC$ lies in $W$ if and only if its edges
  $x|_{\Delta^{\{i,i+1\}}}\colon \{i,i+1\}\rightarrow\cC$ lie in $W$ for $0\leq
  i<n$.

  $W$ then contains all invertible maps of $\cC$ if and only if the
  aforementioned inclusion is an isofibration.
\end{rmk}

\begin{defn}
  Let $\cC$, $\cD$ be $\infty$-categories with subcategories of weak
  equivalences $W\subset\cC$, $W'\subset\cD$. A functor
  $f\colon\cC\rightarrow\cD$ has the \emph{right approximation property} if the
  following conditions hold:
  \begin{enumerate}
    \item a morphism in $\cC$ is in $W$ if and only if its image under $f$ is in
      $W'$;
    \item given objects $c$, $d$ in $\cC$, $\cD$ respectively and a map
      $\psi\colon d\rightarrow f(c)$ in $\cD$, there is a map $\phi\colon
      c'\rightarrow c$ in $\cC$ and a weak equivalence $u\colon d\rightarrow
      f(c')$ in $\cD$ such that  the triangle
      \[\begin{tikzcd}
        d & {f(c)} \\
        {f(c')}
        \arrow["\psi", from=1-1, to=1-2]
        \arrow["u"', from=1-1, to=2-1]
        \arrow["{f(\phi)}"', from=2-1, to=1-2]
      \end{tikzcd}\]
      commutes.
  \end{enumerate}
\end{defn}

\begin{prop}\label{762}
  7.6.2

  A functor $f\colon\cC\rightarrow\cD$ between $\infty$-categories such that the
  induced functor on the homotopy categories $ho(f)\colon ho(\cC)\rightarrow
  ho(\cD)$ is an equivalence of categories has the right approximation property.
\end{prop}
\begin{proof}
  Consider a map $\psi\colon d\rightarrow f(c)$. Since $ho(f)$ is essentially
  surjective, there exists an invertible map $d\rightarrow ho(f)(c')=f(c')$ in
  $ho(\cD)$, which comes from an invertible map $d\rightarrow f(c')$ of $\cD$.
  In $ho(\cD)$ we can then complete this to a triangle
  \[\begin{tikzcd}
    d & {f(c)} \\
    {f(c')}
    \arrow["{[\psi]}", from=1-1, to=1-2]
    \arrow["\sim"', from=1-1, to=2-1]
    \arrow["{[\phi]}"', from=2-1, to=1-2]
  \end{tikzcd}\]
  and, since $ho(f)$ is fully faithful, $\phi$ can be lifted to
  $\tilde{\phi}\colon c'\rightarrow c$ in $ho(\cC)$. This gives us a commutative
  triangle
  \[\begin{tikzcd}
    d & {f(c)} \\
    {f(c')}
    \arrow["\psi", from=1-1, to=1-2]
    \arrow["\sim"', from=1-1, to=2-1]
    \arrow["{f(\tilde{\phi})}"', from=2-1, to=1-2]
  \end{tikzcd}\]
  in $\cD$.
\end{proof}

\begin{exmp}
  Given an $\infty$-category with weak equivalences and fibrations $\cC$, the
  inclusion $\cC_f\rightarrow\cC$ has the right approximation property.
\end{exmp}

\begin{exmp}\label{764}
  Given a saturated $\infty$-category with weak equivalences and fibrations
  $\cC$, the localization functor $\cC\rightarrow L(\cC)$ has the right
  approximation property (\cite[Ex.\ 7.6.4]{Cis19}).
\end{exmp}

%\begin{lem}
%  7.6.5
%\end{lem}
%\begin{proof}
%  4.2.9
%\end{proof}

%\begin{lem}
%  7.6.7
%\end{lem}
%\begin{proof}
%  4.3.15, 7.4.19, 7.5.5
%\end{proof}

\begin{thm}\label{7610}
  7.6.10

  Let $f\colon\cC\rightarrow\cD$ be a functor between $\infty$-categories with
  finite limits. If $f$ commutes with them, then the following conditions are
  equivalent:
  \begin{enumerate}
    \item the functor $f$ is an equivalence of $\infty$-categories;
    \item the functor $ho(f)\colon\cC\rightarrow\cD$ is an equivalence of
      categories;
    \item the functor $f$ has the right approximation property.
  \end{enumerate}
\end{thm}
\begin{proof}
  7.3.29, 7.6.2, 7.6.5, 7.6.7

  We trivially have that (i) implies (ii) and, by \ref{762}, (iii) follows from
  (ii), hence we only have to show that (iii) gives (i). Let's assume then that
  $f$ has the right approximation property.

  Given a final object $e$ of $\cC$, $f(e)$ is still final in $\cD$ by
  \ref{7110}, thus for any object $d$ of $\cD$ we have a map $d\rightarrow f(e)$
  and, by the right approximation property, we get a commutative triangle with
  an isomorphism \wfd{(SPECIFY WHICH STRUCTURE YOU ARE CONSIDERING ON THE
  $\infty$-CATEGORIES)} $d\rightarrow f(c)$ for some $c$ in $\cC$, which gives
  us essential surjectivity.

  We are still missing full faithfulness. To do this, we use that the right
  approximation property implies that we have an equivalence of
  $\infty$-groupoids $k(f)\colon k(\cC)\rightarrow k(\cD)$ (\cite[Lem.\
  7.6.7]{Cis19}) and that, for any object $c$ of $\cC$, the map
  $\cC/c\rightarrow\cD/f(c)$ induced on the slices still has the right
  approximation property (\cite[Prop.\ 7.6.7]{Cis19}), therefore again we get an
  equivalence of $\infty$-groupoids $k(\cC/c)\rightarrow k(\cD/f(c))$.

  Keeping these facts in mind, let's look at the projection
  $\cC/c\rightarrow\cC$. This functor is conservative, thus the square
  \[\begin{tikzcd}
    {k(\cC/c)} & {\cC/c} \\
    {k(\cC)} & \cC
    \arrow[from=1-1, to=2-1]
    \arrow[from=2-1, to=2-2]
    \arrow[from=1-2, to=2-2]
    \arrow[from=1-1, to=1-2]
  \end{tikzcd}\]
  is a pullback. We observe by pasting that the pullback of $k(\cC/c)\rightarrow
  k(\cC)$ along $c'\colon\Delta^0\rightarrow k(\cC)$ is $\cC(c',c)$, as it is
  clear from the diagram
  \[\begin{tikzcd}
    {\cC(x,y)} & {k(\cC/c)} & {\cC/c} \\
    {\Delta^0} & {k(\cC)} & \cC
    \arrow[from=1-2, to=2-2]
    \arrow[from=2-2, to=2-3]
    \arrow[from=1-3, to=2-3]
    \arrow[from=1-2, to=1-3]
    \arrow["{c'}"', from=2-1, to=2-2]
    \arrow[from=1-1, to=2-1]
    \arrow[from=1-1, to=1-2]
  \end{tikzcd}.\]
  In the same way, we get that the pullback of $k(\cD/f(c))\rightarrow k(\cD)$
  along $f(c')\colon\Delta^0\rightarrow k(\cD)$ is $\cD(f(c'),f(c))$. Since we
  have a commutative square
  \[\begin{tikzcd}
    {k(\cC/c)} & {k(\cD/f(c))} \\
    {k(\cC)} & {k(\cD)}
    \arrow[from=1-2, to=2-2]
    \arrow[from=1-1, to=1-2]
    \arrow[from=2-1, to=2-2]
    \arrow[from=1-1, to=2-1]
  \end{tikzcd}\]
  where the horizontal maps are equivalences of $\infty$-groupoids, the
  induced map $\cC(c',c)\rightarrow\cD(f(c'),f(c))$ is again an equivalence of
  $\infty$-groupoids, which is what we wanted.

  Since $f$ is essentially surjective and fully faithful, it is an equivalence
  of $\infty$-categories.
\end{proof}

\begin{cor}\label{7613}
  7.6.13

  Let $\cC$ be an $\infty$-category with weak equivalences and fibrations and
  consider a localization functor $\gamma\colon\cC\rightarrow L(\cC)$. For any
  fibrant object $x$ of $\cC$, the canonical functor $\cC/x\rightarrow
  L(\cC)/\gamma(x)$, $t\mapsto\gamma(t)$, induces an equivalence of
  $\infty$-categories $L(\cC/x)\simeq L(\cC)/\gamma(x)$.
\end{cor}
\begin{proof}
  7.5.18, 7.5.22, 7.5.24, 7.5.28, 7.6.4, 7.6.10

  By \ref{7522}, we can assume that $\cC$ is saturated. Our objective is to show
  that the induced functor $\phi\colon L(\cC/x)\rightarrow L(\cC)/\gamma(x)$ has
  the right approximation property and it preserves finite limits, which will
  allow us to apply \ref{7610} and conclude.

  To show condition (1) we only need to prove that $\phi$ is conservative, which
  can be reduced to showing that a map in $\cC/x$ becomes invertible in
  $L(\cC/x)$ if and only if it becomes an insomorphism in $L(\cC)$. This however
  is true by
  saturation of $\cC$. We still need to check condition (2), which can be done
  on $\cC/x\rightarrow L(\cC)/\gamma(x)$, but this follows from the fact that
  $\gamma$ has it, as mentioned in \ref{764}.

  To apply \ref{7610} we still need to show that $\phi$ preserves limits. To do
  this, we use the fact that $\cC/x$ has the structure of an
  $\infty$-category with weak equivalences and fibrations. Given that $\gamma$
  is left exact, the functor $\cC/x\rightarrow L(\cC)/\gamma(x)$ maps weak
  equivalences to isomorphisms and we can apply \ref{7524} to prove that $\phi$
  is its right derived functor. Finally, through \ref{7528} we get that $\phi$
  is also left exact.
\end{proof}

\begin{thm}\label{7616}
  7.6.16

  Let $\cC$ be an $\infty$-category with weak equivalences and fibrations. Given
  a fibrant object $x$, let $\cC(x)$ be the full subcategory of fibrant objects
  of $\cC/x$ \wfd{(INCLUDE 7.6.12)}, which will be an $\infty$-category of
  fibrant objects. Assume that, for any fibration between fibrant objects
  $p\colon x\rightarrow y$, the pullback functor
  $p^*\colon\cC(y)\rightarrow\cC(x)$, $(y'\rightarrow y)\mapsto(y'\times_y
  x\rightarrow x)$ has a right adjoint $p_*\colon\cC(x)\rightarrow\cC(y)$
  preserving trivial fibrations. Then, for any map $p\colon x\rightarrow y$ in
  $L(\cC)$, the pullback functor $p^*\colon\cC(y)\rightarrow\cC(x)$ has a right
  adjoint.
\end{thm}
\begin{proof}
  6.1.6, 6.1.7, 6.1.8, 7.1.14, 7.4.14, 7.5.18, 7.6.13

  Given a localization functor $\gamma\colon\cC\rightarrow L(\cC)$, one reduces
  the problem to proving that, for any fibration between fibrant objects
  $p\colon x\rightarrow y$, the pullback functor
  \[\gamma(p)^*\colon L(\cC)/\gamma(y)\rightarrow L(\cC)/\gamma(x)\]
  has a right adjoint.

  A consequence of Brown's Lemma \ref{7413} is that any functor preserving
  trivial fibrations between fibrant objects also preserves weak equivalences,
  and, since $p^*\colon\cC(y)\rightarrow\cC(x)$ has both a right and a left
  adjoint named $p_*$ and $p_!$ (the latter given by post-composing with $p$)
  which do preserve them, by \ref{7114} we have a pair of adjunctions on the
  localizations, namely
  \[\adjunction{\overline{p}^*}{L(\cC(y))}{L(\cC(x))}{\overline{p}_*},\]
  \[\adjunction{\overline{p}_!}{L(\cC(x))}{L(\cC(y))}{\overline{p}^*}.\]

  Given that $(\cC/z)_f=\cC(z)$ for all fibrant objects $z$ of $\cC$, by
  \ref{7518} we have that $L(\cC(z))\simeq L(\cC/z)$ and, by $\ref{7613}$, we
  also know that $L(\cC/z)\simeq L(\cC)/\gamma(z)$, hence $L(\cC(z))\simeq
  L(\cC)/\gamma(z)$. Notice that $\overline{p}_!$ is equivalent to
  $\gamma(p)_!\colon L(\cC)/\gamma(x)\rightarrow L(\cC)/\gamma(y)$ and, by
  essential uniqueness of the adjoints, this extends to $\gamma(p)^*$ and
  $\overline{p}^*$, therefore $\gamma(p)^*$ has a right adjoint induced by
  $\overline{p}_*$.
\end{proof}

