\chapter{Localizations of Syntactic Categories}

In this chapter we finally talk about localizing syntactic categories. To do
this, we first specify a class of maps we localize at, explaining the rationale
behind our choice, and then from this we present a fibrational structure which
will allow us to apply the results from the previous section to prove
Theorem \ref{finalthm}. We also briefly mention the \emph{Internal Languages
Conjecture}.

\section{Bi-Invertibility}

As anticipated, we introduce a notion of weak equivalence in our context.

\begin{defn}
  Given a contextual category with $\Ids$-structure $\sfC$, a morphism
  $f\colon\Gamma.A\rightarrow\Gamma.B$ over $\Gamma$ is \emph{simply
    bi-invertible over $\Gamma$} if there exist:
  \begin{enumerate}
    \item a morphism $g_1\colon\Gamma.B\rightarrow\Gamma.A$;
    \item a section
      $\eta\colon\Gamma.A\rightarrow\Gamma.A.(1_{A},g_1f)^*\Ids_{A}$;
    \item a morphism $g_2\colon\Gamma.B\rightarrow\Gamma.A$;
    \item a section
      $\epsilon\colon\Gamma.B\rightarrow\Gamma.B.(1_{B},fg_2)^*\Ids_{B}$.
  \end{enumerate}
\end{defn}

We now generalize the above definition to arbitrary context extensions in
$\Syn{T}$ by working as usual with $\Syn{T}^{cxt}$ to provide the notion given
in \cite[Def.~1.4]{Kap17}.

\begin{defn}
  Given a dependent type theory with $\Ids$-types $\T$, a morphism
  $f\colon\Gamma.\Delta\rightarrow\Gamma.\Theta$ over $\Gamma$ in $\Syn{T}$ is
  \emph{bi-invertible over $\Gamma$} if it is simply bi-invertible over $\Gamma$
  as a morphism between the two corresponding simple context extensions of
  $\Gamma$ in $\Syn{T}^{cxt}$. It is then called \emph{bi-invertible} if
  $\Gamma$ is the empty context.
\end{defn}

\begin{rmk}
  A morphism $f\colon\Gamma.\Delta\rightarrow\Gamma.\Theta$ bi-invertible over
  $\Gamma$ is also bi-invertible since we can extend a section
  $\Gamma.\Delta\rightarrow\Gamma.\Delta.(1_\Delta,gf)^*\Ids_\Delta$
  to a section
  $\Gamma.\Delta\rightarrow\Gamma.\Delta.(1_{\Gamma.\Delta},gf)^*\Ids_{\Gamma.\Delta}$
  thanks to the map
  $\refl_\Gamma\colon\Gamma\rightarrow\Gamma.p^*_\Gamma\Gamma.\Id_\Gamma$.
\end{rmk}

\begin{rmk}
  Our definition of bi-invertible map is less general than the one generally
  presented (which concerns arbitrary contextual categories with an
  $\Ids$-structure) because that one relies on strong $\Sigmas$-types, which we
  do not assume, or on the $\Ids$-structure on iterated context extensions,
  however we did not provide the construction for arbitrary contextual
  categories.
\end{rmk}

\begin{rmk}
  Our interest in simply bi-invertible morphisms stems from the fact that simple
  ones model the right notion of invertible map in dependent type theory:
  indeed, from the
  required data for a simply bi-invertible map $f$ over $\Gamma$ we can provide
  a section $\Gamma\rightarrow\Gamma.\ishiso{f}$ \wfd{(MAYBE ACTUALLY DO IT?)}
  as defined in \cite[Def.~B.3.3]{KL12}, which itself is the translation into
  the language of contextual categories of the notion of bi-invertible map in
  type theory \cite[Def~4.3.1]{Uni13}.

  It is important to note that, while type theory has no way to encode
  internally the concept of isomorphism of the contextual model, it does have
  its own internal notion of isomorphism. However, given a map $f$, the type
  $\isiso{f}$ is not, in general, a \emph{mere proposition}
  \cite[Def.~3.3.1]{Uni13}, unlike $\ishiso{f}$
  \cite[Thm.~4.3.2]{Uni13}, which makes the latter
  preferable. Also, every bi-invertible map can be given the structure of an
  isomorphism and viceversa, hence they are closely related.
\end{rmk}

\begin{rmk}
  Assuming the Initiality Conjecture \ref{initconj}, localizing a contextual
  category with an
  $\Ids$-structure at bi-invertible maps gives us an $\infty$-category modeling a
  dependent type theory with $\Ids$-types. It has been conjectured that such a
  type theory should provide the internal language of $\infty$-categories, but a
  precise statement of this correspondence has not yet been produced and only a
  few results in this direction have been proven so far. A conjecture of this kind
  would amount to an equivalence between an $\infty$-category of contextual
  categories with some extra structures and an $\infty$-category of structured
  $\infty$-categories induced by the localization at bi-invertible maps, which
  means that first have to determine what properties the localization has. Notable
  in this sense is the Internal Languages Conjecture mentioned in Remark
  \ref{internal}.
\end{rmk}

To study the localizations of syntactic categories we want to apply
the results from the previous section, which require us to specify a
fibrational structure by also providing a class
of fibrations.

\begin{defn}[\cite{Bro73}]\label{fibcats}
  A \emph{fibration category} is a triple $(\cP,W,\fib)$ where $\cP$ is a
  category and $W$, $\fib$ are wide subcategories such that:
  \begin{enumerate}
    \item $\cP$ has a terminal object;
    \item maps to the terminal object lie in $\fib$;
    \item $\fib$ and $W\cap \fib$ are closed under pullback along any map in $\cP$;
    \item every map in $\cP$ can be factored as a map in $W$ followed by one in
      $\fib$;
    \item $W$ has the 2-out-of-6 property.
  \end{enumerate}
\end{defn}

\begin{rmk}
  Seeing $\cP$, $W$ and $\fib$ in the above definition as $\infty$-categories, we
  notice that the triple has canonically the structure of an $\infty$-category
  of fibrant objects, hence we shall adopt the conventions we used in that
  context.
\end{rmk}

\begin{rmk}[\cite{Shu14}]
  Our definition differs slightly from the original one by Brown
  because it asks for $W$ to be closed under the 2-out-of-6 property
  instead of the more classical 2-out-of-3, but, as shown by
  Cisinski, if all of
  the other axioms are satisfied then the following are equivalent:
  \begin{enumerate}
    \item $W$ has the 2-out-of-6 property;
    \item $W$ has the 2-out-of-3 and is saturated, that is a morphism in $\cP$
      becomes invertible in $\ho(\cP)$ if and only if it lies in $W$.
  \end{enumerate}
  See \cite[Thm.~7.2.7]{RB06}. Another difference is that Brown only requires
  factorizations of the diagonal maps $X\rightarrow X\times X$, but then he
  derives our factorization condition from the other properties.
\end{rmk}

\noindent
Now we have enough to specify the fibrational structure.

\begin{prop}[\cite{AKL15}]
  Any contextual category with $\Sigmas$-, $\Ids$-, $\Nat$- and
  $\mathsf{1}$-structures $\sfC$ carries the structure of a fibration category,
  where the class of fibrations is given by all maps isomorphic to dependent
  projections and weak equivalences correspond to all of the bi-invertible ones.
\end{prop}

\noindent
The above result can be however generalized.

\begin{prop}\label{fibcat}
  A contextual category with $\Sigmas$- and $\Ids$- structures $\sfC$ carries
  the structure of a fibration category given by the same classes of maps as
  above.
\end{prop}
\begin{proof}
  It suffices to note that at no point the proof uses the other structures.
\end{proof}

\begin{rmk}
  We shall refer to the above classes of maps as weak equivalences and
  fibrations even in absence of a $\Sigmas$-structure.
\end{rmk}

\begin{rmk}
  The results as stated rely on the Initiality Conjecture \ref{initconj} (since
  in their reasoning the authors used internal languages) and their proof makes
  use of strong $\Sigmas$-types, which they adopted to say that every dependent
  projection is isomorphic to a basic one. We
  can avoid relying on strong $\Sigmas$-types by constructing the fibrational
  structure on $\Syn{T}^{cxt}$ (where the condition on dependent projections
  holds by construction) to later carry it back to $\Syn{T}$ through the
  equivalence, while we do not need the Initiality Conjecture because we can
  argue in $\Syn{T}^{cxt}$ using the dependent type theory $\T$.
\end{rmk}

\begin{cor}\label{fincompl1}
  Given a dependent type theory with $\Sigmas$- and $\Ids$-types $\T$, the
  $\infty$-category $L(\Syn{T})$ is finitely complete.
\end{cor}
\begin{proof}
  It follows directly from Proposition \ref{fibcat} and Proposition \ref{7518}.
\end{proof}

\section{Local Cartesian Closure}

We are now ready to provide a few results needed to show that, given a dependent
type theory with $\Sigmas$-, $\Ids$-, $\Pies$-types and function
extensionality $\T$, the hypothesis of Theorem \ref{7616} are satisfied by
$\Syn{T}$ with respect to the above fibrational structure. Henceforth, we shall
write $\sfC$ for $\Syn{T}$.

\begin{lem}[\cite{Kap17}]\label{radj}
  For any dependent projection $p_\Delta\colon\Gamma.\Delta\rightarrow\Gamma$ in
  $\sfC$, the pullback functor
  $p_\Delta^*\colon\sfC(\Gamma)\rightarrow\sfC(\Gamma.\Delta)$ between the
  fibrant slices admits a right adjoint.
\end{lem}
\begin{proof}
  Let's set $(p_\Delta)_*(\Gamma.\Delta.\Theta)=\Gamma.\Pis(\Delta,\Theta)$. Our
  counit shall be given by
  \[\epsilon_{\Gamma.\Delta.\Theta}\colon\Gamma.\Delta.p^*_\Delta\Pis(\Delta,\Theta)
  \xrightarrow{\exch_{\Delta,\Pis(\Delta,\Theta)}}\Gamma.\Pis(\Delta,\Theta).p^*_{\Pis(\Delta,\Theta)}\Delta
  \xrightarrow{\app_{\Delta,\Theta}}\Gamma.\Delta.\Theta\]
  and it is then sufficient to prove that, for any context morphism
  $f\colon\Gamma.\Delta.p^*_\Delta\Psi\rightarrow\Gamma.\Delta.\Theta$ over
  $\Gamma.\Delta$, there is
  a unique $\hat{f}\colon\Gamma.\Psi\rightarrow\Gamma.\Pis(\Delta,\Theta)$
  making the diagram
  \[\begin{tikzcd}
    {\Gamma.\Delta.p^*_\Delta\Psi} \\
    {\Gamma.\Delta.p^*_\Delta\Pis(\Delta,\Theta)} & {\Gamma.\Delta.\Theta}
    \arrow["f", from=1-1, to=2-2]
    \arrow["{\epsilon_{\Gamma.\Delta.\Theta}}"', from=2-1, to=2-2]
    \arrow["{p^*_\Delta(\hat{f})}"', dotted, from=1-1, to=2-1]
  \end{tikzcd}\]
  commute. This will then uniquely define how the right adjoint acts on the
  morphisms.

  We start by specifying the unit
  $\eta_{\Gamma.\Psi}\colon\Gamma.\Psi\rightarrow\Gamma.\Pis(\Delta,p^*_{\Delta}\Psi)$.

  Let's consider the commutative square
  \[\begin{tikzcd}[column sep=huge]
    {\Gamma.\Psi.p^*_\Psi\Pis(\Delta,p^*_\Delta\Psi)} & {\Gamma.\Pis(\Delta,p^*_\Delta\Psi)} \\
    {\Gamma.\Psi} & \Gamma
    \arrow["{p_{\Pis(\Delta,p^*_\Delta\Psi)}}"', from=1-1, to=2-1]
    \arrow["{p_{\Psi}}"', from=2-1, to=2-2]
    \arrow["{p_{\Pis(\Delta,p^*_\Delta\Psi)}}", from=1-2, to=2-2]
    \arrow["{q(p_\Psi,\Pis(\Delta,p^*_\Delta\Psi))}", from=1-1, to=1-2]
  \end{tikzcd}\]
  where the map $q(p_\Psi,\Pis(\Delta,p^*_\Delta\Psi))$ acts as $(y,g)\mapsto
  g$. If we can provide a section of the left map
  corresponding to $y\mapsto(y,\lambda(x:\Delta).y)$ we are
  done as we can then compose
  it with $q(p_\Psi,\Pis(\Delta,p^*_\Delta\Psi))$ to get our unit
  $y\mapsto\lambda(x:\Delta).y$.
  
  To construct it we pull back the map
  $\id_{\Psi}\colon\Gamma\rightarrow\Gamma.[\Psi,\Psi]$ from Construction
  \ref{idterm} along $p_{p^*_\Psi\Delta}$, getting a section
  \[p^*_{p^*_\Psi\Delta}\id_{\Psi}\colon\Gamma.\Psi.p^*_\Psi\Delta\rightarrow\Gamma.\Psi.p^*_\Psi\Delta.p^*_{p^*_\Psi\Delta}\Psi.\]

  We then apply $\lambda_{\Delta,\Psi}$, thereby getting a section
  \[\lambda_{\Delta,\Psi}(p^*_{p^*_\Psi\Delta}\id_\Psi)
    =\lambda_{\Delta,\Psi}(1_{p^*_\Psi\Delta},p_{p^*_\Psi\Delta})
\colon\Gamma.\Psi\rightarrow\Gamma.\Psi.p^*_\Psi\Pis(\Delta,p^*_\Delta\Psi),\]
  which is exactly what we were looking for.

  \noindent
  Let us define our lift $\hat{f}$ as the composite
  $\Gamma.\Pis(\Delta,f)\cdot\eta_{\Gamma.\Psi}$, where $\Gamma.\Pis(\Delta,f)$
  was specified in Construction \ref{postcomp}. The commutativity of the above
  triangle follows by type-theoretic reasoning from
  \begin{align*}
    (x,y) \\
    ^{\eta_{\Gamma.\Psi}}
    &\mapsto(x,\lambda(x:\Delta).y) \\
    ^{\Gamma.\Pis(\Delta,f)}
    &\mapsto(x,\lambda(x:\Delta).\app(f,\app(\lambda(x:\Delta).y,x))) \\
    ^{\beta\text{-reduction}}
    &\equiv(x,\lambda(x:\Delta).\app(f,y)) \\
    ^{\exch_{\Pis(\Delta,\Psi),\Delta}}
    &\mapsto(\lambda(x:\Delta).\app(f,y),x) \\
    ^{\app_{\Delta,\Psi}}
    &\mapsto(x,\app(\lambda(x:\Delta).\app(f,y),x)) \\
    ^{\beta\text{-reduction}}
    &\equiv(x,\app(f,y))
  \end{align*}
  and our description of
  $f\colon\Gamma.\Delta.p^*_\Delta\Psi\rightarrow\Gamma.\Delta.\Theta$ in
  Remark \ref{applyf}, while the
  uniqueness of $\hat{f}$ comes from the fact that we can not construct,
  from every term $y:\Psi$, another term
  $g(y):\Pi(\Delta,\Theta)$ such that $\app(g(y),x)\equiv\app(f,y)$ for any
  $x:\Delta$. Indeed, we can derive
  \begin{align*}
    g(y) \\
         ^{\eta\text{-conversion}}&\equiv\lambda(x:\Delta).\app(g(y),x) \\
         &\equiv\lambda(x:\Delta).\app(f,y) \\
         &\equiv\app(\hat{f},y),
  \end{align*}
  which concludes the proof.

  \noindent
  This also shows that $(p_{\Delta})_*(f)=\Gamma.\Pis(\Delta,f)$, meaning that
  the construction of $\Gamma.\Pis(\Delta,f)$ is functorial.
\end{proof}

\noindent
We know that every fibration in $\sfC$ is isomorphic to a dependent
projection, so this tells us that every fibration induces an adjunction
between fibrant slices.

\noindent
To apply Theorem \ref{7616}, we need to show that $(p_\Delta)_*$ is left exact
(actually, we need to prove that it preserves trivial fibrations, but it
essentially requires the same amount of work), which we shall do in two steps.

\begin{lem}[\cite{KL18}]\label{wepreserved}
  Consider a bi-invertible
  map $f\colon\Gamma.\Delta.\Theta\rightarrow\Gamma.\Delta.\Psi$ over
  $\Gamma.\Delta$ in $\sfC$. The
  map $\Gamma.\Pis(\Delta,f)$ is then bi-invertible over $\Gamma$.
\end{lem}
\begin{proof}
  We shall construct an homotopical right inverse to $\Gamma.\Pis(\Delta,f)$ by
  working with $g_1$ and $\eta$ since the other part of the construction
  involving $g_2$ and $\epsilon$ is essentially identical. To do so, we consider
  the commutative diagram
  \[\begin{tikzcd}[column sep=huge]
    {\Gamma.\Delta.\Psi.(1_\Psi,fg_1)^*\Ids_{\Psi}} & {\Gamma.\Delta.\Psi.p^*_\Psi\Psi.\Ids_\Psi} \\
    {\Gamma.\Delta.\Psi} & {\Gamma.\Delta.\Psi.p^*_\Psi\Psi}
    \arrow["{p_{(1_\Psi,fg_1)^*\Ids_{\Psi}}}", from=1-1, to=2-1]
    \arrow["\eta", curve={height=-12pt}, from=2-1, to=1-1]
    \arrow["{p_{\Ids_\Psi}}", from=1-2, to=2-2]
    \arrow["{(1_\Psi,fg_1)}"', from=2-1, to=2-2]
    \arrow["{q((1_\Psi,fg_1),\Ids_\Psi)}", from=1-1, to=1-2]
  \end{tikzcd}\]
  in $\sfC(\Gamma.\Delta)$ and apply the right adjoint $(p_\Delta)_*$, which
  gives us
  \begin{align*}
    (p_\Delta)_*(q((1_\Psi,fg_1),\Ids_\Psi)\cdot\eta) &\colon\Gamma.\Pis(\Delta,\Psi)\rightarrow \\
                             &\Gamma.\Pis(\Delta,\Psi.p^*_\Psi\Psi.\Ids_\Psi)=
    \Gamma.\Pis(\Delta,\Psi).p^*_{\Pis(\Delta,\Psi)}\Pis(\Delta,\Psi).\Htp_{\Delta,\Psi}
  \end{align*}
  in $\sfC(\Gamma)$. Postcomposing with the function extensionality map
  \[\ext_{\Delta,\Psi}\colon
    \Gamma.\Pis(\Delta,\Psi).p^*_{\Pis(\Delta,\Psi)}\Pis(\Delta,\Psi).\Htp_{\Delta,\Psi}
    \rightarrow
  \Gamma.\Pis(\Delta,\Psi).p^*_{\Pis(\Delta,\Psi)}\Pis(\Delta,\Psi).\Ids_{\Pis(\Delta,\Psi)},\]
  we obtain a morphism
  \[\Gamma.\Pis(\Delta,\Psi)\rightarrow
  \Gamma.\Pis(\Delta,\Psi).p^*_{\Pis(\Delta,\Psi)}\Pis(\Delta,\Psi).\Ids_{\Pis(\Delta,\Psi)}\]
  fitting in the commutative diagram
  \[\scriptsize\begin{tikzcd}[column sep=small]
    {\Gamma.\Pis(\Delta,\Psi)} \\
    & {\Gamma.\Pis(\Delta,\Psi).(1_{\Psi(\Delta,\Psi)},\Gamma.\Pis(\Delta,(1_\Psi,fg_1))^*\Ids_{\Pis(\Delta,\Psi)}}
    & {\Gamma.\Pis(\Delta,\Psi).p^*_{\Pis(\Delta,\Psi)}\Pis(\Delta,\Psi).\Ids_{\Pis(\Delta,\Psi)}} \\
    & {\Gamma.\Pis(\Delta,\Psi)}
    & {\Gamma.\Pis(\Delta,\Psi).p^*_{\Pis(\Delta,\Psi)}\Pis(\Delta,\Psi)}
    \arrow["{p_{\Ids_{\Pis(\Delta,\Psi)}}}", from=2-3, to=3-3]
    \arrow["{(1_{\Psi(\Delta,\Psi)},\Gamma.\Pis(\Delta,(1_\Psi,fg_1)))}"', from=3-2, to=3-3]
    \arrow["{p_{(1_{\Psi(\Delta,\Psi)},\Gamma.\Pis(\Delta,(1_\Psi,fg_1))^*\Ids_{\Pis(\Delta,\Psi)}}}", from=2-2, to=3-2]
    \arrow[from=2-2, to=2-3]
    \arrow[curve={height=-12pt}, from=1-1, to=2-3]
    \arrow[curve={height=30pt}, Rightarrow, no head, from=1-1, to=3-2]
    \arrow[dotted, from=1-1, to=2-2]
  \end{tikzcd}\]
  and thereby inducing the factorization shown, which is what we needed.
\end{proof}

\begin{lem}[\cite{Kap17}]\label{lccfc}
  In the above conditions, the functor
  $(p_\Delta)_*\colon\sfC(\Gamma.\Delta)\rightarrow\sfC(\Gamma)$ is left exact.
\end{lem}
\begin{proof}
  As a right adjoint, $(p_\Delta)_*$ preserves limits and in particular
  pullbacks and the terminal object. Also, by Lemma \ref{piequal}, it preserves
  dependent projections and Lemma \ref{wepreserved} tells us that the same
  goes for weak equivalences, which concludes the proof.
\end{proof}

Again, the above extends to all fibrations in $\sfC$ and it allows to prove our
desired result.

\begin{thm}[\cite{Kap14}]\label{finalthm}
  Given a dependent type theory with $\Sigmas$-, $\Ids$-, $\Pies$-types and
  function extensionality $\T$ the $\infty$-category $L(\Syn{T})$ is locally
  cartesian closed.
\end{thm}
\begin{proof}
  We already know that it is finitely complete by Corollary \ref{fincompl1}. The
  hypothesis of Theorem \ref{7616} are satisfied by Lemma \ref{radj} and Lemma
  \ref{lccfc}.
\end{proof}

\section{Conclusions}
