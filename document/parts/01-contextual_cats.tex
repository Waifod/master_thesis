\chapter{Contextual Categories}

There are many models of dependent type theory from category theory in
the literature, like \emph{category with attributes}, \emph{categories with
families} and \emph{comprehension categories}, which are
fairly similar among them, as shown by the adjunctions between their categories.
Here we shall work with \emph{contextual
categories}, which were first explored by Cartmell and Streicher in
\wfd{(INSERT REFS)} and later by Voevodsky, under the name $C$\emph{-systems},
in \wfd{(MORE REFS)}. Lumsdaine in his phd thesis claims that contextual
categories are the right framework PROP 1.2.5.

\section{Definitions}

We begin by providing the basic definitions modeling structural rules.

\begin{defn}
  A \emph{contextual category} $\sfC$ is a category with the following data:
  \begin{enumerate}
    \item a small category, which we also call $\sfC$, with a grading on objects
      $\Ob\sfC=\coprod_{n\in\bbN}\Ob_n\sfC$;
    \item an object $*\in\Ob_0\sfC$;
    \item for each $n\in\bbN$, a map
      $\cft_n\colon\Ob_{n+1}\sfC\rightarrow\Ob_n\sfC$, often simply denoted
      $\cft$;
    \item for each $n\in\bbN$ and $X\in\Ob_{n+1}\sfC$, a map $p_X\colon
      X\rightarrow\cft X$;
    \item for each $n\in\bbN$, $X\in\Ob_{n+1}\sfC$ and $f\colon Y\rightarrow
      \cft X$, an object $f^*X$ and a map $q(f,X)\colon f^*X\rightarrow X$;
  \end{enumerate}
  such that:
  \begin{enumerate}
    \item $*$ is the unique element of $\Ob_0\sfC$;
    \item $*$ is terminal;
    \item for each $n$, $X\in\Ob_{n+1}\sfC$ and $f\colon Y\rightarrow \cft X$, we
      have $\cft(f^*X)=Y$ and the square
      \[\begin{tikzcd}
        {f^*X} & X \\
        Y & \cft X
        \arrow["f"', from=2-1, to=2-2]
        \arrow["{p_X}", from=1-2, to=2-2]
        \arrow["{q(f,X)}", from=1-1, to=1-2]
        \arrow["{p_{f^*X}}"', from=1-1, to=2-1]
      \end{tikzcd}\]
      is a pullback;
    \item for each $n\in\bbN$, $X\in\Ob_{n+1}\sfC$ and pair of maps $f\colon
      Y\rightarrow \cft X$, $g\colon Z\rightarrow Y$, we have $(fg)^*X=g^*f^*X$,
      $1^*_{\cft X}X=X$, $q(fg,X)=q(f,X)\cdot q(g,f^*X)$ and $q(1_{\cft X},X)=1_X$.
  \end{enumerate}
\end{defn}

\begin{rmk}
  The last condition in the definition means that our choice of pullbacks is
  functorial, which allows us to see contextual categories as a strict model of
  dependent type theory, not requiring keeping track of coherency maps.
  Other models, like comprehension categories, do not
  have such requirements, which makes them more general but also
  interpretation harder, unless they are first strictified, that is replaced by
  an equivalent strict model of the same kind, like split comprehension
  categories \wfd{(HOW DO YOU STRICTIFY THEM?)}. Famously, we also have
  homotopical models of dependent type
  theories, like tribes and fibration categories \wfd{(REFERENCE)}, whose
  internal languages are precisely dependent type theories with intensional
  $\Ids$-types and $\Sigmas$-types.
\end{rmk}

A motivating example for contextual categories is the \emph{syntactic category
of a dependent type theory}, which constructs from a dependent type theory
$\T$ a contextual category $\Syn{\T}$ explicitly modeling it as we will show.

\begin{construction}\cite[2.6]{Car78}
  \wfd{(KL12 also mentions other stuff! Should I?)}
  Given a dependent type theory $\T$ with the structural rules specified in
  \cite[A.1]{KL12}, its syntactic category $\Syn{\T}$ has:
  \begin{enumerate}
    \item $\Ob_n\Syn{\T}$ given by contexts $[x_1:A_1,\ldots,x_n:A_n]$ of length
      $n$, modulo definitional equality and renaming of free variables;
    \item maps are \emph{context morphisms}, or \emph{substitutions}, modulo
      definitional equalities and renaming of free variables. This means that a
      map
      \[[f]\colon[x_1:A_1,\ldots,x_n:A_n]\rightarrow[y_1:B_1,\ldots,y_m:B_m(y_1,\ldots,y_{m-1})]\]
      is an equivalence class of sequences of terms $f_1,\ldots,f_m$ such that
      \begin{align*}
        x_1:A_1,\ldots,x_n:A_n &\vdash f_1:B_1, \\
        \vdots & \\
        x_1:A_1,\ldots,x_n:A_n &\vdash f_m:B_m(f_1,\ldots,f_{m-1}),
      \end{align*}
      are all derivable judgements and two such sequences
      $(f_1,\ldots,f_n),(g_1,\ldots g_n)$ are equivalent if we have
      \[x_1:A_1,\ldots,x_n:A_n\vdash f_i\equiv g_i\:B_i(f_1,\ldots,f_{i-1})\]
      for every $i$, where $\equiv$ represents judgemental equality; we shall
      henceforth write $[f_i]$ for such sequences of terms up to equivalence
      specifying maps between contexts;
    \item composition is given by substitution;
    \item the identity $\Gamma\rightarrow\Gamma$ is given by the variables of
      $\Gamma$, considered as
      terms, that is we take the sequence $[x_i]$ given by
      \begin{align*}
        x_1:A_1,\ldots,x_n:A_n &\vdash x_1:A_1, \\
        \vdots & \\
        x_1:A_1,\ldots,x_n:A_n &\vdash x_n:A_n(x_1,\ldots,x_{n-1});
      \end{align*}
    \item the terminal object is the empty context $[]$;
    \item
      $ft([x_1:A_1,\ldots,x_{n+1}:A_{n+1}])=[x_1:A_1,\ldots,x_n:A_n]$;
    \item for $\Gamma=[x_1:A_1,\ldots,x_n:A_{n+1}]$,
      $p_{\Gamma}\colon\Gamma\rightarrow \cft\Gamma$ is the \emph{dependent
      projection context morphism} $[x_1,\ldots,x_n]$, defined by
      \begin{align*}
        x_1:A_1,\ldots,x_{n+1}:A_{n+1} &\vdash x_1:A_1, \\
        \vdots & \\
        x_1:A_1,\ldots,x_{n+1}:A_{n+1} &\vdash x_n:A_n(x_1,\ldots,x_n)
      \end{align*}
      and thereby simply forgetting the last variable of $\Gamma$;
    \item given contexts
      \begin{align*}
        \Gamma &=[x_1:A_1,\ldots,x_{n+1}:A_{n+1}(x_1,\ldots,x_n)], \\
        \Delta &=[y_1:B_1,\ldots,y_m:B_m(y_1,\ldots,y_{m-1})]
      \end{align*}
      and a map $f=[f_i(y)]\colon\Delta\rightarrow\cft\Gamma$ (where $y$ is a
      vector of variables of length $m$), the pullback $f^*\Gamma$ is the
      context
      \[[y_1:B_1,\ldots,y_m:B_m(y_1,\ldots,y_{m-1}),y_{m+1}:A_{n+1}(f_1(y),\ldots,f_n(y))]\]
      for some new variable $y_{m+1}$, while $q(\Gamma,f)\colon
      f^*\Gamma\rightarrow\Gamma'$ is specified by $[f_1,\ldots,f_n,y_{m+1}]$.
  \end{enumerate}
\end{construction}

\begin{rmk}
  Given a dependent type theory $\T$, the terms $a:A$ of a type over a context
  $\Gamma$ can be recovered (up to definitional equality) from the syntactic
  category $\Syn{\T}$ by looking at sections of the basic dependent projection
  $p_{[\Gamma,x:A]}\colon[\Gamma,x:A]\rightarrow[\Gamma]$, which indeed act as
  identities over $\Gamma$ and furthermore specify a term $\Gamma\vdash a:A$.
  Given the importance of such maps, we shall often simply write ``sections'' to
  refer to sections of basic dependent projections, without specifying which ones
  unless it creates ambiguity.

  The above construction also tells us how to think about the other elements in
  the definition of contextual categories: basic dependent
  projections $p_{\Gamma.A.B}\colon\Gamma.A.B\rightarrow\Gamma.A$ represent
  dependent types $B$ over $A$ in the context $\Gamma$ and pulling back along
  a dependent projection corresponds to substituting variables, while a choice of
  a term corresponds to a choice of a section of a basic dependent projection
  and so on for the other objects in the definition.

  It is also worth noting that, in $[x_1:A_1,\ldots,x_n:A_n]$, both the type
  $A_i$ and the term $x_i:A_i$ can depend on all of the previous terms, hence we
  should be writing $A_i(x_1,\ldots,x_{i-1})$ and $x_i(x_1,\ldots,x_{i-1})$,
  however we will avoid doing it as we have done so far unless it creates
  ambiguity.
\end{rmk}

\begin{notation}
  We shall also make use of some conventions inspired by this construction.
  Namely, given a contextual category $\sfC$ and an object $\Gamma\in\Ob_n\sfC$,
  we shall
  write $(\Gamma,A_1,\ldots,A_k)$, $\Gamma.A_1.\ldots.A_k$ and $\Gamma.\Delta$
  interchangeably
  for an object $X$ in $\Ob_{n+k}\sfC$ with $\cft^kX=\Gamma$ and call it a
  \emph{context extension of $\Gamma$ of length $k$}. We shall
  also write $p_{(A_1,\ldots,A_k)}$, $p_{A_1.\ldots.A_k}$ and $p_\Delta$ for the
  composition of the basic dependent projections
  $p_{\Gamma.A_1.\ldots.A_i}$, with $i$
  ranging from $1$ to $k$, and the resulting map will be called a
  \emph{dependent projection}. In the case where $k=1$, $p_{A_1}$ corresponds to
  a basic dependent projection and the context extension of $\Gamma$ will be
  \emph{simple}, while if $k=0$ we have $p=1_{\Gamma}$ and then the context
  extension will be \emph{trivial}. We shall also write $1_\Delta,
  1_{A_1.\ldots.A_k},\ldots,1_{A_k}$ for $1_{\Gamma.\Delta}$, depending on what
  we want to emphasize. Greek letters
  shall be used to indicate context extensions of arbitrary length, while Latin
  ones will be reserved to simple extensions.

  Continuing, given a dependent projection $p_{A_1.\ldots.A_k}=p_\Theta$ as
  above and a context morphism $f\colon\Delta\rightarrow\Gamma$, we define
  inductively $f^*(\Gamma.\Theta)=\Delta.f^*\Theta=\Delta.f^*A_1.\ldots.f^*A_k$
  and $q(f,\Gamma.\Theta)=q(f,\Theta)=q(f,A_1.\ldots.A_k)$ by looking at the
  pasting of pullback squares
  \[\begin{tikzcd}
    {\Delta.f^*A_1.\ldots.f^*A_k} & {\Delta.f^*A_1.\ldots.f^*A_{k-1}} & \cdots & {\Delta.f^*A_1} & \Delta \\
    {\Gamma.A_1.\ldots.A_k} & {\Gamma.A_1.\ldots.A_{k-1}} & \cdots & {\Gamma.A_1} & \Gamma
    \arrow["f", from=1-5, to=2-5]
    \arrow["{p_{A_1}}", from=2-4, to=2-5]
    \arrow["{p_{f^*A_1}}"', from=1-4, to=1-5]
    \arrow["{q(f,A_1)}"', from=1-4, to=2-4]
    \arrow[from=2-3, to=2-4]
    \arrow[from=1-3, to=1-4]
    \arrow[from=1-2, to=1-3]
    \arrow[from=2-2, to=2-3]
    \arrow["{q(f,A_1.\ldots.A_{k-1})}", from=1-2, to=2-2]
    \arrow["{q(f,A_1.\ldots.A_k)}"', from=1-1, to=2-1]
    \arrow["{p_{f^*A_k}}"', from=1-1, to=1-2]
    \arrow["{p_{A_k}}", from=2-1, to=2-2]
    \arrow["{p_{f^*A_1.\ldots.f^*A_k}}", curve={height=-24pt}, from=1-1, to=1-5]
    \arrow["{p_{A_1.\ldots.A_k}}"', curve={height=24pt}, from=2-1, to=2-5]
  \end{tikzcd},\]
  which also shows that
  $$q(f,A_1.\ldots.A_k)=q(q(f,A_1),A_2.\ldots.A_k)=q(q(f,A_1.\ldots.A_{k-1}),A_k).$$
  As usual, if $k=0$ we have $q(f,\Theta)=f$, $f^*(\Gamma.\Theta)=\Gamma$, while
  for $k=1$ we have $q(f,A_1)=q(f,\Gamma.A_1)$, $\Delta.f^*A_1=f^*(\Gamma.A_1)$
  agreeing with the base structure of $\sfC$.

  Finally, given a section $a\colon\Gamma\rightarrow\Gamma.A$ and a context
  morphism $f\colon\Delta\rightarrow\Gamma$, we also want to specify $f^*a$,
  that is the section which we get by switching context. This is given by
  the map $(1_{\Delta},a\cdot f)$ specified by the pullback square
  \[\begin{tikzcd}
    \Delta \\
    & {\Delta.f^*A} & {\Gamma.A} \\
    & \Delta & \Gamma
    \arrow["f"', from=3-2, to=3-3]
    \arrow["{p_{f^*A}}", from=2-2, to=3-2]
    \arrow["{p_A}", from=2-3, to=3-3]
    \arrow["{q(f,A)}"', from=2-2, to=2-3]
    \arrow[curve={height=12pt}, Rightarrow, no head, from=1-1, to=3-2]
    \arrow["{a\cdot f}", curve={height=-12pt}, from=1-1, to=2-3]
    \arrow["{(1_\Delta,a\cdot f)}"{description}, dotted, from=1-1, to=2-2]
  \end{tikzcd},\]
  and, as shown by the commutative diagram
  \[\begin{tikzcd}
    \Delta & \Gamma \\
    {\Delta.f^*A} & {\Gamma.A} \\
    \Delta & \Gamma
    \arrow["f"', from=3-1, to=3-2]
    \arrow["{p_{f^*A}}", from=2-1, to=3-1]
    \arrow["{p_A}"', from=2-2, to=3-2]
    \arrow["{q(f,A)}"', from=2-1, to=2-2]
    \arrow["a"', from=1-2, to=2-2]
    \arrow["{f^*a}", from=1-1, to=2-1]
    \arrow["f", from=1-1, to=1-2]
    \arrow[curve={height=-24pt}, Rightarrow, no head, from=1-2, to=3-2]
    \arrow[curve={height=24pt}, Rightarrow, no head, from=1-1, to=3-1]
  \end{tikzcd},\]
  it corresponds to the pullback of $a$ along $q(f,A)$. By the techniques we
  provided earlier, we extend this construction to contexts extensions of
  arbitrary length.
\end{notation}

\begin{defn}
  A \emph{contextual functor} between contextual categories
  $F\colon\sfC\rightarrow\sfD$ is a functor on the underlying categories which
  preserves the grading, basic dependent projections and such that
  $q(Ff,FX)=F(q(f,X))$.
\end{defn}

\begin{rmk}
  Our definition allows us to see contextual categories as models for an
  essentially algebraic theory with sorts indexed by $\bbN+\bbN\times\bbN$. In
  that context, we get a notion of morphisms between models of this theory,
  which coincides with the one we have just provided. The category of models for
  this theory will be the category of contextual categories, denoted by $\Cxl$,
  which is complete and cocomplete as the category of models of an
  essentially algebraic theory REFS.
\end{rmk}

A problem of providing a model of dependent type theory is exhibiting the
basic structural rules, namely context substitution, variable binding,
variable substitution and so on for us. The defining properties of contextual
categories are meant to model them and therefore 
take care of all of that for us, meaning that as long
as we can show that something models a contextual category we automatically get
an interpretation of dependent type theory, thereby eliminating a lot of
bureaucracy. For this to be true, however, we need the following statement,
which we shall assume because the results we need rely on it.

\begin{named}[Initiality]
  Given a dependent type theory $\T$, its syntactic category $\Syn{\T}$ is
  initial in the category of contextual categories with the appropriate
  structure.
\end{named}

\begin{rmk}
Here by ``appropriate structure'' we mean extra algebraic structures meant to
model the
logical rules of the type theory, like $\Sigmas$-types, $\Pis$-types and
$\Ids$-types. Indeed, the definition of contextual category as we said deals with
the structural rules, but nothing more. We will introduce the such structures in
the next section.
\end{rmk}

\begin{rmk}
This conjecture has been partially proven for a simple variant of dependent type
theory in \cite{Str91}, however we still do not have a fully rigorous general
statement.

Such a statement would first require a general notion of dependent type theory,
which has been worked on in \cite{Isa17, Uem19, Bru20, BHL20, NU22}, and
there is an ongoing effort to provide a formalization and a flexible proof of
some variants of the conjecture via proof assistants \cite{BL20}.

If we could do that, then for any contextual category $\sfC$ we would have a
contextual functor $\Syn{\T}\rightarrow\sfC$ explaining how to interpret $\T$ in
$\sfC$. This is essentually an algorithmic problem: it reduces to explaining
inductively to a computer how to construct the aforementioned functor.
\end{rmk}

\section{Logical Structures}
We now define the extra structures on contextual categories we mentioned
earlier. Our definitions shall be taken from \cite{KL12} and \cite{KL18}.

\begin{defn}
  A \emph{$\Sigmas$-type structure} on a contextual category $\sfC$ consists of:
  \begin{enumerate}
    \item for each $(\Gamma,A,B)\in\Ob_{n+2}\sfC$, an object
      $(\Gamma,\Sigmas(A,B))\in\Ob_{n+1}\sfC$;
    \item for each $(\Gamma,A,B)\in\Ob_{n+2}\sfC$, a morphism $\pair_{A,B}\colon
      (\Gamma,A,B)\rightarrow(\Gamma,\Sigmas(A,B))$ over $\Gamma$;
    \item for each $(\Gamma,A,B),(\Gamma,\Sigmas(A,B),C)\in\Ob_{n+2}\sfC$, and
      $d\colon(\Gamma,A,B)\rightarrow(\Gamma,\Sigmas(A,B),C)$ with $p_C\cdot
      d=\pair_{A,B}$, a section
      $\msplit_d\colon(\Gamma,\Sigmas(A,B))\rightarrow(\Gamma,\Sigma(A,B),C)$ such
      that $\msplit_d\cdot\pair_{A,B}=d$;
    \item where all of the above is compatible with context substitution, that
      is given a map $f\colon\Delta\rightarrow\Gamma$ we have
      \begin{align*}
        f^*(\Gamma,\Sigmas(A,B)) &=(\Delta,\Sigma(f^*A,f^*B)), \\
        f^*\pair_{A,B} &=\pair_{f^*A,f^*B}, \\
        f^*\msplit_d &=\msplit_{f^*d}.
      \end{align*}
  \end{enumerate}
\end{defn}

\begin{defn}
  A \emph{$\Ids$-type structure} on a contextual category $\sfC$ consists of:
  \wfd{COMPLETE}
\end{defn}

\begin{defn}
  A \emph{$\Pis$-type structure} on a contextual category $\sfC$ consists of:
  \begin{enumerate}
    \item for each $(\Gamma,A,B)\in\Ob_{n+2}\sfC$, an object
      $(\Gamma,\Pis(A,B))\in\Ob_{n+1}\sfC$;
    \item for each $(\Gamma,A,B)\in\Ob_{n+2}\sfC$, a map
      $\app_{A,B}\colon(\Gamma,\Pis(A,B),p^*_{\Pis(A,B)}A)\rightarrow(\Gamma,A,B)$
      over $\Gamma$, that is such that $p_B\cdot\app_{A,B}=q(\Pis(A,B),A)$;
    \item for each $(\Gamma,A,B)\in\Ob_{n+2}\sfC$ and section
      $b\colon(\Gamma,A)\rightarrow(\Gamma,A,B)$, a section
      $\lambda_{A,B}(b)\colon\Gamma\rightarrow(\Gamma,\Pis(A,B))$;
    \item such that for any sections
      $k\colon\Gamma\rightarrow(\Gamma,\Pis(A,B))$,
      $a\colon\Gamma\rightarrow\Gamma.A$
      the map $\app_{A,B}(k,a)$ defined as the composition of $\app_{A,B}$ with
      $(k,a)$ specified by the factorization through the pullback
      \[\begin{tikzcd}
        \Gamma \\
        & {\Gamma.\Pis(A,B).p^*_{\Pis(A,B)}A} & {\Gamma.A} \\
        & {\Gamma.\Pis(A,B)} & \Gamma
        \arrow["{p_{\Pis(A,B)}}"', from=3-2, to=3-3]
        \arrow["{p_A}", from=2-3, to=3-3]
        \arrow["{q(p_{\Pis(A,B)},A)}"', from=2-2, to=2-3]
        \arrow["{p_{p^*_{\Pis(A,B)}A}}", from=2-2, to=3-2]
        \arrow["a", curve={height=-12pt}, from=1-1, to=2-3]
        \arrow["k"', curve={height=12pt}, from=1-1, to=3-2]
        \arrow["{(k,a)}"{description}, dotted, from=1-1, to=2-2]
      \end{tikzcd},\]
      we have $p_B\cdot\app_{A,B}(k,a)=a$;
    \item such that for any $(\Gamma,A,B)$, $a\colon\Gamma\rightarrow(\Gamma,A)$
      and $b\colon(\Gamma,A)\rightarrow(\Gamma,A,B)$ we have
      \[\app(\lambda_{A,B}(b),a)=b\cdot a;\]
    \item all of the above is compatible with context substitution, that is for
      any $f\colon\Delta\rightarrow\Gamma$ we have
      \begin{align*}
        f^*(\Gamma,\Pis(A,B)) &=(\Delta,\Pis(f^*A,f^*B)), \\
        f^*\lambda_{A,B}(b) &=\lambda_{f^*A,f^*B}(f^*b), \\
        f^*(\app_{A,B}(k,a)) &=\app_{f^*A,f^*B}(f^*k,f^*a).
      \end{align*}
  \end{enumerate}

  We shall say that the the $\Pis$-type structure satisfies the
  \emph{$\Pies$-rule} if the equation
  \[q(p_{\Pis(A,B)},\Pis(A,B))\cdot\lambda(1_{p^*_{\Pis(A,B)}A},\app_{A,B})=1_{\Gamma.\Pis(A,B)}\]
  is satisfied, in which case the structure will be called a
  \emph{$\Pies$-type structure}. 


  \wfd{(WE STILL NEED $\Piext$, WHICH REQUIRES $\Ids$-TYPE STRUCTS.)}
\end{defn}

\begin{rmk}
  The definition we provided matches \cite[Def.~2.5]{KL18}, while the one in
  \cite[App.~B.1.1]{KL12} is mildly different as it starts from
  $\app_{A,B}(f,a)$ instead of $\app_{A,B}$ and constructs the latter from the
  former. We prefer this because it will allow us to more easily extend the
  $\Pis$-structure to arbitrary context extensions and check the necessary
  properties in the next chapter.
\end{rmk}

\begin{rmk}
  Internally, the map $\app_{A,B}$
  corresponds to \[(f,a)\mapsto(a,\app(f,a)).\] Also, 
  the association $b\mapsto\lambda_{A,B}(b)$ for some section
  $\Gamma.A\rightarrow\Gamma.A.B$ corresponds to \emph{lambda
  abstraction}, hence $\lambda_{A,B}(b)$ specifies the term
  $\lambda(a:A).b(a):\Pis(A,B)$.
  The map on the left of the definition of the $\Pies$-rule is the
  \emph{$\eta$-expansion map}, that is it corresponds internally to the map
  \[f\mapsto\lambda(a:A).\app(f,a)\] over $\Gamma$, thus
  the property means that we have a judgement
  \[\Gamma\vdash f\equiv\lambda(a:A).\app(f,a):\Pis(A,B).\]
  The map $(1_{p^*_{\Pis(A,B)}A},\app_{A,B})$ is specified by the following
  factorization through the pullback.
  \[\begin{tikzcd}
    {\Gamma.\Pis(A,B).p^*_{\Pis(A,B)}A} \\
    & {\Gamma.\Pis(A,B).p^*_{\Pis(A,B)}A.p^*_{\Pis(A,B)}B} & {\Gamma.A.B} \\
    & {\Gamma.\Pis(A,B).p^*_{\Pis(A,B)}A} & {\Gamma.A}
    \arrow["{q(p_{\Pis(A,B)},A)}"', from=3-2, to=3-3]
    \arrow["{p_A}", from=2-3, to=3-3]
    \arrow["{q(p_{\Pis(A,B)},A.B)}"', from=2-2, to=2-3]
    \arrow["{p_{p^*_{\Pis(A,B)}A}}", from=2-2, to=3-2]
    \arrow["{\app_{A,B}}", curve={height=-12pt}, from=1-1, to=2-3]
    \arrow[curve={height=12pt}, Rightarrow, no head, from=1-1, to=3-2]
    \arrow["{(1_{p^*_{\Pis(A,B)}A},\app_{A,B})}"{description}, dotted, from=1-1, to=2-2]
  \end{tikzcd}.\]
\end{rmk}

\begin{construction}\label{applyf}
  Let $\sfC$ be a contextual category with a $\Pis$-type structure,
  $f\colon\Gamma.A\rightarrow\Gamma.B$ a map over $\Gamma$ and therefore
  corresponding internally to $a\mapsto b(a)$. We want to provide
  a section $\Gamma\rightarrow\Gamma.\Pis(A,B)$, which we will also denote $f$,
  corresponding to $\lambda(a:A).b(a):\Pis(A,B)$. We do so by looking at the
  commutative diagram
  \[\begin{tikzcd}
    {\Gamma.A} \\
    & {\Gamma.A.p^*_AB} & {\Gamma.B} \\
    & {\Gamma.A} & \Gamma
    \arrow["f", curve={height=-12pt}, from=1-1, to=2-3]
    \arrow[curve={height=12pt}, Rightarrow, no head, from=1-1, to=3-2]
    \arrow["{(1_A,f)}"{description}, dotted, from=1-1, to=2-2]
    \arrow["{p_B}", from=2-3, to=3-3]
    \arrow["{p_A}"', from=3-2, to=3-3]
    \arrow["{q(p_A,B)}"', from=2-2, to=2-3]
    \arrow["{p_{p^*_AB}}", from=2-2, to=3-2]
  \end{tikzcd}\]
  and then taking $\lambda_{A,p^*_AB}(1_A,f)$. The original
  $f\colon\Gamma.A\rightarrow\Gamma.B$ can then be described internally as
  $a\mapsto\app(f,a)$.
\end{construction}

\begin{construction}
  Let $\sfC$ be a contextual category with a $\Pis$-type structure,
  $f\colon\Gamma.A.B\rightarrow\Gamma.A.B'$ a morphism over
  $\Gamma.A$, meaning a map $b\mapsto\app(f,b)$ for the term
  $f:\Pis(B,B')$ it induces. We want to construct a morphism
  $$\Gamma.\Pis(A,f)\colon\Gamma.\Pis(A,B)\rightarrow\Gamma.\Pis(A,B')$$
  modeling the postcomposition by $f$, that is
  $g\mapsto\lambda(a:A).\app(f,\app(g,a))$.

  We do so by looking at the commutative diagram
  \[\begin{tikzcd}[column sep=large]
    {\Gamma.\Pis(A,B).p^*_{\Pis(A,B)}A} && {\Gamma.A.B} \\
                                          &
    {\Gamma.\Pis(A,B).p^*_{\Pis(A,B)}A.p^*_{\Pis(A,B)}B'} & {\Gamma.A.B'} \\
    & {\Gamma.\Pis(A,B).p^*_{\Pis(A,B)}A} & {\Gamma.A}
    \arrow["{q(p_{\Pis(A,B)},A)}"', from=3-2, to=3-3]
    \arrow["{p_{B'}}"', from=2-3, to=3-3]
    \arrow["{q(p_{\Pis(A,B)},A.B')}"', from=2-2, to=2-3]
    \arrow["{p_{p^*_{\Pis(A,B)}B'}}", from=2-2, to=3-2]
    \arrow[curve={height=12pt}, Rightarrow, no head, from=1-1, to=3-2]
    \arrow["f"', from=1-3, to=2-3]
    \arrow["{p_B}", curve={height=-36pt}, from=1-3, to=3-3]
    \arrow["{\app_{A,B}}", from=1-1, to=1-3]
    \arrow["{(1_{p^*_{\Pis(A,B)}A},f\cdot\app_{A,B})}"{pos=1}, from=1-1, to=2-2, dotted]
  \end{tikzcd}\]
  and then applying $\lambda_{p^*_{\Pis(A,B)}A,p^*_{\Pis(A,B)}B'}$ to the map
  given by the universal property of the pullback, which provides us with the
  section
  \[\lambda_{p^*_{\Pis(A,B)}A,p^*_{\Pis(A,B)}B'}
  (1_{p^*_{\Pis(A,B)}A},
  f\cdot\app_{A,B})\colon
  \Gamma.\Pis(A,B)\rightarrow
  \Gamma.\Pis(A,B).\Pis(A,B').\]
  All we have to do now is postcompose with
  $q(p_{\Pis(A,B)},\Pis(A,B'))$.
\end{construction}

\begin{rmk}
  The previous construction is such that $\Gamma.\Pis(A,1_B)=1_{\Pis(A,B)}$ for
  every context $\Gamma.A.B$ in $\sfC$ if and only if the $\Pis$-structure
  satisfies the $\Pies$-rule.
\end{rmk}

