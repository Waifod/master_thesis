\chapter*{Categorical Models of Type Theory as Locally Cartesian Closed Fibration Categories}
\addcontentsline{toc}{chapter}{Locally Cartesian Closed Fibration Categories}

To apply the results from the previous section, we first need to specify a
fibrational structure on categorical models of type theory.

\begin{defn}
  A \emph{fibration category} is a triple $(\cP,W,Fib)$ where $\cP$ is a
  category and $W$, $Fib$ are wide subcategories such that:
  \begin{enumerate}
    \item $\cP$ has a terminal object;
    \item maps to the terminal object lie in $Fib$;
    \item $Fib$ and $W\cap Fib$ are closed under pullback along any map in $\cP$;
    \item every map in $\cP$ can be factored as a map in $W$ followed by one in
      $Fib$;
    \item $W$ has the 2-out-of-6 property.
  \end{enumerate}
\end{defn}

\begin{rmk}
  Seeing $\cP$, $W$ and $Fib$ in the above definition as $\infty$-categories, we
  notice that the triple has canonically the structure of an $\infty$-category
  with weak equivalences and fibrations with fibrant objects, hence we shall
  adopt the conventions we used in that context.
\end{rmk}

Now we have to specify the classes of maps which will provide the desired
structure.

\begin{defn}
  Given a categorical model of type theory $\cC$, a morphism
  $f\Gamma.A\rightarrow\Gamma.B$ over $\Gamma$ is \emph{simply bi-invertible
  over $\Gamma$} if there exist:
  \begin{enumerate}
    \item a morphism$g_1\colon\Gamma.B\rightarrow\Gamma.A$;
    \item a section $\eta\colon\Gamma.A\rightarrow\Gamma.(1_{\Gamma.A},g_1f)^*\Id_{A}$
      of the canonical projection
      $p_{(1_{\Gamma.A},g_1f)^*\Id_{A}}\colon
      \Gamma.(1_{\Gamma.A},g_1f)^*\Id_{A}\rightarrow\Gamma.A$;
    \item a morphism $g_2\colon\Gamma.B\rightarrow\Gamma.A$;
    \item a section
      $\epsilon\colon\Gamma.A\rightarrow\Gamma.(1_{\Gamma.A},fg_2)^*\Id_{A}$ of
      the canonical projection
      $p_{(1_{\Gamma.A},fg_2)^*\Id_{A}}\colon
      \Gamma.(1_{\Gamma.A},fg_2)^*\Id_{A}\rightarrow\Gamma.A$.
  \end{enumerate}
\end{defn}

We now generalize the above definition to general context extensions by working
as usual with $\cC^{cxt}$.

\begin{defn}
  Given a categorical model of type theory $\cC$, a morphism
  $f\colon\Gamma.\Delta\rightarrow\Gamma.\Theta$ over $\Gamma$ is
  \emph{bi-invertible over $\Gamma$} if it is as a morphism in $\cC^{cxt}$. It
  is simply called \emph{bi-invertible} when $\Gamma$ is the terminal
  context.
\end{defn}

\begin{defn}
  A fibration category $\cP$ is \emph{locally cartesian closed} if, for any
  fibration $p\colon a\rightarrow b$, the pullback functor
  $p^*\colon\cP\downarrow b\rightarrow\cP\downarrow a$ admits a right adjoint
  $p_*$ which is an exact functor.
\end{defn}

\begin{rmk}
  Our interest in bi-invertible morphisms stems from the fact that they model
  the right notion of invertible map in dependent type theory: indeed, from the
  required data for a simply bi-invertible map $f$ over $\Gamma$ we can provide
  a section $\Gamma\rightarrow\Gamma. isHIso(f)$ of the dependent projection
  $p_{\Gamma.isHIso(f)}\colon\Gamma.isHIso(f)\rightarrow\Gamma$ REFS, which is
  the translation into the language of contextual category of the notion of
  bi-invertible map in type theory MORE REFS.

  It is important to note that, while type theory has no way to encode
  internally the concept of isomorphism of the contextual model, it does have
  its own internal notion of isomorphism. However, given a map $f$, the type
  $isIso(f)$ is not, in general, a \emph{mere proposition}. On the other hand,
  the more lax notion of bi-invertibility is such that $isHIso(f)$ is a mere
  proposition \wfd{(UNDER WHICH HP? LEMMA 5.12 from 1203.3253)}, which makes it
  preferable. Also, every bi-equivalent map can be given the structure of an
  isomorphism and viceversa, hence they are closely related.
\end{rmk}

\begin{prop}\wfd{(REFS, 1507, AKL15, 1610.00037)}
  A categorical model of type theory $\cC$ carries the structure of a fibration
  category where maps isomorphic to dependent projections are the fibrations and
  bi-invertible ones are the weak equivalences.
\end{prop}

\wfd{(CAREFUL: AS STATED IN AKL15, WE ALSO NEED THE UNIT TYPE AND OTHER STUFF!
BUT KAPULKIN DOESN'T TAKE IT IN 1507)}

The above result can be however generalized.

\begin{prop}
  A contextual category with an $\Id$-structure $\cC$ carries the
  structure of a fibration category similar to the one above.
\end{prop}
\begin{proof}
  As noted in AKL15, 1507 and 1610.00037, the previous result does not depend on a
  $\Sigma$-structure because all we need is that every dependent projection is
  isomorphic to a basic one, which we have by working with context extensions in
  $\cC^{cxt}$. Indeed, after constructing the fibrational structure there, we may
  lift it back through the equivalence of categories $\cC\rightarrow\cC^{cxt}$.
\end{proof}

How does Kapulkin prove that a categorical model of Type Theory is a locally
cartesian closed fibration category?

First of all, he refers to AKL15 to show that $\cP$ has a fibrational structure,
then he goes on to show the following results, whose proofs are extremely terse
and therefore should be expanded.


\begin{defn}
  Given $p_A\colon\Gamma.A\rightarrow\Gamma$, a section
  $a\colon\Gamma\rightarrow\Gamma.A$ and $f\colon\Delta\rightarrow\Gamma$, we
  look at the commutative diagram
  \[\begin{tikzcd}
    \Delta \\
    & {\Delta.f^*A} & {\Gamma.A} \\
    & \Delta & \Gamma \\
    \arrow["{p_A}", from=2-3, to=3-3]
    \arrow["f"', from=3-2, to=3-3]
    \arrow["{p_{f^*A}}", from=2-2, to=3-2]
    \arrow["{q(f,A)}"', from=2-2, to=2-3]
    \arrow["\lrcorner"{anchor=center, pos=0.125}, draw=none, from=2-2, to=3-3]
    \arrow[curve={height=12pt}, Rightarrow, no head, from=1-1, to=3-2]
    \arrow["{a\cdot f}", curve={height=-12pt}, from=1-1, to=2-3]
    \arrow["{f^*a}"{description}, dotted, from=1-1, to=2-2]
  \end{tikzcd},\]
  which gives us $f^*a$ as the factorization through the pullback square of the
  pair $(\id_\Delta,a\cdot f)$.

  \wfd{(THIS CAN BE JUSTIFIED BY LOOKING AT LEMMA 2.15 IN 1706.03605. THE
  FOLLOWING RESULT SHOWS THAT THIS ASSIGNMENT IS ALSO FUNCTORIAL.)}
\end{defn}

\begin{lem}
  For any dependent projection $p_\Delta\colon\Gamma.\Delta\rightarrow\Gamma$ in
  a categorical model of type theory $\cC$, the pullback functor
  $p_\Delta^*\colon\cC(\Gamma)\rightarrow\cC(\Gamma.\Delta)$ between the fibrant
  slices admits a right adjoint.
\end{lem}
\begin{proof}
  Let's set $(p_\Delta)_*(\Gamma.\Delta.\Theta)=\Gamma.\Pi(\Delta.\Theta)$. Our
  counit shall be given by
  \[\epsilon_{\Gamma.\Delta.\Theta}\colon\Gamma.\Delta.p^*_\Delta\Pi(\Delta,\Theta)
  \xrightarrow{exch_{\Delta,\Pi(\Delta,\Theta)}}\Gamma.\Pi(\Delta,\Theta).p^*_{\Pi(\Delta,\Theta)}\Delta
  \xrightarrow{\app_{\Delta,\Theta}}\Gamma.\Delta.\Theta\]
  and it is then sufficient to prove that, for any context morphism
  $f\colon\Gamma.\Delta.p^*_\Delta\Psi\rightarrow\Gamma.\Delta.\Theta$ over
  $\Gamma.\Delta$, there is
  a unique $\tilde{f}\colon\Gamma.\Psi\rightarrow\Gamma.\Pi(\Delta,\Theta)$
  making the diagram
  \[\begin{tikzcd}
    {\Gamma.\Delta.p^*_\Delta\Psi} \\
    {\Gamma.\Delta.p^*_\Delta\Pi(\Delta,\Theta)} & {\Gamma.\Delta.\Theta}
    \arrow["f", from=1-1, to=2-2]
    \arrow["{\epsilon_{\Gamma.\Delta.\Theta}}"', from=2-1, to=2-2]
    \arrow["{p^*_\Delta\tilde{f}}"', dotted, from=1-1, to=2-1]
  \end{tikzcd}\]
  commute. This will then uniquely specify how the right adjoint acts on the
  morphisms.

  We start by specifying the unit
  $\eta_{\Gamma.\Psi}\colon\Gamma.\Psi\rightarrow\Gamma.\Pi(\Delta,p^*_{\Delta}\Psi)$.

  Let's consider the commutative square
  \[\begin{tikzcd}[column sep=huge]
    {\Gamma.\Psi.p^*_\Psi\Pi(\Delta,p^*_\Delta\Psi)} & {\Gamma.\Pi(\Delta,p^*_\Delta\Psi)} \\
    {\Gamma.\Psi} & \Gamma
    \arrow["{p_{\Pi(\Delta,p^*_\Delta\Psi)}}"', from=1-1, to=2-1]
    \arrow["{p_{\Psi}}"', from=2-1, to=2-2]
    \arrow["{p_{\Pi(\Delta,p^*_\Delta\Psi)}}", from=1-2, to=2-2]
    \arrow["{q(p_\Psi,\Pi(\Delta,p^*_\Delta\Psi))}", from=1-1, to=1-2]
  \end{tikzcd}\]
  where the map $q(p_\Psi,\Pi(\Delta,p^*_\Delta\Psi))$ acts by forgetting the
  term of $\Psi$. If we can provide a section of the vertical map on the left
  corresponding to the sequence $[\Gamma,y:\Psi,\lambda(x:\Delta).y]$ we are
  done as we can then compose
  it with $q(p_\Psi,\Pi(\Delta,p^*_\Delta\Psi))$ to get our unit.
  
  We construct it by looking at the commutative square
  \[\begin{tikzcd}
    {\Gamma.\Psi} \\
    & {\Gamma.\Psi.p^*_\Psi\Psi} & {\Gamma.\Psi} \\
    & {\Gamma.\Psi} & \Gamma
    \arrow["{p_\Psi}", from=2-3, to=3-3]
    \arrow["{p_\Psi}"', from=3-2, to=3-3]
    \arrow["{p_{p^*_\Psi\Psi}}", from=2-2, to=3-2]
    \arrow["{q(p_\Psi,\Psi)}"', from=2-2, to=2-3]
    \arrow[curve={height=12pt}, Rightarrow, no head, from=1-1, to=3-2]
    \arrow[curve={height=-12pt}, Rightarrow, no head, from=1-1, to=2-3]
    \arrow["{(1_\Psi,1_\Psi)}"{description}, dotted, from=1-1, to=2-2]
  \end{tikzcd},\]
  corresponding to the sequence $[\Gamma,x:\Psi,x:\Psi]$. Then, we pull back along
  $p_{p^*_\Psi\Delta}$, getting a section
  $p^*_{p^*_\Psi\Delta}(1_\Psi,1_\Psi)\colon\Gamma.\Psi.p^*_\Psi\Delta\rightarrow\Gamma.\Psi.p^*_\Psi\Delta.p^*_{p^*_\Psi\Delta}\Psi$.
  We then apply $\lambda$, which gives us a section
  \[\lambda(p^*_{p^*_\Psi\Delta}(1_\Psi,1_\Psi))=\lambda(1_{p^*_\Psi\Delta},p_{p^*_\Psi\Delta})
\colon\Gamma.\Psi\rightarrow\Gamma.\Psi.p^*_\Psi\Pi(\Delta,p^*_\Delta\Psi)\]
  and we can then conclude by post-composing with
  $q(p_\Psi,\Pi(\Delta,p^*_\Delta\Psi))\colon\Gamma.\Psi.p^*_\Psi\Pi(\Delta,p^*_\Delta\Psi)\rightarrow\Gamma.\Pi(\Delta,p^*_\Delta\Psi)$,
  which provides our unit.

  We then define our lift $\tilde{f}$ as the composite
  $\Gamma.\Pi(\Delta,f)\cdot\eta_{\Gamma.\Psi}$. The commutativity of the above
  triangle follows by $\beta$-reduction \wfd{(WE DO NOT HAVE IT!!!)}, while the
  uniqueness of the $\tilde{f}$
  giving the desired factorization by the $\Pi_\eta$-property. \wfd{(PLEASE
  PROVE IT)}

  This also shows that $(p_{\Delta})_*(f)=\Gamma.\Pi(\Delta,f)$, meaning that
  the construction of $\Gamma.\Pi(\Delta,f)$ is functorial.
\end{proof}

We know that every fibration in $\cC$ is isomorphic to a composite of dependent
projections , so this tells us that every fibration induces an adjunction
between fibrational slices \wfd{(YOU SHOULD DEFINE THEM, MAYBE AS INFTY-CATS OF
FIBRANT OBJECTS INDUCED FROM THE SLICES)}.

We are finally ready to prove the result leading to the final one we want.

\begin{prop}\label{lccfc}
  A categorical model of type theory $\cC$ is a locally cartesian closed
  fibration category.
\end{prop}
\begin{proof}
  Kapulkin 1507.02648, Proposition 5.4.

  We already know that it is a
  fibration category by PREVIOUS RESULT. Also, for any basic dependent projection
  $\Gamma.\Delta\rightarrow\Gamma$, the pullback functor between fibrant slices
  $p^*_\Delta$ has a right adjoint $(p_\Delta)_*$ by the previous lemma.
  We only need to check that this right adjoint is
  compatible with the fibrational structure.

  As a right adjoint, $(p_\Delta)_*$ preserves limits and in particular pullbacks and
  the terminal object. Also, by LEMMA ABOUT PI OBJECTS BEING EQUAL, it preserves
  dependent projections and, by LEMMA ABOUT PRESERVING BI-INVERTIBLE MAPS, weak
  equivalences.
\end{proof}

\begin{thm}
  Given a categorical model of type theory $\cC$, the $\infty$-category $L(\cC)$
  is locally cartesian closed.
\end{thm}
\begin{proof}
  Since a fibration category is more generally a $\infty$-category with
  fibrations and weak equivalences, we can apply \ref{7616} as the hypothesis
  are satisfied by \ref{lccfc}.
\end{proof}


