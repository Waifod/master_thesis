\documentclass{minimal}

%\usepackage{tikz-cd}
\usepackage[utf8]{inputenc}
\usepackage[british]{babel}
%\setbeamertemplate{theorems}[numbered]


\newcommand{\Id}{Id}


\begin{document}
The objective of this thesis is to provide a more modern proof of a result by
Kapulkin stating that the quasi-categorical localization of a categorical model
of type theory is a locally cartesian closed $\infty$-category. Henceforth,
I will say ``localization'' to refer to the $\infty$-categorical one.

Dependent type theory talks about dependent types $A$ over contexts $\Gamma$ and
their terms. It provides structural rules telling us that we can
substitute definitionally equal variables and types. There are then logical
rules specifying how to construct the objects we talk about, in particular
$\Sigma$-types $\Sigma(A,B)$, that is the type of pairs $(a,b)$ of elements,
$\Pi$-types $\Pi(A,B)$, the type of maps from $A$ to $B$, and $\Id$-types
$\Id_A$, the type of proof of equalities between terms of $A$.

To study a theory, logicians work with its models, but here it's hard:
structural rules require lots of checks. A solution is defining a class of
algebraic models with simple axioms.

Such a class is given by contextual categories, categories specifying for each
object a length and a father function to reduce them. Each object $\Gamma.A$ has
a map to its reduction $p_A$ which we call a basic dependent projection. The
notation specifies that $\Gamma.A$ extends $\Gamma$ by 1 and $p_A$ wants to
express that $A$ is a dependent type in context $\Gamma$. Also, we want
functorial pullbacks of basic dependent projections because they
describe substitutions, which are strictly associative. This structure models
the structural rules.

To model the logical rules we need some extra structure. For $\Id$-types, for
$\Gamma.A$ we need an identity object, for $\Pi$-types we want a hom object,
a map describing function evaluation and so on. What we are interested in are
categorical models of type theory, that is contextual categories $C$ with
$\Sigma$, $\Id$ and $\Pi$ structures.

We want to talk about localizations. The right notion of invertibility in
dependent type theory is bi-invertibility and for a map
$f\colon\Gamma\rightarrow\Delta$ in a contextual category to model it it needs
maps $g_1,g_2$ in the opposite direction and maps $\eta,\epsilon$ which tell us
that they are pointwise left and right inverse to $f$. Thinking about types as
spaces, as suggested by the groupoidal model, identity types are path spaces,
hence $\eta$ and $\epsilon$ are more like homotopies. What if we localize at
bi-invertible maps? To answer that question we need some tools.

An $\infty$-category $C$ with weak equivalences $W$ and fibrations $Fib$ is a
triple generalizing the 1-categorical notion of fibration category. Essentially,
we have some stability under pullbacks of fibrations, trivial fibrations, the
factorization condition and other things. A contextual category with $\Sigma$
and $\Id$ structures is one if we take bi-invertible maps as weak equivalences
and maps isomorphic to dependent projections as fibrations, so this also applies
to categorical models of type theory.

Cisinski's theory of localizations tells us that localizing such an
$\infty$-category at weak equivalences always gives us a finitely complete
$\infty$-category, but also if there is a right adjoint to the pullback functor
induced by a fibration $f$ from x to y, both fibrant, on the fibrant slices they
define and the right adjoint preserves trivial fibrations, then the localization
is also locally cartesian.

We now state the theorem we mentioned at the beginning. Given a categorical
model of type theory $C$, its localization is a locally cartesian closed
$\infty$-category. It is enough to construct a right adjoint to the pullback
functor induced by $p_A$ from $\Gamma.A$ to $\Gamma$, which we do by sending
$\Gamma.A.\Theta$ to $\Gamma.\Pi(A,\Theta)$ and using the function evaluation
map as a counit. Here the $\Pi$-object on the right
is defined by extending the basic $\Pi$-structure.

Why is dependent type theory interesting? It's a foundation for mathematics
which has become of interest because it is closely linked to computer science
and it allows the creation of proof assistants like Agda and Lean, that is
software which allows us to formalize mathematical reasoning, easily check
proofs and even write some tedious bits automatically. There are already ongoing
projects to formalize all of mathematics, like the math library. Secondly, this
theory merges the two strata of classical foundations into one, generalizing
both sets and propositions through types. Thirdly, we can reason about proofs
within the theory, without a metatheory. Finally, equality is more structured:
only objects of the same type can be compared and a proof of equality specifies
in what sense two objects are equal.

Why is this interesting? Because it goes towards using $\infty$-categories to
reason about dependent types and, viceversa, using dependent types to reason
about $\infty$-categories, something which folkloristically we know to be
doable. In 2016, Kapulkin and Lumsdaine conjectured that the functor induced by
localizing contextual categories modeling some dependent type theories
would induce equivalences between their $\infty$-categories of contextual
categories and $\infty$-categories of finitely complete or locally cartesian
closed $\infty$-categories, an equivalence which hopefully will extend to
Homotopy Type Theory and Elementary $\infty$-toposes.

\end{document}
