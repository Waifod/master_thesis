\documentclass{beamer}
\usetheme{Copenhagen}

%\usepackage{tikz-cd}
\usepackage{quiver}
\usepackage{amsthm}
\usepackage{eucal}
\usepackage[utf8]{inputenc}
%\setbeamertemplate{theorems}[numbered]



%Information to be included in the title page:
\title{Localizations of Models of Dependent Type Theory}
\author{Author: Matteo Durante \quad\quad\quad Advisor: Hoang-Kim Nguyen}
\institute{Regensburg University}

\DeclareMathOperator{\Id}{\mathsf{Id}}
\DeclareMathOperator{\sfC}{\mathsf{C}}
\DeclareMathOperator{\N}{\mathbb{N}}
\DeclareMathOperator{\cC}{\mathcal{C}}
\DeclareMathOperator{\cxl}{CxlCat}
\DeclareMathOperator{\HoTT}{HoTT}
\DeclareMathOperator{\ElTopos}{ElTopos}
\DeclareMathOperator{\lexi}{Lex_\infty}
\DeclareMathOperator{\lccci}{LCCC_\infty}
\DeclareMathOperator{\Ob}{\mathsf{Ob}}
\DeclareMathOperator{\ft}{ft}
\DeclareMathOperator{\app}{\mathsf{app}}
\DeclareMathOperator{\nat}{Nat}


\begin{document}

\theoremstyle{plain}

\newtheorem{thm}{Theorem}[section]
\newtheorem{prop}{Proposition}[section]
\newtheorem{defn}{Definition}[section]
\newtheorem{conj}{Conjecture}[section]
\newtheorem{lem}{Lemma}[section]

\frame{\titlepage}

\begin{frame}
  \frametitle{Objective}

  A modern proof of the following theorem.

  \begin{thm}[Kapulkin 2015]
    Given a categorical model of type theory $\sfC$, its $\infty$-categorical
    localization $L(\sfC)$ is a locally cartesian closed $\infty$-category.
  \end{thm}
\end{frame}

\begin{frame}
  \frametitle{Dependent Type Theory}
  
  \begin{block}{What}
    A theory of computations and a foundation of mathematics.
  \end{block}
  \pause

  \begin{block}{Objects}
    \emph{Dependent types $A$ over contexts
    $\Gamma$} and their \emph{terms $x:A$}.
  \end{block}
  \pause

  \begin{block}{Structural rules}
    How to work with \emph{variables}.
  \end{block}
  \pause

  \begin{block}{Logical rules}
    Construct new types and their terms from old, carry out
    computations, provide \emph{$\Sigma$-types} $\Sigma(A,B)$, \emph{$\Pi$-types}
    $\Pi(A,B)$ and \emph{$\Id$-types} $\Id_A$, \emph{natural-numbers-type}
    $\nat$...
  \end{block}
\end{frame}

\begin{frame}
  \frametitle{Models}

  \begin{block}{Problem}
    Providing a model of Dependent Type Theory is hard.
  \end{block}
  \pause

  \begin{block}{Solution}
    Defining a class of algebraic models.
  \end{block}
\end{frame}

\begin{frame}
  \frametitle{Modeling structural rules}

  \begin{definition}[contextual categories]
    A category $\sfC$ with:
    \begin{enumerate}
      \item a grading on objects $\Ob\sfC=\coprod_{n\in\N}\Ob_n\sfC$;
      \item a map $\ft_n\colon\Ob_{n+1}\sfC\rightarrow\Ob_n\sfC$ for each
        $n\in\N$;
      \item \emph{basic dependent projections}
        $p_A\colon\Gamma.A\rightarrow\ft_n(\Gamma.A)=\Gamma$;
      \item a functorial choice of pullback squares
        \[\begin{tikzcd}[ampersand replacement=\&]
          {\Delta.f^*A} \& {\Gamma.A} \\
          \Delta \& \Gamma
          \arrow["{p_A}", from=1-2, to=2-2]
          \arrow["{p_{f^*A}}"', from=1-1, to=2-1]
          \arrow["{q(f,A)}", from=1-1, to=1-2]
          \arrow["f"', from=2-1, to=2-2]
        \end{tikzcd};\]
      \item ...
    \end{enumerate}
  \end{definition}
\end{frame}

\begin{frame}
  \frametitle{Modeling logical rules}

  \begin{block}{Extra structure}
    $\Id$-types require from $\Gamma.A$ an object $\Gamma.A.A.\Id_A$...
    
    $\Pi$-types require from $\Gamma.A.B$ an object $\Gamma.\Pi(A,B)$, a map
    $\app_{A,B}\colon\Gamma.\Pi(A,B).A\rightarrow\Gamma.A.B$...
  \end{block}
  \pause

  \begin{defn}
    A \emph{categorical model of type theory} is a contextual category $\sfC$
    with $\Sigma$, $\Id$ and $\Pi$ structures.
  \end{defn}
\end{frame}

\begin{frame}
  \frametitle{Bi-invertibility}

  \begin{defn}[Bi-invertible map]
    A map $f\colon\Gamma\rightarrow\Delta$ in a contextual category with
    $\Id$-structure $\sfC$ for which we can provide:
    \begin{enumerate}
      \item maps $g_1\colon\Delta\rightarrow\Gamma$,
        $\eta\colon\Gamma\rightarrow\Gamma.(1_\Gamma,g_1\cdot f)^*\Id_\Gamma$;
      \item maps $g_2\colon\Delta\rightarrow\Gamma$,
        $\epsilon\colon\Delta\rightarrow\Delta.(1_\Delta,f\cdot g_2)^*\Id_\Delta$.
    \end{enumerate}
  \end{defn}
  \pause

  \begin{block}{Question}
    What if we localize at bi-invertible maps?
  \end{block}
\end{frame}

\begin{frame}
  \frametitle{Fibrational structure}
  
  \begin{defn}[$\infty$-categories with weak equivalences and fibrations]
    A triple $(\cC,W,Fib)$ where:

    ...a weakening of the definition of fibration categories, with $\cC$ an
    $\infty$-category.
  \end{defn}
  \pause

  \begin{thm}[Avigad-Kapulkin-Lumsdaine 2013]
    A contextual category with $\Sigma$ and $\Id$ structures defines a
    fibration category, where weak
    equivalences are bi-invertible maps and
    fibrations are maps isomorphic to  dependent projections.
  \end{thm}
\end{frame}

\begin{frame}
  \frametitle{Localizing fibrational categories}

  \begin{prop}[Cisinski]
    The localization at weak equivalences of an $\infty$-category with weak
    equivalences and fibrations $\cC$ is a finitely complete $\infty$-category.
  \end{prop}

  \begin{prop}[Cisinski]
    Given an $\infty$-category with weak equivalences and fibrations $\cC$, if
    for every fibration $f\colon x\rightarrow y$ between fibrant objects the
    pullback functor between fibrant slices $f^*\colon\cC(y)\rightarrow\cC(x)$
    has a right adjoint preserving trivial fibrations, then $L(\cC)$ is locally
    cartesian closed.
  \end{prop}
\end{frame}

\begin{frame}
  \frametitle{Localizations of models are Cartesian closed}

  \begin{thm}[Kapulkin 2015]
    Given a categorical model of type theory $\sfC$, its localization
    $L(\sfC)$ is a locally cartesian closed $\infty$-category.
  \end{thm}
  \pause

  \begin{proof}
    For any basic dependent projection $p_A\colon\Gamma.A\rightarrow\Gamma$,
    there exists a right adjoint to
    $p_A^*\colon\sfC(\Gamma)\rightarrow\sfC(\Gamma.A)$ given by
    \[(p_A)_*(\Gamma.A.\Theta)=\Gamma.\Pi(A,\Theta)\]
    with counit induced by $\app_{A,\Theta}$. It preserves the fibrational
    structure.
  \end{proof}
\end{frame}

\begin{frame}
  \center{Thank you for your attention!}
\end{frame}

\begin{frame}
  \frametitle{Why is Dependent Type Theory cool?}

  \begin{enumerate}
    \item Closely linked to \emph{computations} and \emph{computer science},
      makes proof assistants possible
    \item Enough by itself as a foundation, unlike Set Theory or Propositional
      Calculus
    \item \emph{Proofs} are internal objects
    \item Better treatment of \emph{equality}
  \end{enumerate}
\end{frame}

\begin{frame}
  \frametitle{Internal Languages Conjecture}

  \begin{conj}[Kapulkin-Lumsdaine 2016]
    The horizontal maps, given by simplicial localization, induce
    equivalences of $\infty$-categories.
    \[\begin{tikzcd}[ampersand replacement=\&]
      {\cxl_{\Sigma,1,\Id,\Pi}} \& {\lccci} \\
      {\cxl_{\Sigma,1,\Id}} \& {\lexi}
      \arrow[from=1-1, to=2-1]
      \arrow[from=1-2, to=2-2]
      \arrow[from=2-1, to=2-2]
      \arrow[from=1-1, to=1-2]
    \end{tikzcd}\]
  \end{conj}

  A proof by Nguyen-Uemura has recently become available on arxiv.

  One hopes to extend this to an equivalence between $\cxl_{\HoTT}$ and
  $\ElTopos_\infty$.
\end{frame}

\end{document}

