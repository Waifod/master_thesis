\documentclass[a4paper,12pt]{scrartcl}
\usepackage{reg2021}
\usepackage{quiver}
\usepackage{bbold}

\begin{document}
\noindent\textbf{Author}\hfill\textbf{Year} \linebreak
\vspace*{-.1cm} Matteo Durante\hfill 2022 \\

\noindent
\rule{\linewidth}{1pt}
\begin{center}
\Large
    \textbf{On Fibration Categories and\\ Finitely Complete $(\infty,1)$-categories} \\
\end{center}
\rule{\linewidth}{1pt}
\\

%%%%%%%%%%%%%%%%%%%%%%%%%%%%%%%%%%%%%%%%%%%%%%%%%%%%%%%%%%%%%%%%%%%%%%%%%%%%%%%

\newcommand{\La}{\Lambda}
\newcommand{\pa}{c}
\newcommand{\ob}{\operatorname{Ob}}
\newcommand{\mor}{\operatorname{Mor}}
\newcommand{\sto}{\twoheadrightarrow}

\newcommand{\plim}{\varprojlim}
\newcommand{\sst}{\subseteq}
\newcommand{\eq}{\operatorname{eq}}

\newcommand{\f}{\varphi}

\newcommand{\sing}{\operatorname{Sing}}

\newcommand{\ihom}{\underline{\Hom}}

\section{This is [not] an outline}

The goal of this thesis is to redefine the functor: $F\colon
N(\FibCat)\rightarrow\Lexi$, from the nerve of the homotopical category of
fibration categories to $\infty$-category of finitely complete
$\infty$-categories and left exact functors, and prove that it induces an
equivalence after localizing the domain at its weak equivalences.

The functor $F$ is simply defined as the composite of the full embedding
$\iota\colon N(\FibCat)\rightarrow \FibCat_\infty$, induced by the nerve
functor, \wfd{(should we instead work with a stronger notion of fibration
$\infty$-category? also we need to define this $\infty$-category)} and the functor $\Hoi\colon
\FibCat_\infty\rightarrow\Lexi$ given by localizing fibration
$\infty$-categories. We know by \cite[Thm.\ 7.5.6]{Cis19} that localizing a
fibration $\infty$-category gives a finitely complete $\infty$-category and the
mapping on the maps is specified by the universal property of localizations.
A map between fibration $\infty$-categories is a left exact functor as defined
in \cite[Def.\ 7.5.2]{Cis19} and \cite[Thm.\ 7.5.28]{Cis19} guarantees that it
is indeed sent to a left exact functor, as needed.

To prove that $F$ induces the desired equivalence of $\infty$-categories we
define a (weak) fibrational structure on $\FibCat$ and $\FibCat_\infty$ and
verify that the right derived functors associated to $\iota$ and
$\Hoi$ are equivalences.

We already have notions of weak equivalence and fibration
internal to $\FibCat$ as provided in \cite{KS19}, so we only need to specify
ones internal to $\FibCat_\infty$ satisfying the definition of (weak)
fibration category given in \cite[Def.\ 7.4.12]{Cis19} (PLEASE LIST THE
PROPERTIES). To do so, we try to extend the definition given in the
1-categorical context, so that the functor $\iota$ will preserve the desired
structure. Hopefully we can just take the same definitions, without adaptations.

To show that the right derived functor of $\Hoi$ induces an equivalence we
remember that a finitely
complete $\infty$-category has a canonical fibrational structure, where weak
equivalences are the isomorphisms and fibrations are all the maps. Also, under
this construction a left exact functor between the induced fibration $\infty$-categories
is just a left exact functor in the usual sense, so we can fully embed
$\Lexi$ into $\FibCat_\infty$. This should define a right adjoint to
$\Hoi$, where the counit is the identity and the unit is given by the
localization maps $\cP\rightarrow L(\cP)$. We observe that such maps are weak
equivalences in $\FibCat_\infty$, so localizing this $\infty$-category gives us
an equivalence $L(\FibCat_\infty)\cong\Lexi$ involving the right derived
functor of $\Hoi$, as we desired.

Once we have done this, we need to check that $\iota$ is left exact, i.e.\ it
satisfies \cite[Def.\ 7.5.2]{Cis19} (PLEASE LIST THE PROPERTIES), for which we
only need condition (iii). It should be easy to check because it is induced by
the nerve functor.

To finally apply \cite[Thm.\ 7.6.15]{Cis19} to $\iota$ we still need to check
two more conditions (PLEASE LIST THEM). Notice that (a) is trivial, so we
actually only need to check (b).

\section{This is [not] a fibration}

One piece of the issue is to define a fibration $\infty$-category structure on
$\FibCat_\infty$ and to do so we need to first provide the definitions of the
elements at play.

\begin{defn}
  A \emph{fibration category} is a category $\cP$ with a subcategory whose
  morphisms are called \emph{weak equivalences} (denoted by
  $\xrightarrow{\sim}$) and one of \emph{fibrations} (denoted by
  $\twoheadrightarrow$) subject to the following axioms:
  \begin{enumerate}
    \item $\cP$ has a terminal object 1 and all objects are fibrant, that is
      every map to the terminal object is a fibration;
    \item pullbacks along fibrations exist in $\cP$ and (acyclic) fibrations are
      stable under pullback;
    \item every morphism in $\cP$ factors as a weak equivalence followed by a
      fibration;
    \item weak equivalences satisfy the 2-out-of-6 property.
  \end{enumerate}
\end{defn}

\begin{defn}
  A functor between fibration categories is \emph{left exact} if it preserves weak
  equivalences, fibrations, terminal objects and pullbacks along fibrations.
\end{defn}

Notice that this corresponds to the definition of \emph{exact functor} given
in \cite[Def.\ 2.7]{KS19}, however it also coincides with the one given in the
more general context of fibration $\infty$-categories in \cite[Def.\
7.5.2]{Cis19} (whose results we rely on), therefore to avoid conclusion I
adopted the naming given in the latter.

The category $\FibCat$ is then constructed by taking small fibration categories
as objects and left exact functors as arrows.

\begin{defn}
  A morphism in $\FibCat$ is a \emph{weak equivalence} if it induces an
  equivalence of categories on the associated homotopy categories.
\end{defn}

\begin{defn}
  A morphism $P\colon\cE\rightarrow\cD$ in $\FibCat$ is a \emph{fibration} if:
  \begin{enumerate}
    \item it is an isofibration;
    \item it lifts weak factorizations;
    \item it lifts pseudo-factorizations.
  \end{enumerate}
\end{defn}

The three conditions and the map being an isofibration are necessary to ensure
that a fibration of fibration categories is sent to a fibration in $\Lexi$.

Notice that this is just taken from \cite[Def.\ 4.3]{KS19} and, under our
definition, every map to the terminal fibration category is a fibration.

\begin{prop}
  These classes of maps specify a fibration category structure on $\FibCat$.
\end{prop}
\begin{proof}
  Check \cite{KS19}.
\end{proof}

We want to switch to the $\infty$-categorical context, so let's introduce the
corresponding notions.

\begin{defn}
  A fibration $\infty$-category is a triple $(\cC,W,Fib)$ where $W\subset\cC$
  is a sub-$\infty$-category with the 2-out-of-3 property and $Fib$ is a class
  of fibrations such that:
  \begin{enumerate}
    \item for any pullback square
      \[\begin{tikzcd}
        {x'} & x \\
        {y'} & y
        \arrow["p", from=1-2, to=2-2]
        \arrow["{p'}"', from=1-1, to=2-1]
        \arrow["v"', from=2-1, to=2-2]
        \arrow["u", from=1-1, to=1-2]
      \end{tikzcd}\]
      in $\cC$ in which $p$ is a fibration between fibrant objects, if $p\in W$
      then $p'\in W$;
    \item given a map $f\colon x\rightarrow y$ in $\cC$ with $y$ fibrant, there
      exists a map $w\colon x\rightarrow x'$ in $W$ and a fibration $p\colon
      x'\rightarrow y$ that $f=p\cdot w$.
  \end{enumerate}
  The maps in $W$ are generally called \emph{weak equivalences}, while
  fibrations which are also in $W$ are \emph{trivial fibrations}.
\end{defn}

\begin{defn}
  Let $\cC,\cD$ be fibration $\infty$-categories. A functor
  $F\colon\cC\rightarrow\cD$ is \emph{left exact} if it has the following
  properties:
  \begin{enumerate}
    \item it preserves terminal objects;
    \item it preserves fibrations and trivial fibrations;
    \item given a pullback square
      \[\begin{tikzcd}
        {x'} & x \\
        {y'} & y
        \arrow["p", from=1-2, to=2-2]
        \arrow["{p'}"', from=1-1, to=2-1]
        \arrow["v"', from=2-1, to=2-2]
        \arrow["u", from=1-1, to=1-2]
      \end{tikzcd}\]
      in $\cC$ where $p$ is a fibration and $y$ and $y'$ are fibrant, the square
      \[\begin{tikzcd}
        {Fx'} & Fx \\
        {Fy'} & Fy
        \arrow["Fp", from=1-2, to=2-2]
        \arrow["{Fp'}"', from=1-1, to=2-1]
        \arrow["Fv"', from=2-1, to=2-2]
        \arrow["Fu", from=1-1, to=1-2]
      \end{tikzcd}\]
      is a pullback in $\cD$.
  \end{enumerate}
\end{defn}

Let's then consider $\FibCat_\infty$, the $\infty$-category of small fibration
$\infty$-categories and left exact functors. We can take the same definition of
fibration for $\FibCat_\infty$ and use the following one for weak equivalences, which then should (CHECK!)
make it a fibration $\infty$-category.

\begin{defn}
    A left exact functor between fibration $\infty$-categories
    $F\colon\cP\rightarrow\cQ$ is a weak equivalence if it induces an
    equivalence $ho(L(\cP))\rightarrow ho(L(\cQ))$. Equivalently, if it
    satisfies the conditions of \cite[Thm. 7.6.15]{Cis19}.
\end{defn}

\begin{prop}
  (DESIDERATA) The $\infty$-category $\FibCat_\infty$ with the specified classes
  of maps is a fibration $\infty$-category.
\end{prop}
\begin{proof}
  The problem is showing the factorization condition \wfd{(SHOW IT)}. Fibrations
  should be stable under pullback (which are computed as in $\sSet$ if we manage
  to describe $\FibCat_\infty$ as a full subcategory of functors with codomain
  in $\Cat_\infty$) and the same applies to trivial fibrations, with the proof
  being the same as the one given in the 1-categorical context. Every map to the
  terminal fibration $\infty$-category is naturally a fibration.
\end{proof}

\section{This is [not] an adjunction}

In this section we provide a right adjoint to the localization functor $\Hoi$.

\begin{rmk}
  Given a finitely complete $\infty$-category, we can provide a fibrational
  structure on it by defining weak equivalences to be the isomorphisms and
  fibrations to be all of the maps. Under this construction we see that, given
  two finitely complete $\infty$-categories, a functor between the induced
  fibration $\infty$-categories is left exact if and only if it is left exact
  when seen as a functor between the underlying $\infty$-categories.
\end{rmk}

\begin{prop}\label{adj}
  There is an adjunction $\Hoi\colon\FibCat_\infty\rightleftarrows\Lexi\colon
  i$, where $i$ is a full embedding, the unit is given by the
  localization maps $\gamma_{\cP}\colon\cP\rightarrow L(\cP)$ and the counit is
  the identity. This exhibits $\Lexi$ as a reflexive sub-$\infty$-category of
  $\FibCat_\infty$.
\end{prop}
\begin{proof}
  Let's consider for any finitely complete $\infty$-category $\cC$ the pullback
  diagram
  \[\begin{tikzcd}
    {\Hoi/\cC} & {\Lexi/\cC} \\
    {\FibCat_\infty} & {\Lexi}
    \arrow["\Hoi"', from=2-1, to=2-2]
    \arrow[from=1-2, to=2-2]
    \arrow[from=1-1, to=2-1]
    \arrow[from=1-1, to=1-2]
  \end{tikzcd}.\]
  We see that $\Hoi/\cC$ has a terminal object, namely the pair $(\cC,\id_\cC)$,
  where $\cC$ has the fibrational structure specified above: indeed, given any
  other object $(\cP,F)$, the space of maps
  $(\Hoi/\cC)((\cP,F),(\cC,\id_\cC))$ is contractible as it corresponds to the
  fiber at $F$ of the equivalence
  $\FibCat_\infty(\cP,\cC)\cong\Lexi(L(\cP),\cC)$ given by \cite[Prop.\
  7.5.11]{Cis19}.

  Unravelling everything, the constructed right adjoint is then the full
  embedding mapping everything in $\Lexi$ to its copy in $\FibCat_\infty$.
\end{proof}

Given this adjunction we present the following result, which reduces the
problem of proving that the functor we constructed initially between $L(\FibCat)$
and $\Lexi$ is an equivalence to showing it for the right derived functor of
$\iota$.

\begin{prop}
  The right derived functor of $\Hoi$ is an equivalence of $\infty$-categories
  $L(\FibCat_\infty)\cong\Lexi$.
\end{prop}
\begin{proof}
  By \ref{adj} we know that $\Hoi$ has a fully faithful right adjoint, so by
  \cite[Prop.\ 7.1.18]{Cis19} it is enough to show that a map in
  $\FibCat_\infty$ is mapped by $\Hoi$ to an isomorphism if and only if it is a
  weak equivalence. By \cite[Prop.\ 7.6.11]{Cis19} a left exact functor between
  fibration $\infty$-categories induces an equivalences on the localizations if
  and only if it does so on the homotopy categories, which is also the condition
  under which it is a weak equivalence.
\end{proof}

\section{This is [not] a \texorpdfstring{$\infty$}{∞}-category}

Our approach relies on defining a $\infty$-category $\FibCat_\infty$. How to do this?

1- Construct a $\infty$-category where objects are pairs of functors (specifying weak equivalences and fibrations) with the same codomain between $\infty$-categories, that is $\Fun(\{0\rightarrow 01\leftarrow 1\},\Cat_\infty)$. Then the morphisms are triples of functors making the proper diagram commute. We take the full subcategory of objects satisfying the axioms of a $\infty$-category with weak equivalences and fibrations, but how can we ask for pullbacks along fibrations to be preserved by the maps in this $\infty$-category? We also have to show that the hom-spaces are what we want. How to show then that a pullback along a fibration is a fibration? It may not be immediate because we are considering only triples of functors which make the diagrams commute as morphisms!

What if we take pairs of functors $\coprod_W\Delta^1\rightarrow\cD$,
$Fib\rightarrow\cC^{[1]}\rightarrow\cC$, where the composite is a Cartesian
fibration and $Fib\rightarrow\cC^{[1]}$ preserves Cartesian morphisms?
Essentially this is the definition of comprehension category, plus some extra
data given by the first functor. We call fibrations the maps in the image of
$Fib\rightarrow\cC$ and their compositions, while weak equivalences are the
maps identified by the first functor; we then ask for the axioms of fibration
$\infty$-category to be satisfied. This gives us more freedom when specifying
weak equivalences then just providing the definition we give when constructing a
tribe from a comprehension category. Also, in this way weak equivalences are
preserved.

Problem: maps between such objects may not be what we want! Namely, I want the
functor $\cC^{[1]}\rightarrow\cD^{[1]}$ to be induced by post-composing with
$\cC\rightarrow\cD$, but I don't know how to specify this. This (with the
condition of $Fib\rightarrow\cC$ being a cartesian fibration and the
factorization $Fib\rightarrow\cC^{[1]}\rightarrow\cC$ preserving cartesian
morphisms) would guarantee that pullbacks along fibrations are preserved by maps
between fibration categories.

2- Take the 1-category of $\infty$-categories with fibrations and weak
equivalences, then localize at categorical equivalences. What do the hom-spaces
look like? If instead we localized directly at weak equivalences we would get
directly $L(\FibCat_\infty)$, which supposedly is the ``true'' $\infty$-category
of $\infty$-categories with fibrations and weak equivalences.

It's important to relate the hom-spaces $\FibCat_\infty(\cP,\cQ)$ to the ones we
want, that is $k(\Fun_{lex}(\cP,\cQ))$. Also, what are the ones of $\Lexi$ like?
Supposedly $k(\Fun_{lex}(\cC,\cD))$ and we may show both claims using
\cite[Prop.\ 7.10.6]{Cis19}, but I do not think it would be easy.

\section{Other ideas}

1- We could also try to get an indirect comparison by looking at the inclusion of the category of fibration categories and of the category of finitely complete $\infty$-categories into the one of $\infty$-categories with weak equivalences and fibrations. We then give to the latter the structure of a fibrations category by taking the definition above. The axioms should be easy to show because a pullback along a fibration will simply be a pullback in $\sSet$ and then we can give it the desired structure as in the context of fibration categories. Problem: showing the factorization condition.

We have then to show that the inclusion functors are left exact as functors
between fibration categories. This means that the pullback of left exact
functors between finitely complete $\infty$-categories is again a finitely
complete $\infty$-category, which is proven in \cite[Lem.\ 2.11]{Szu17b}.

After this we have to show that the hypothesis of \cite[Thm.\ 7.6.15]{Cis19} hold. This is easy for (i), but for (ii) it's hard when it comes to the inclusion of fibration categories into $\infty$-categories with fibrations and weak equivalences. Namely, what if the domain of the morphism is a $\infty$-category? We have to find a span linking it to a weakly equivalent fibration category!

2- We can work with the 1-category of fibration $\infty$-categories and left
exact functors and give it a fibrational structure. Problems: how do we show
that after localizing a fibration of fibration categories we get an inner
fibration? That is needed for the localization functor to be left exact and
maybe we can use the ideas from \cite[Prop.\ 3.5]{Szu17b}. Also, how to show
condition (ii) of \cite[Thm.\ 7.6.15]{Cis19} for the inclusion of fibration
categories into fibration $\infty$-categories? We need to approximate a map from
a fibration $\infty$-category to a fibration category by one between fibration
categories, the same problem as before. Everything else should be easy.

3- Starting from the fibration categories $\FibCat$ and $\Lexi$ and the functor
$\Hoi$ between them we can show the approximation properties of \cite[Thm.\
7.6.15]{Cis19}. The problems are the same as in (2), that is  how can we prove
that localizing a fibration in $\FibCat$ gives a fibration in $\Lexi$? Also, the
approximation property (i) is easy, while for (ii) given a morphism
$\cC\rightarrow L(\cP)$, we may try to consider a finitely complete
<<<<<<< HEAD
category $(\sSet_0/\cC)^{\op}$, whose fibrational structure is the dual of
the cofibrational one specified in \cite[Def.\ 4.1]{Szu17b}, specifically the
fibrations are given by monomorphisms and weak equivalences by maps sent to
isomorphisms under the functor $\lim\colon(\sSet_0/\cC)^{\op}\rightarrow\cC$;
the proof that this is indeed a fibration category is the same as the one
concerning $\sSet_\kappa/\cC$ provided in \cite[Prop.\ 4.2]{Szu17b}. If the map
$\widetilde{\lim}\colon L(\sSet_0/\cC)^{\op}\rightarrow\cC$ is a weak
equivalence (check the following proposition!), then we get the desired
commutative square by looking at the category $\sSet_0/\cC$, whose fibrational
structure is essentially the dual of
the one specified in \cite[Def.\ 4.1]{Szu17b}, specifically the fibrations are
given by monomorphisms and weak equivalences by maps sent to isomorphisms under
the functor $\lim\colon(\sSet_0/\cC)^{\op}\rightarrow\cC$; the proof that this
is indeed a fibration category is the same as the one concerning
$\sSet_\kappa/\cC$ provided in \cite[Prop.\ 4.2]{Szu17b}. If the map
$\widetilde{\lim}\colon L(\sSet_0/\cC)^{\op}\rightarrow\cC$ is a weak equivalence,
then we get the desired commutative square by looking at
\[\begin{tikzcd}
	\cC & {L(\cP)} \\
  {L(\sSet_0/\cC)}^{\op} & {L(\sSet_0/\cC)^{\op}}
	\arrow[from=2-2, to=1-2]
	\arrow[from=1-1, to=1-2]
  \arrow["\widetilde{\lim}", from=2-1, to=1-1]
	\arrow[Rightarrow, no head, from=2-1, to=2-2]
\end{tikzcd}\]
and somehow lift the map $L(\sSet_0/\cC)^{\op}\rightarrow L(\cP)$ through $L$,

\wfd{(I do not believe this lift to always exist. Indeed, take the $\infty$-groupoid
$\cC=N(\{0\leftrightarrows 1\})$, $\cP=[1]$ with the minimal fibrational
structure making $0\rightarrow 1$ a weak equivalence and the identity as the
upper map. Given in $(\sSet/\cC)^{\op}$ the objects
$0\colon\Delta^0\rightarrow\cC$, $(1\rightarrow 0)\colon\Delta^1\rightarrow\cC$
and the obvious morphism $f$ between them, we see that the lifted map
$(\sSet/\cC)^{\op}\rightarrow\cP$ would send $f$ to a morphism with domain $1$
and codomain $0$, which does not exist.)}

\begin{prop}
  The map $\widetilde{\lim}$ is a categorical equivalence.
\end{prop}
\begin{proof}
  We shall verify that the hypothesis of \cite[Prop.\ 7.6.15]{Cis19} are
  satisfied.

  First of all, the functor $\lim$ preserves finite limits as they commute with
  finite limits, therefore it is left exact. Also, a morphism in
  $(\sSet_0/\cC)^{\op}$ is a weak equivalence if and only if it is mapped to an
  isomorphism, thus condition (i) is trivially satisfied.

  We are only missing condition (ii), so let's consider a morphism
  $\phi\colon c\rightarrow\lim_K D$ in $\cC$. This corresponds to an object
  $(c,\hat{\phi})$ in $\cC/D$, that is a map
  $\hat{\phi}\colon\Delta^0*K\rightarrow\cC$ such that $\hat{\phi}|_K=D$,
  $\hat{\phi}|_{\Delta^0}=c$. We have then a map $D\rightarrow\hat{\phi}$ in
  $\sSet_0/\cC$ given
  by the inclusion $K\rightarrow\Delta^0*K$ and inducing the morphism
  $\phi\colon c=\lim_{\Delta^0*K}\hat{\phi}\rightarrow\lim_KD$, which allows us
  to construct the desired commutative square.
\end{proof}

We may also specify a weak equivalence $\cC\rightarrow L(\sSet_0/\cC)$ as in
\cite[Def.\ 4.6]{Szu17b} \wfd{(I don't really see how to adapt this)}.
We extend $\cC\rightarrow L(\cP)$ first to a functor $\sSet_0/\cC\rightarrow
L(\cP)$, $D\colon K\rightarrow\cC\mapsto\lim_KD$ \wfd{(doubtful about doing
this)} and then localize to get $L(\sSet_0/\cC)\rightarrow L(\cP)$, granting us a
diagram
\[\begin{tikzcd}
	\cC & {L(P)} \\
	\cC & {L(\sSet_0/\cC)}
	\arrow[from=2-2, to=1-2]
	\arrow[from=1-1, to=1-2]
	\arrow[Rightarrow, no head, from=2-1, to=1-1]
	\arrow[from=2-1, to=2-2]
\end{tikzcd}.\]

\section{Plan B: LCCC}

We can also rewrite the paper by Kapulkin about LCCC arising from TT using the language of localizations of quasi-categories. There they develop the relevant theory showing that under some conditions the frame associated to a fibration category is locally cartesian closed, but using Cisinski's results we can prove the same theorem directly using a more mainstream theory.

What should be included in such an overview?

1- Cisinski's theory of localizations (of fibration $\infty$-categories)

2- an introduction to contextual categories: where do they come from? Why are they useful? Check out Voevodsky's papers about C-systems

We explain what dependent type theory is (Martin-Lof's notes from 1984) and why
it's an interesting foundation of mathematics. We mention Homotopy Type Theory
as an effort to provide homotopical foundations which better model how we think
about identities, which explains why intensional identity types are more
interesting to us than extensional ones.

We move on to defining contextual categories (1211.2851, 1406.7413, 1507.02648)
and what
the Pi, Sigma and Id structures are (1406.7413, 1211.2851 Appendix B). To
understand what the link between such structures and syntactically presented
type theories we refer to 1507.02648, Sec.\ 1.1, while the statement of the
conjectured correspondence is in 1304.0680, Sec.\ 2.1.

Where does the link between dependent type theories and $\infty$-categories come
from? We see that $\infty$-categories intuitively model the behavior of type
theories and their type constructions, especially when considering Homotopy Type
Theory, however this relation is known only
partially (references in the intro of 1507.02648). The idea is that the type
theory we are interested in should be the internal language of some class of
$\infty$-categories and a precise statement would require us to provide
homotopical functors in both directions which induce an equivalence on the
associated $\infty$-categories. The idea is to construct the
functor from contextual categories as a localization functor, that is we need to
provide a homotopical structure on contextual categories, as they do in
1507.02648 (there should be an older reference) which then provides an
associated $\infty$-category. This is the object of the Initiality Conjecture,
stated in 1610.00037, in the hope that such a correspondence will extend to
Homotopy Type Theory and some notion of Elementary Higher Toposes, perhaps the
one specified in 1805.03805. At the moment we know that HoTT can be interpreted
in Higher Toposes with some structure. Current progress: 1709.09519, an upcoming
paper by Nguyen-Uemura (HoTTest talk).

Our aim is to show that when taking contextual categories with the structure we
specified earlier we obtain a locally cartesian closed $\infty$-category. To do
so we provide a fibrational structure on contextual categories (1304.0680,
1507.02648), which as we anticipate will imply that their simplicial
localizations are finitely complete. We also prove that the hypothesis of
\cite[Thm.\ 7.6.16]{Cis19} are satisfied, informing that this will be sufficient
to prove Kapulkin's main result from 1507.02648.

We then develop the theory of localizations of $\infty$-categories by Cisinski
and specifically develop the results concerning $\infty$-categories with
fibrations and weak equivalences. Localizations of such $\infty$-categories are
finitely complete. The objective is to show \cite[Thm.\ 7.6.16]{Cis19}. How in
depth should we go?

Why all of this is interesting: we are proving Kapulkin's result internalizing
all of the discussion within the language of $\infty$-category theory and
relying only on its simplicial model.

\section{Localizations of \texorpdfstring{$\infty$}{∞}-Categories}

To prove the that localizing a categorical model of type theory we get a locally
cartesian closed $\infty$-category we need a theory of localizations. We shall
provide one in the general context of $\infty$-categories as developed by
Cisinski in \emph{Higher Categories and Homotopical Algebra} with the aim of
proving \cite[Thm.\ 7.6.16]{Cis19}, which will do the heavy lifting in showing
the desired result. Those familiar with the theory may skip the entire chapter
while keeping in mind yadda yadda \wfd{(LIST THE MAJOR RESULTS)}.

\begin{defn}
  Let $X$ be a simplicial set and $W\subset X$ a simplicial subset. Given an
  $\infty$-category $\cC$, we define $\Home_W(X,\cC)$ to be the full simplicial
  subset of $\Home(X,\cC)$ whose objects are the morphisms $f\colon
  X\rightarrow\cC$ sending the 1-simplices in $W$ to isomorphisms.
\end{defn}

\begin{rmk}
  The above definition induces a canonical pullback square
  \[\begin{tikzcd}
    {\Home_W(X,\cC)} & {\Home(X,\cC)} \\
    {\Home_W(W,\cC)} & {\Home(W,\cC)}
    \arrow[from=1-1, to=2-1]
    \arrow[from=2-1, to=2-2]
    \arrow[from=1-2, to=2-2]
    \arrow[from=1-1, to=1-2]
    \arrow["\lrcorner"{anchor=center, pos=0.125}, draw=none, from=1-1, to=2-2]
  \end{tikzcd}\]
  given by the inclusion $W\rightarrow X$.
\end{rmk}

\begin{defn}
  Given an $\infty$-category $\cC$ and $W\subset\cC$, a \emph{localization of
  $\cC$ by $W$} is a functor $\gamma\colon\cC\rightarrow L(\cC)$ such that:
  \begin{enumerate}
    \item $L(\cC)$ is an $\infty$-category;
    \item $\gamma$ sends the 1-simplices of $W$ to isomorphisms in $L(\cC)$;
    \item for any $\infty$-category $\cD$ there is an equivalence of
      $\infty$-categories
      \[\Home(L(\cC),\cD)\rightarrow\Home_W(\cC,\cD)\]
      given by precomposing with $\gamma$.
  \end{enumerate}
\end{defn}

\wfd{(CISINSKI DOES NOT ASK FOR $\cC$ TO BE AN $\infty$-CATEGORY. SHOULD WE
BE LESS GENERAL AS WE HAVE DONE?)}

\begin{prop}\label{exuniq}
  Given an $\infty$-category $\cC$ and a subsimplicial set $W$, the localization
  of $\cC$ by $W$ always exists and it is essentially unique.
\end{prop}
\begin{proof}
  We begin by proving that a localization exists in the case where $W=\cC$.

  In this context, $\Home_W(\cC,\cD)\cong\Home(\cC,\cD^{\cong})$ canonically,
  where $\cD^{\cong}$ is the maximal sub-$\infty$-subgroupoid of $\cD$.
  Factoring $\cC\rightarrow\Delta^0$ in the Kan model structure, we find an
  anodyne map $\gamma\colon\cC\rightarrow L(\cC)$.

  Remember that for any anodyne map $A\rightarrow B$ we get a trivial fibration
  $\Home(B,\cD^{\cong})\rightarrow\Home(A,\cD^{\cong})$. Looking then at the
  commutative diagram
  \[\begin{tikzcd}
    {\Hom(L(\cC),\cD^{\cong})} & {\Hom_W(\cC,\cD^{\cong})} \\
    {\Home(L(\cC),\cD)} & {\Home_W(\cC,\cD)}
    \arrow["\gamma^*", "\sim"', two heads, from=1-1, to=1-2]
    \arrow[from=1-1, to=2-1]
    \arrow["{\gamma^*}"', from=2-1, to=2-2]
    \arrow["\cong", from=1-2, to=2-2]
    \arrow["\cong"', from=1-1, to=2-1]
  \end{tikzcd},\]
  by the 2-out-of-3 property we see that the lower $\gamma^*$ is an equivalence.
  
  We now move on to the general case. First of all, notice that as a particular
  case of the previous one we get that localizing $\Delta^1$ at its non-trivial
  morphism we obtain $\Delta^1\rightarrow J=L(\Delta^1)\sim\Delta^0$. Taking
  then $W\subset\cC$, we consider the commutative diagram
  \[\begin{tikzcd}
    {\coprod_{f\in W_1}\Delta^1} & \cC \\
    {\coprod_{f\in W_1}J} & {\cC'} \\
    && {L(\cC)}
    \arrow[from=1-1, to=1-2]
    \arrow[from=1-1, to=2-1]
    \arrow[from=2-1, to=2-2]
    \arrow[from=1-2, to=2-2]
    \arrow[from=1-1, to=2-1]
    \arrow["\sim"', hook, from=2-2, to=3-3]
    \arrow["\gamma", curve={height=-12pt}, from=1-2, to=3-3]
    \arrow["\lrcorner"{anchor=center, pos=0.125, rotate=180}, draw=none, from=2-2, to=1-1]
  \end{tikzcd},\]
  where $\cC'\rightarrow L(\cC)$ is an inner anodyne map obtained by taking the
  fibrant replacement of $\cC'$ in the Joyal model structure. This can be done
  functorially via the small object argument.

  For any $\infty$-category $\cD$, we get a trivial fibration
  $\Home(L(\cC),\cD)\rightarrow\Home(\cC',\cD)$ and a pullback square
  \[\begin{tikzcd}
    {\Home(\cC',\cD)} & {\Home(\cC,\cD)} \\
    {\Pi_{f\in W_1}\Home(J,\cD)} & {\Pi_{f\in W_1}\Home(\Delta^1,\cD)}
    \arrow[from=1-2, to=2-2]
    \arrow[from=2-1, to=2-2]
    \arrow[from=1-1, to=2-1]
    \arrow[from=1-1, to=1-2]
  \end{tikzcd},\]
  which together with the pullback
  \[\begin{tikzcd}
    {\Home_W(\cC,\cD)} & {\Home(\cC,\cD)} \\
    {\Pi_{f\in W_1}\Home(\Delta^1,\cD^{\cong})} & {\Pi_{f\in W_1}\Home(\Delta^1,\cD)}
    \arrow[from=1-2, to=2-2]
    \arrow[from=2-1, to=2-2]
    \arrow[from=1-1, to=2-1]
    \arrow[from=1-1, to=1-2]
  \end{tikzcd}\]
  implies by pasting that
  \[\begin{tikzcd}
    {\Home(\cC',\cD)} & {\Home_W(\cC,\cD)} \\
    {\Pi_{f\in W_1}\Home(J,\cD)} & {\Pi_{f\in W_1}\Home(\Delta^1,\cD^{\cong})}
    \arrow[from=1-2, to=2-2]
    \arrow[from=1-1, to=1-2]
    \arrow[from=1-1, to=2-1]
    \arrow["\sim", two heads, from=2-1, to=2-2]
    \arrow[from=1-2, to=2-2]
  \end{tikzcd}\]
  is also a pullback and therefore the upper arrow is a trivial fibration.
  Composing it with the other one we get
  $\gamma^*\colon\Home(L(\cC),\cD)\rightarrow\Home_W(\cC,\cD)$, which is then a
  trivial fibration and therefore an equivalence of $\infty$-categories.

  7.1.4 Observe that, through this construction, one can always construct
  $L(\cC)$ so that $\gamma$ is a bijection on objects because $\cC'\rightarrow
  L(\cC)$ is an inner anodyne extension and therefore a retract of a countable
  composition of sums of pushouts of maps which are the identity on objects,
  that is the inner horn inclusions.

  We now move on to proving that the localization is essentially unique. For
  this, we notice that $\gamma$ establishes then an isomorphism between
  $\pi_0(k(\Home_W(\cC,-)))$ and $\pi_0(\Home(L(\cC),-))=ho(\sSet)(L(\cC),-)$
  with respect to the Joyal model structure, thus by Yoneda $(L(\cC),\gamma)$ is
  unique up to unique isomorphism in $ho(\sSet)$ and up to a contractible space
  of equivalences in $\sSet$.
\end{proof}

\begin{rmk}
  7.1.5

  In this context, we may define $\overline{W}$, the saturation of $W$ in $\cC$,
  as the cartesian square

  such that $\overline{W}$ is precisely the maximal simplicial subset of $\cC$
  whose morphisms are the ones which become invertible in $L(\cC)$.

  We have then inclusions $Sk_1(W)\subset W\subset\overline{W}$ and, for any
  $\infty$-category $\cD$, this induces equalities
  \[\Home_{Sk_1(W)}(\cC,\cD)=\Home_{W}(\cC,\cD)\Home_{\overline{W}}(\cC,\cD),\]
  implying that $(L(\cC),\gamma)$ is also the localization of $\cC$ by $Sk_1(W)$
  and the one by $\overline{W}$, however the inclusion $W\rightarrow\cC$ is a
  fibration in the Joyal model category as it is the pullback of one, implying
  that $\overline{W}$ is itself an $\infty$-category.

  We shall say that $W$ is saturated if $W=\overline{W}$.
\end{rmk}

\begin{rmk}
  7.1.6

  The functor $ho(\cC)\rightarrow ho(L(\cC))$ exibits $ho(L(\cC))$ as the
  1-categorical localization of $\cC$ at $\Arr(\tau(W))$, as can be seen by
  using the universal property.

  On the other hand, given a 1-category $\cC$ and localizing at a set of
  morphisms $W$, not necessarily the induced map $L(N(\cC))\rightarrow
  N(L(\cC))$ is an isomorphism. Indeed, $L(N(\cC))$ can have much better
  properties, as can be seen for example from yadda yadda \wfd{(PROPOSITION
  ABOUT FINITE COMPLETENESS)} and in fact localizing 1-categories after taking
  their nerves gives every $\infty$-category \wfd{(MORE
  PRECISE STATEMENT)}.
\end{rmk}

\begin{prop}
  7.1.9

  Given a universe $\bfU$ and $W$ a simplicial subset of a $\bfU$-small
  $\infty$-category $\cC$, let $\gamma\colon\cC\rightarrow L(\cC)$ be the
  associated localization. Then the functor
  $\gamma^*\colon\Home(L(\cC)^{\op},\cS)\rightarrow\Home(\cC^{\op},\cS)$ is
  fully faithful and its essential image consists of all presheaves
  $F\colon\cC^{\op}\rightarrow\cS$ sending maps $u\colon x\rightarrow y$ in $W$
  to invertible maps $u^*\colon Fy\rightarrow Fx$ in $\cS$.
\end{prop}
\begin{proof}
  The map $\gamma$ gives us a morphism
  \[\gamma^*\colon\Home(L(\cC)^{\op},\cS)\simeq\Home_{W^{\op}}(\cC^{\op})\rightarrow\Home(\cC^{\op},\cS),\]
  which has a left adjoint $\gamma_!$ and a right adjoint $\gamma_*$. Now, for
  any presheaf $F\colon L(\cC)^{\op}\rightarrow\cS$, the unit map
  $F\rightarrow\gamma_*\gamma^*F$ is invertible and, by adjunction, the same
  goes for the counit map $\gamma_!\gamma^*F\rightarrow F$, which means that
  $u^*$ is fully faithful. On the other hand, given a presheaf
  $F\colon\cC^{\op}\rightarrow\cS$ sending 1-simplices in $W$ to invertible maps,
  the counit $\gamma^*\gamma_*F\rightarrow F$ and the unit
  $F\rightarrow\gamma^*\gamma_!F$ are both invertible since the restrictions of
  these adjunctions to $\Home_{W^{\op}}(\cC^{\op},\cS)$ form adjoint
  equivalences of $\infty$-categories as $\gamma^*$ is an equivalence.
\end{proof}

\begin{prop}\label{final}
  7.1.10

  Given an $\infty$-category $\cC$ and a simplicial subset $W$, the localization
  functor $\gamma\colon\cC\rightarrow L(\cC)$ is final and cofinal. In
  particular, if $e\colon\Delta^0\rightarrow\cC$ encodes a final or a cofinal
  object, so does $\gamma(e)$.
\end{prop}
\begin{proof}
  1702.02681, prop.\ 5.13.
\end{proof}

\begin{prop}
  7.1.11
  Let's fix a universe $\bfU$, a $\bfU$-small $\infty$-category $\cC$ and a
  simplicial subset $W$. Consider then a functor $f\colon\cC\rightarrow\cD$,
  where $X$ is a small $\infty$-category. Then $f$ exhibits $\cD$ as the
  localization of $\cC$ by $W$ if and only if the following conditions hold:
  \begin{enumerate}
    \item the functor $f$ sends the 1-simplices of $W$ to invertible maps of
      $\cD$;
    \item the functor $f$ is essentially surjective;
    \item the functor $f^*$ induces an equivalence of $\infty$-categories
      \[f^*\colon\Home(\cD^{\op},\cS)\rightarrow\Home_{W^{\op}}(\cC^{\op},\cS).\]
  \end{enumerate}
\end{prop}
\begin{proof}
  One implication is trivial (for (2) look at the construction in \ref{exuniq}).
  For the converse, let's pick a localization $\gamma\colon\cC\rightarrow
  L(\cC)$ and, through condition (1), we get a factorization $g\colon
  L(\cC)\rightarrow\cD$ such that $g\cdot\gamma\cong f$, giving us a triangle
  \[\begin{tikzcd}
    {\Home(\cD^{\op},\cE)} && {\Home(L(\cC)^{\op},\cE)} \\
                           & {\Home_{W^{\op}}(\cC^{\op},\cE)}
    \arrow["{g^*}", from=1-1, to=1-3]
    \arrow["{f^*}"', from=1-1, to=2-2]
    \arrow["{\gamma^*}", from=1-3, to=2-2]
  \end{tikzcd}\]
  commuting up to $J$-homotopy for any $\infty$-category $\cE$. Picking
  $\cE=\cS$, $\gamma^*$ and $f^*$ are equivalences
  of $\infty$-categories, the latter by (3). It follows by 2-out-of-3 that $g^*$
  is one too, and therefore the same applies to its left adjoint $g_!$, which is
  then fully faithful. This is equivalent to $g$ being fully faithful \wfd{(FUN
  THEOREM, MAYBE STATE IT AT LEAST 6.1.5)} and, since
  $f$ is essentially surjective by (2), the same goes for $g$. It follows that
  $g$ is an equivalence of $\infty$-categories.
\end{proof}

\begin{prop}
  7.1.14

  Let $f\colon\cC\rightarrow\cD$ be a functors between $\infty$-categories with
  a right adjoint $g\colon\cD\rightarrow\cC$ and suppose that we are given
  simplicial subsets $V\subset\cC$, $W\subset\cD$ such that $f(V)\subset W$,
  $g(W)\subset V$. Then we can lift them to an adjunction
  $\adjunction{\overline{f}}{L(\cC)}{L(\cD)}{\overline{g}}$ such that the
  diagrams
  \[\begin{tikzcd}
    \cC & \cD \\
    {L(\cC)} & {L(\cD)}
    \arrow["{\gamma_\cD}", from=1-2, to=2-2]
    \arrow["{\gamma_\cC}"', from=1-1, to=2-1]
    \arrow["{\overline{f}}"', from=2-1, to=2-2]
    \arrow["f", from=1-1, to=1-2]
  \end{tikzcd},
  \quad
  \begin{tikzcd}
    \cD & \cC \\
    {L(\cD)} & {L(\cC)}
    \arrow["{\gamma_\cC}", from=1-2, to=2-2]
    \arrow["{\gamma_\cD}"', from=1-1, to=2-1]
    \arrow["{\overline{g}}"', from=2-1, to=2-2]
    \arrow["g", from=1-1, to=1-2]
  \end{tikzcd}
  \]
\end{prop}
\begin{proof}
  Let's write $\Home_V^W(\cC,\cD)$ for the full subcategory of $\Home(\cC,\cD)$
  whose objects are functors $\phi$ such that $\phi(V)\subset W$. The
  equivalence
  $\gamma^*_\cC\colon\Hom(L(\cC),L(\cD))\rightarrow\Home_V(\cC,L(\cD))$ allows
  us to construct a functor
  $\Home^W_V(\cC,\cD)\rightarrow\Home_V(\cC,L(\cD))\rightarrow\Home(L(\cC,\cD))$
  which associates to any $\phi$ as above a functor $\overline{\phi}$ making the
  square
  \[\begin{tikzcd}
    \cD & \cC \\
    {L(\cD)} & {L(\cC)}
    \arrow["{\gamma_\cC}", from=1-2, to=2-2]
    \arrow["{\gamma_\cD}"', from=1-1, to=2-1]
    \arrow["{\overline{\phi}}"', from=2-1, to=2-2]
    \arrow["\phi", from=1-1, to=1-2]
  \end{tikzcd}\]
  commute up to $J$-homotopy.

  The proof works by observing that our map also lifts natural transformations
  functorially, which allows us to show the triangle identities for the lifted
  unit and counit.
\end{proof}

\begin{prop}
  7.1.18

  Let $u\colon\cC\rightarrow\cD$ be a functor between $\infty$-categories with a
  fully faithful right adjoint $v$ and consider $W=k(\cD)\times_\cD\cC$, the
  subcategory of maps of $\cC$ which become invertible in $\cD$. Then $u$
  exhibits $\cD$ as the localization of $\cC$ by $W$>
\end{prop}
\begin{proof}
  Given a localization $\gamma\colon\cC\rightarrow L(\cC)$ by $W$, we get a
  functor $\gamma\cdot v\colon\cD\rightarrow L(\cC)$ which, paired with the
  $\overline{u}$ obtained from the construction in the previous proof, lifts the
  adjunction $u\dashv v$ to the localizations (where $L(\cD)\cong\cD$ as we
  localize at the identities). Lifting maintains the counit invertible, which
  allows us to conclude that $\gamma\cdot v$ is fully faithful.

  Essential surjectivity
  follows from the fact that, for any object $c$ in $\cC$, the unit $\eta_c$ is
  such that $\epsilon_{u(c)}\cdot u(\eta_c)=\id_{u(c)}$ and, since $\epsilon$ is
  invertible, so is $u(\eta_c)$, thus $\eta_c$ becomes invertible in $L(\cC)$
  and shows that $(\gamma_{\cC}\cdot v)(u(c))=\gamma_{\cC}(vu(c))\cong c$.
  Notice that here we used that $L(\cC)_0=\cC_0$, which is permissible up to
  equivalence as previously noted.

  \wfd{(PLEASE CHECK PROOF)}
\end{proof}

\begin{cor}
  7.2.5
\end{cor}
\begin{proof}
  7.1.5
\end{proof}

\begin{cor}
  7.2.8
\end{cor}
\begin{proof}
  7.1.5, 7.1.18, 7.2.5
\end{proof}

\begin{cor}
  7.2.10
\end{cor}

\begin{cor}
  7.2.15
\end{cor}

\begin{cor}
  7.2.16
\end{cor}

\begin{cor}
  7.2.18
\end{cor}
\begin{proof}
  7.2.16
\end{proof}

\begin{cor}
  7.2.25
\end{cor}
\begin{proof}
  7.2.8, 7.2.10, 7.2.15
\end{proof}

\begin{cor}
  7.3.2
\end{cor}

\begin{cor}
  7.3.4
\end{cor}

\begin{cor}
  7.3.5
\end{cor}
\begin{proof}
  7.1.10, 7.3.2, 7.3.4
\end{proof}

\begin{cor}
  7.3.8
\end{cor}
\begin{proof}
  7.1.10, 7.1.12
\end{proof}

\begin{cor}
  7.3.9
\end{cor}
\begin{proof}
  7.3.8
\end{proof}

\begin{cor}
  7.3.10
\end{cor}
\begin{proof}
  7.3.9
\end{proof}

\begin{prop}
  7.3.11
\end{prop}

\begin{cor}
  7.3.15
\end{cor}
\begin{proof}
  7.1.12, 7.3.9, 7.3.10
\end{proof}

\begin{cor}
  7.3.16
\end{cor}
\begin{proof}
  7.3.5, 7.3.15
\end{proof}

\begin{cor}
  7.3.26
\end{cor}

\begin{cor}
  7.3.27
\end{cor}
\begin{proof}
  7.3.16, 7.3.26
\end{proof}

\begin{cor}
  7.3.28
\end{cor}

\begin{cor}
  7.3.29
\end{cor}
\begin{proof}
  7.3.27, 7.3.28
\end{proof}

\begin{prop}
  7.4.2
\end{prop}

\begin{prop}
  7.4.11
\end{prop}
\begin{proof}
  7.3.11, 7.4.2
\end{proof}

\begin{defn}
  An \emph{$\infty$-category with weak equivalences and fibrations} is a triple
  $(\cC,W,Fib)$ where $\cC$ is an $\infty$-category with a final object,
  $W\subset\cC$ is a subcategory with the 2-out-of-3 property whose maps are
  called weak equivalences and $Fib\subset\cC$ is a class of fibrations such
  that:
  \begin{enumerate}
    \item for any pullback square of $\cC$
      \[\begin{tikzcd}
        {x'} & x \\
        {y'} & y
        \arrow["p", from=1-2, to=2-2]
        \arrow["{p'}"', from=1-1, to=2-1]
        \arrow["v"', from=2-1, to=2-2]
        \arrow["u", from=1-1, to=1-2]
      \end{tikzcd}\]
      in which $y$ is fibrant and $p$ lies in $Fib$ (and $W$), the same goes
      for $p'$;
    \item for any map $f\colon x\rightarrow y$ with fibrant codomain can be
      factored as a map in $W$ followed by one in $Fib$.
  \end{enumerate}
  By \emph{fibrant object} we mean an object whose map to the terminal one is in
  $Fib$.

  We shall call \emph{weak equivalences} the maps in $W$ and \emph{fibrations}
  the ones in $Fib$. Maps which are both shall be referred to as \emph{trivial
  fibrations}.
\end{defn}

\begin{prop}
  7.4.13
\end{prop}

\begin{cor}
  7.4.14
\end{cor}
\begin{proof}
  7.4.13
\end{proof}

\begin{prop}
  7.4.16
\end{prop}
\begin{proof}
  7.4.13
\end{proof}

\begin{prop}
  7.4.19
\end{prop}
\begin{proof}
  7.4.11
\end{proof}

\begin{prop}
  7.5.5
\end{prop}

\begin{prop}\label{756}
  7.5.6

  Given an $\infty$-category with weak equivalences and fibrations $\cC$, the
  localization $L(\cC_f)$ has finite limites and the localization functor
  $\cC_f\rightarrow L(\cC_f)$ is left exact. Moreover, for any $\infty$-category
  $\cD$ with finite limits and any left exact functor
  $f\colon\cC_f\rightarrow\cD$, the induced functor $\overline{F}\colon
  L(\cC_f)\rightarrow\cD$ is left exact.
\end{prop}
\begin{proof}
  Maybe do not prove it? It relies on a bunch of results from ch. 7.2, 7.3, 7.4
  which we do not really want to prove.

  7.1.10, 7.2.18, 7.2.25, 7.3.27, 7.4.13, 7.4.16
\end{proof}

\begin{defn}
  An $\infty$-category of fibrant objects is an $\infty$-category with weak
  equivalences and fibrations $\cC$ in which all objects are fibrant.
\end{defn}

\begin{exmp}
  For any $\infty$-category with weak equivalences and fibrations $\cC$, its
  full subcategory given by fibrant objects is an $\infty$-category of fibrant
  objects. We shall denote it by $\cC_f$.
\end{exmp}

\begin{prop}\label{7516}
  7.5.16

  Let $x$ be a fibrant object in an $\infty$-category with weak equivalences and
  fibrations $\cC$. The induced functor
  $\cC_f/\gamma_f(x)\rightarrow\cC/\gamma(x)$ is final.
\end{prop}
\begin{proof}
  We have that $\cC_f/\gamma_f(x)=L(\cC_f)/\gamma_f(x)\times_{L(\cC_f)}\cC_f$
  and $\cC/\gamma(x)=L(\cC)/\gamma(x)\times_{L(\cC)}\cC$ and the functor we are
  considering is induced by $\overline{\iota}\colon L(\cC_f)\rightarrow L(\cC)$.
  
  To prove that it is final, it is sufficient to show that for any object
  $(c,u)$ of $L(\cC)/\gamma(x)$ the coslice $(c,u)\\(\cC_f/\gamma_f(x))$ is
  weakly contractible and, to do this, by \cite[Lem.\ 4.3.15]{Cis19} we can show
  that any functor $F\colon E\rightarrow (c,u)\\(\cC_f/\gamma_f(x))$, where $E$
  is the nerve of a finite partially ordered set, is $\Delta^1$-homotopic to a
  constant functor. This can be done through the theory of Reedy fibrant
  diagrams developed in \cite[Ch.\ 7.4]{Cis19}.
\end{proof}

\begin{prop}\label{7517}
  7.5.17

  Let $\bfU$ be a universe and $\cC$ a $\bfU$-small $\infty$-category with weak
  equivalences and fibrations. For any $\infty$-category $\cD$ with $\bfU$-small
  colimits and any functor $F\colon\cC\rightarrow\cD$, we have an isomorphism
  \[(\gamma_f)_!\iota^*(F)\cong\overline{\iota}^*\gamma_!(F)\]
  induced by the square
  \[\begin{tikzcd}
    {\cC_f} & \cC \\
    {L(\cC_f)} & {L(\cC)}
    \arrow["\gamma", from=1-2, to=2-2]
    \arrow["{\gamma_f}"', from=1-1, to=2-1]
    \arrow["\iota", from=1-1, to=1-2]
    \arrow["{\overline{\iota}}"', from=2-1, to=2-2]
  \end{tikzcd},\]
  which commutes up to $J$-homotopy.
\end{prop}
\begin{proof}
  We only need to prove that the evaluation of the canonical map
  $(\gamma_f)_!\iota^*(F)\cong\overline{\iota}^*\gamma_!(F)$ at any object $x$
  of $\cC_f$ is invertible. This evaluation is equivalent by \cite[Prop.\
  6.4.9]{Cis19} to the map
  \[\colim_{\cC_f/\gamma_f(x)}i^*(F)/\gamma_f(x)\rightarrow\colim_{\cC/\gamma(x)}F/\gamma(x),\]
  where $F/\gamma(x)$ is define by composing $F$ with the canonical projection
  $\cC/\gamma(x)\rightarrow\cC$ and similarly for $i^*(F)/\gamma_f(x)$. Using
  \ref{7516} and the commutativity of the square above, we get that the desired
  map is indeed invertible for all $x$.
\end{proof}

\begin{prop}
  7.5.18
\end{prop}
\begin{proof}
  7.5.6, 7.5.17

  We already know that $\overline{\iota}$ is essentially surjective as every
  object in $\cC$ is weakly equivalent to one in $\cC_f$ and the localization
  functors are essentially surjective themselves, thus it is enough to prove
  that it is fully faithful. To do this, we may fix a universe $\bfU$ such that
  $\cC$ is $\bfU$-small and prove that the functor
  \[\overline{\iota}_!\colon\Home(L(\cC_f),\cS)\rightarrow\Home(L(\cC),\cS)\]
  is fully faithful by \cite[Prop.\ 6.1.15]{Cis19}, which is equivalent to
  proving that the unit map $1\rightarrow\overline{\iota}^*\overline{\iota}_!$
  of the adjunction $\overline{\iota}^*\dashv\overline{\iota}_!$ is invertible.

  We know that $\overline{\iota}_*$ and $\overline{\iota}^*$ both have right
  adjoints, thus they preserve colimits. Also, every $\cS$-valued functor
  indexed by a $\bfU$-small $\infty$-category can be obtained as a colimit of
  representable ones, hence it is enough to check that the condition holds for
  any representable functor $F$. Also, $\gamma_f$ is essentially surjective,
  which means that it is sufficient to check that map
  $(\gamma_f)_!\rightarrow\overline{\iota}^*\overline{\iota}_!(\gamma_f)_!$
  which we get by precomposing the unit with $(\gamma_f)_!$ is invertible.

  We have then the chain of isomorphisms
  \begin{align*}
    (\gamma_f)_! &\cong(\gamma_f)_!\overline{\iota}^*\overline{\iota}_! \\
                 &\cong\overline{\iota}^*\gamma_f\iota_! \\
                 \overline{\iota}^*\overline{\iota}_!(\gamma_f)_!,
  \end{align*}
  where the first isomorphism comes from the full faithfulness of $\iota$, the
  second one from \ref{7517} and the last one the fact that
  $\overline{\iota}\cdot\gamma_f\cong\gamma\cdot\iota$, as noted in \ref{7517}.

  The second claim follows directly from the first one and \ref{756}.
\end{proof}

\begin{cor}
  7.5.19

  Let $\cC$ be an $\infty$-category with weak equivalences and fibrations. For a
  morphism between fibrant objects $p\colon x\rightarrow y$, the following
  conditions are equivalent:
  \begin{enumerate}
    \item the morphism $p$ has a section in $ho(L(\cC))$;
    \item there exists a morphism $p'\colon x'\rightarrow x$ s.t.\ the
      composition of $p'$ and $p$ is a weak equivalence;
    \item there exists a fibration $p'\colon x'\rightarrow x$ s.t.\ the
      composition of $p'$ and $p$ is a weak equivalence.
  \end{enumerate}
\end{cor}
\begin{proof}
  7.5.18

  Should we prove it? Uses right calculus of fractions, but it's rather simple.
\end{proof}

\begin{construction}\label{7522}
  7.5.22
\end{construction}
\begin{proof}
  7.5.19, 7.5.20
\end{proof}

\begin{rmk}
  Let $\cC$ be an $\infty$-category with weak equivalences $W$ and
  $F\colon\cC\rightarrow\cD$ be a functor. The precomposition functor
  $\gamma^*\colon\Home(L(\cC),\cD)\rightarrow\Home(\cC,\cD)$ does not have a
  left adjoint in general, but we may ask whether $\Hom(F,\gamma^*(-))$ is
  representable in $\Home(L(\cC),\cD)$. If it is, a representative is denoted by
  $\bfR F\colon L(\cC)\rightarrow\cD$ and is called the \emph{right derived
  functor of $F$}. Beware that to be precise one would have to specify the
  natural transformation $F\rightarrow\bfR F\cdot\gamma$ exhibiting it as such.
  Dually, a representative of $\Hom(\gamma^*(-),F)$ is the \emph{left derived
  functor of $F$}.
\end{rmk}

\begin{prop}\label{7524}
  7.5.24

  If $F\colon\cC\rightarrow\cD$ sends weak equivalences to isomorphisms, then
  the functor $\overline{F}\colon L(\cC)\rightarrow\cD$, associated to $F$ by
  the universal property of $L(\cC)$, is the right derived functor of $F$.
\end{prop}
\begin{proof}
  Let's fix a universe $\bfU$ such that $\cC$ and $\cD$ are $\bfU$-small and let
  $G\colon L(\cC)\rightarrow\cD$ be any functor. Then the invertible map
  $\overline{F}\cdot\gamma\cong F$ and the equivalence of $\infty$-categories
  $\Home(L(\cC),\cD)\simeq\Home_W(\cC,\cD)$ induce invertible maps
  $\Hom(\overline{F},G)\simeq\Hom(\overline{F}\cdot\gamma,G\cdot\gamma)\simeq\Hom(F,G\cdot\gamma)$
  in $\cS$, functorially in $G$.
\end{proof}

\begin{construction}\label{7525}
  \wfd{(NOT COMPLETE)}

  Let $\cC$ be an $\infty$-category with weak equivalences and fibrations. Given
  any functor $F\colon\cC\rightarrow\cD$ sending weak equivalences between
  fibrant objects to invertible maps then has a right derived functor $\bfR F$,
  which may be constructed as follows.

  First we choose a quasi-inverse $R\colon L(\cC)\rightarrow L(\cC_f)$ of the
  equivalence of $\infty$-categories specified in \ref{7518}, then we choose a
  functor $\overline{F}\colon L(\cC_f)\rightarrow\cD$ and a natural isomorphism
  $j\colon\overline{F}\cdot\gamma_f\rightarrow F\cdot\iota$. We set then $\bfR
  F=\overline{F}\cdot R$.
\end{construction}

There are some interesting remarks which may be included!!!!

\begin{prop}
  7.5.28

  For any left exact functor $F\colon\cC\rightarrow\cD$ between
  $\infty$-categories with weak equivalences and fibrations, the right derived
  functor $\bfR F\colon L(\cC)\rightarrow L(\cD)$ is left exact.
\end{prop}
\begin{proof}
  7.5.6

  We have a square
  \[\begin{tikzcd}
    {L(\cC_f)} & {L(\cD_f)} \\
    {L(\cC)} & {L(\cD)}
    \arrow["{\overline{F}}", from=1-1, to=1-2]
    \arrow[from=1-2, to=2-2]
    \arrow[from=1-1, to=2-1]
    \arrow["{\bfR F}"', from=2-1, to=2-2]
  \end{tikzcd}\]
  commuting up to $J$-homotopy, where the vertical maps are equivalences of
  $\infty$-categories and $\overline{F}$ is the functor obtained by restricting
  $F$ to the subcategories of fibrant objects $\cC_f$ and $\cD_f$. It therefore
  suffices to show that $\overline{F}$ is left exact, but this follows from
  \ref{756}.
\end{proof}

\begin{lem}
  7.6.2
\end{lem}
\begin{proof}
  7.2.10.4, 7.2.18, 7.4.13
\end{proof}

\begin{lem}
  7.6.4
\end{lem}
\begin{proof}
  7.2.10.4, 7.2.18, 7.4.13
\end{proof}

\begin{lem}
  7.6.5
\end{lem}
\begin{proof}
  4.2.9
\end{proof}

\begin{lem}
  7.6.7
\end{lem}
\begin{proof}
  4.3.15, 7.4.19, 7.5.5
\end{proof}

\begin{thm}
  7.6.10
\end{thm}
\begin{proof}
  7.3.29, 7.6.2, 7.6.5, 7.6.7
\end{proof}

\begin{cor}
  7.6.13
\end{cor}
\begin{proof}
  7.5.18, 7.5.22, 7.5.24, 7.5.28, 7.6.4, 7.6.10
\end{proof}

\begin{thm}
  7.6.16
\end{thm}
\begin{proof}
  6.1.6, 6.1.7, 6.1.8, 7.1.14, 7.4.14, 7.5.18, 7.6.13
\end{proof}

\section{Categorical Models of TT as Locally Cartesian Closed Fibration
Categories}

\begin{defn}
  A fibration category $\cP$ is \emph{locally cartesian closed} if, for any
  fibration $p\colon a\rightarrow b$, the pullback functor
  $p^*\colon\cP\downarrow b\rightarrow\cP\downarrow a$ admits a right adjoint
  $p_*$ which is an exact functor.
\end{defn}

How does Kapulkin prove that a categorical model of Type Theory is a locally
cartesian closed fibration category?

First of all, he refers to AKL15 to show that $\cP$ has a fibrational structure,
then he goes on to show the following results, whose proofs are extremely terse
and therefore should be expanded.

\wfd{(I DON'T UNDERSTAND WHAT $f^*b$ SHOULD BE IN 1211.2851, DEF. 1.2.4. WHAT
FOLLOWS IS MY IDEA.)}

\begin{defn}
  Given $p_A\colon\Gamma.A\rightarrow\Gamma$, a section
  $a\colon\Gamma\rightarrow\Gamma.A$ and $f\colon\Delta\rightarrow\Gamma$, we
  look at the commutative diagram
  \[\begin{tikzcd}
    \Delta \\
    & {\Delta.f^*A} & {\Gamma.A} \\
    & \Delta & \Gamma \\
    \arrow["{p_A}", from=2-3, to=3-3]
    \arrow["f"', from=3-2, to=3-3]
    \arrow["{p_{f^*A}}", from=2-2, to=3-2]
    \arrow["{q(f,A)}"', from=2-2, to=2-3]
    \arrow["\lrcorner"{anchor=center, pos=0.125}, draw=none, from=2-2, to=3-3]
    \arrow[curve={height=12pt}, Rightarrow, no head, from=1-1, to=3-2]
    \arrow["{a\cdot f}", curve={height=-12pt}, from=1-1, to=2-3]
    \arrow["{f^*a}"{description}, dotted, from=1-1, to=2-2]
  \end{tikzcd},\]
  which gives us $f^*a$ as the factorization through the pullback square of the
  pair $(\id_\Delta,a\cdot f)$.
\end{defn}

\begin{lem}
  For any dependent projection $p_\Delta\colon\Gamma.\Delta\rightarrow\Gamma$ in
  a categorical model of type theory $\cC$, the pullback functor
  $p_\Delta^*\colon\cC\downarrow\Gamma\rightarrow\cC\downarrow\Gamma.\Delta$
  admits a right adjoint.
\end{lem}
\begin{proof}
  Kapulkin 1507.02648, Lemma 5.5.
\end{proof}

We know that every fibration in $\cC$ is isomorphic to a composite of dependent
projections , so this tells us that every fibration induces an adjunction
between fibrational slices \wfd{(YOU SHOULD DEFINE THEM)}.

\begin{lem}
  Consider an iterated context extension $\Gamma.\Delta.\Theta.\Psi$ in a
  categorical model of type theory $\cC$. Then the contexts
  \[\Gamma.\Pi(\Delta,\Theta).\Pi(p^*_{\Pi(\Delta,\Theta)}\Delta.app^*_{\Delta,\Theta}.\Psi)
    \text{ and }
  \Gamma.\Pi(\Delta,\Theta.\Psi)\]
  are isomorphic (actually equal) in $\cC$.
\end{lem}
\begin{proof}
  Kapulchin 1507.02648, Lemma 5.5. It simply refers to the construction of the
  $\Pi$-structure on $\cC^{cxt}$.
\end{proof}

We are finally ready to prove the result leading to the final one we want.

\begin{prop}\label{lccfc}
  A categorical model of type theory $\cC$ is a locally cartesian closed
  fibration category.
\end{prop}
\begin{proof}
  Kapulkin 1507.02648, Proposition 5.4.
\end{proof}

\begin{thm}
  Given a categorical model of type theory $\cC$, the $\infty$-category $L(\cC)$
  is locally cartesian closed.
\end{thm}
\begin{proof}
  Since a fibration category is more generally a $\infty$-category with
  fibrations and weak equivalences, we can apply \cite[Prop.\ 7.6.16]{Cis19} as
  the hypothesis are satisfied by \ref{lccfc}.
\end{proof}

\section{Pushforward}

One may ask whether cocartesian fibrations in $\sSet$ model Pi types, which is a
piece needed to understand a novel model of dependent type theory provided by
$\sSet$. To answer this question, an explicit description of the right adjoint
$p_*$ of the pullback functor $p^*\colon\sSet/Y\rightarrow\sSet/X$ induced by a
morphism $p\colon X\rightarrow Y$ is needed.

Consider an object $f\colon T\rightarrow X$ in $\sSet/X$. What is $p_*(f)\colon
T'\rightarrow Y$? We know that a $n$-simplex $t'$ of $T'$ corresponds
bijectively to a map $t'\colon\Delta^n\rightarrow T'$, which in turn corresponds
bijectively to a commutative diagram
\[\begin{tikzcd}
	{\Delta^n} & {} & {T'} \\
	& Y
	\arrow["{t'}", from=1-1, to=1-3]
	\arrow["y"', from=1-1, to=2-2]
	\arrow["{p_*(f)}", from=1-3, to=2-2]
\end{tikzcd}\]
and, under the adjunction $p^*\dashv p_*$, we get bijectively another
commutative diagram
\[\begin{tikzcd}
	U & {} & T \\
	& X
	\arrow["t", from=1-1, to=1-3]
	\arrow["{p^*(y)}"', from=1-1, to=2-2]
	\arrow["f", from=1-3, to=2-2]
\end{tikzcd},\]
from which follows that
\[T'_n\cong\{(y,t)\ |\ y\in Y_n,\ t\in\sSet/X(p^*(y),f)\}\]
and the map $p_*(f)$ then sends $(y,t)\in T'_n$ to $y\in Y_n$.

The same method can be extended to give us the pushforward along a map of marked
simplicial sets $p\colon (X,E_X)\rightarrow (Y,E_Y)$ in $\mSet$. Specifically,
our previous construction can be adapted to give us the $n$-simplices by
starting from maps $(\Delta^n)_\flat\rightarrow p_*(T,E_T)=(T',E_{T'})$, telling
us again that
\[T'_n\cong\{(y,t)\ |\ y\in Y_n,\ t\in\mSet/X(p^*(y),f)\},\]
while to get the markings we notice that every marked edge in $p_*(T,E_T)$
corresponds to a unique map $(\Delta^1)_\natural\rightarrow p_*(T,E_T)$ and the
same procedure allows us to write
\[E_{T'}=\{(y,t)\ |\ y\in E_Y,\ t\in\mSet/X(p^*(y),f)\},\]
which fully specifies the needed data.

Now, under which conditions on $p$ does this specify a Quillen adjunction when
the slices of $\mSet$ are equipped with the contravariant model structure? If
it is a coCartesian fibration, it generally doesn't, but it does when it is a
Cartesian fibration. How can we specify an approximation $q$ of $p_*$ such that,
after localizing in the infinity-sense, we get an adjunction $p^*\dashv q$?

Idea: use the theory of bifibrations. From a coCartesian fibration $\phi\colon
X\rightarrow Y$ we can construct a bifibration $E\rightarrow X\times Y$ by
constructing the maps $p\colon E\rightarrow X$, $q\colon E\rightarrow Y$ by
first taking the pullback of $\phi$ along $ev_0\colon Y^{\Delta^1}\rightarrow Y$
and then composing the map $E\rightarrow Y^{\Delta^1}$ with $ev_1$.
\[\begin{tikzcd}
	E & X \\
	{Y^{\Delta^1}} & Y \\
	Y
	\arrow["\phi", from=1-2, to=2-2]
	\arrow["{ev_0}", from=2-1, to=2-2]
	\arrow["p", from=1-1, to=1-2]
	\arrow[from=1-1, to=2-1]
	\arrow["{ev_1}", from=2-1, to=3-1]
	\arrow["\lrcorner"{anchor=center, pos=0.125}, draw=none, from=1-1, to=2-2]
	\arrow["q"', curve={height=20pt}, from=1-1, to=3-1]
\end{tikzcd}\]

We want to show that, for any (coCartesian?)\ morphism $f\colon A\rightarrow X$,
we have an isomorphism between $\phi_*(f)$ and $q_*p^*(f)$.

Canonically, we have
\[E_n=\{(x,\Delta^n\times\Delta^1\xrightarrow{g}Y)\ |\ x\in X_n,\
\phi(x)=g|_{\Delta^n\times\{0\}}\}\]
and, by pasting the pullback squares, we also get
\[(\dom(p^*f))_n=\{(a,\Delta^n\times\Delta^1\xrightarrow{g} Y)\ |\ a\in A_n,\
f(a)=g|_{\Delta^n\times\{0\}}\}\]
and therefore
\begin{align*}
  (\dom(q_*p^*(f)))_n &=\{(y,q^*(y)\xrightarrow{g} p^*(f))\ |\ y\in Y_n\} \\
                      &=\{(y,p_!q^*(y)\xrightarrow{g} f)\ |\ y\in Y_n\},
\end{align*}
which we want to relate to $\phi_*(f)$.

To do this, we want to understand the maps $p_!q^*(y)\xrightarrow{g} f$ and
somehow relate them to $\phi^*(y)\rightarrow f$. By definition,
\begin{align*}
  \dom(p_!q^*(y))_k&=\dom(q^*(y))_k \\
                   &=\{(x,\Delta^k\times\Delta^1\xrightarrow{h}Y,t)\ |\ x\in
                     X_k,\ \phi(x)=h|_{\Delta^k\times\{0\},\ t\in
                   (\Delta^n)_k},\ y(t)=h|_{\Delta^k\times\{1\}}\},
\end{align*}
with $q^*(y)(x,h,t)=(x,h)$, thus $p_!q^*(y)(x,h,t)=x$.

On the other hand, we have
\[\dom(\phi^*(y))_k=\{(x,t)\ |\ x\in X_k,\ t\in(\Delta^n)_k,\ \phi(x)=y(t)\}\]
and $\phi^*(y)(t,x)=x$.

If we can create a bijection between morphisms of the form $p_!q^*(y)\rightarrow
f$ and $\phi^*(y)\rightarrow f$ we are done. Unfortunately, I do not see how we
can do this: any morphism $p_!q^*(y)\rightarrow f$ induces a
morphism $\phi^*(y)\rightarrow f$ by precomposing with the inclusion
$\phi^*(y)\rightarrow p_!q^*(y)$, $(x,t)\mapsto(x,h_{\phi(x)},t)$, where
$h_{\phi(x)}$ is obtained by precomposing $\phi(x)\colon\Delta^k\rightarrow Y$
with $p_{\Delta^k}\colon\Delta^k\times\Delta^1\rightarrow\Delta^k$, but this
association is only surjective, not injective, and I have no good idea about how
to construct others.

\printbibliography

\end{document}
