\documentclass[a4paper,12pt,openany]{scrartcl}
\usepackage{reg2021}
\usepackage{quiver}
\usepackage{bbold}

\begin{document}
\noindent\textbf{Author}\hfill\textbf{Year} \linebreak
\vspace*{-.1cm} Matteo Durante\hfill 2022 \\

\noindent
\rule{\linewidth}{1pt}
\begin{center}
\Large
    \textbf{On Fibration Categories and\\ Finitely Complete $(\infty,1)$-categories} \\
\end{center}
\rule{\linewidth}{1pt}
\\

%%%%%%%%%%%%%%%%%%%%%%%%%%%%%%%%%%%%%%%%%%%%%%%%%%%%%%%%%%%%%%%%%%%%%%%%%%%%%%%

\newcommand{\La}{\Lambda}
\newcommand{\pa}{c}
\newcommand{\ob}{\operatorname{Ob}}
\newcommand{\mor}{\operatorname{Mor}}
\newcommand{\sto}{\twoheadrightarrow}

\newcommand{\plim}{\varprojlim}
\newcommand{\sst}{\subseteq}
\newcommand{\eq}{\operatorname{eq}}

\newcommand{\f}{\varphi}

\newcommand{\sing}{\operatorname{Sing}}

\newcommand{\ihom}{\underline{\Hom}}

\section{This is [not] an outline}

The goal of this thesis is to redefine the functor $F\colon
N(\FibCat)\rightarrow\Lexi$, from the nerve of the homotopical category of
fibration categories to $\infty$-category of finitely complete
$\infty$-categories and left exact functors, and prove that it induces an
equivalence after localizing the domain at its weak equivalences.

The functor $F$ is simply defined as the composite of the full embedding
$\iota\colon N(\FibCat)\rightarrow \FibCat_\infty$, induced by the nerve
functor, \wfd{(should we instead work with a stronger notion of fibration
$\infty$-category? also we need to define this $\infty$-category)} and the functor $\Hoi\colon
\FibCat_\infty\rightarrow\Lexi$ given by localizing fibration
$\infty$-categories. We know by \cite[Thm.\ 7.5.6]{Cis19} that localizing a
fibration $\infty$-category gives a finitely complete $\infty$-category and the
mapping on the maps is specified by the universal property of localizations.
A map between fibration $\infty$-categories is a left exact functor as defined
in \cite[Def.\ 7.5.2]{Cis19} and \cite[Thm.\ 7.5.28]{Cis19} guarantees that it
is indeed sent to a left exact functor, as needed.

To prove that $F$ induces the desired equivalence of $\infty$-categories we
define a (weak) fibrational structure on $\FibCat$ and $\FibCat_\infty$ and
verify that the right derived functors associated to $\iota$ and
$\Hoi$ are equivalences.

We already have notions of weak equivalence and fibration
internal to $\FibCat$ as provided in \cite{KS19}, so we only need to specify
ones internal to $\FibCat_\infty$ satisfying the definition of (weak)
fibration category given in \cite[Def.\ 7.4.12]{Cis19} (PLEASE LIST THE
PROPERTIES). To do so, we try to extend the definition given in the
1-categorical context, so that the functor $\iota$ will preserve the desired
structure. Hopefully we can just take the same definitions, without adaptations.

To show that the right derived functor of $\Hoi$ induces an equivalence we
remember that a finitely
complete $\infty$-category has a canonical fibrational structure, where weak
equivalences are the isomorphisms and fibrations are all the maps. Also, under
this construction a left exact functor between the induced fibration $\infty$-categories
is just a left exact functor in the usual sense, so we can fully embed
$\Lexi$ into $\FibCat_\infty$. This should define a right adjoint to
$\Hoi$, where the counit is the identity and the unit is given by the
localization maps $\cP\rightarrow L(\cP)$. We observe that such maps are weak
equivalences in $\FibCat_\infty$, so localizing this $\infty$-category gives us
an equivalence $L(\FibCat_\infty)\cong\Lexi$ involving the right derived
functor of $\Hoi$, as we desired.

Once we have done this, we need to check that $\iota$ is left exact, i.e.\ it
satisfies \cite[Def.\ 7.5.2]{Cis19} (PLEASE LIST THE PROPERTIES), for which we
only need condition (iii). It should be easy to check because it is induced by
the nerve functor.

To finally apply \cite[Thm.\ 7.6.15]{Cis19} to $\iota$ we still need to check
two more conditions (PLEASE LIST THEM). Notice that (a) is trivial, so we
actually only need to check (b).

\section{This is [not] a fibration}

One piece of the issue is to define a fibration $\infty$-category structure on
$\FibCat_\infty$ and to do so we need to first provide the definitions of the
elements at play.

\begin{defn}
  A \emph{fibration category} is a category $\cP$ with a subcategory whose
  morphisms are called \emph{weak equivalences} (denoted by
  $\xrightarrow{\sim}$) and one of \emph{fibrations} (denoted by
  $\twoheadrightarrow$) subject to the following axioms:
  \begin{enumerate}
    \item $\cP$ has a terminal object 1 and all objects are fibrant, that is
      every map to the terminal object is a fibration;
    \item pullbacks along fibrations exist in $\cP$ and (acyclic) fibrations are
      stable under pullback;
    \item every morphism in $\cP$ factors as a weak equivalence followed by a
      fibration;
    \item weak equivalences satisfy the 2-out-of-6 property.
  \end{enumerate}
\end{defn}

\begin{defn}
  A functor between fibration categories is \emph{left exact} if it preserves weak
  equivalences, fibrations, terminal objects and pullbacks along fibrations.
\end{defn}

Notice that this corresponds to the definition of \emph{exact functor} given
in \cite[Def.\ 2.7]{KS19}, however it also coincides with the one given in the
more general context of fibration $\infty$-categories in \cite[Def.\
7.5.2]{Cis19} (whose results we rely on), therefore to avoid conclusion I
adopted the naming given in the latter.

The category $\FibCat$ is then constructed by taking small fibration categories
as objects and left exact functors as arrows.

\begin{defn}
  A morphism in $\FibCat$ is a \emph{weak equivalence} if it induces an
  equivalence of categories on the associated homotopy categories.
\end{defn}

\begin{defn}
  A morphism $P\colon\cE\rightarrow\cD$ in $\FibCat$ is a \emph{fibration} if:
  \begin{enumerate}
    \item it is an isofibration;
    \item it lifts weak factorizations;
    \item it lifts pseudo-factorizations.
  \end{enumerate}
\end{defn}

The three conditions and the map being an isofibration are necessary to ensure
that a fibration of fibration categories is sent to a fibration in $\Lexi$ and
correspond to the requirement that it is a fibration on the underlying fibration
categories.

Notice that this is just taken from \cite[Def.\ 4.3]{KS19} and, under our
definition, every map to the terminal fibration category if a fibration.

\begin{prop}
  These classes of maps specify a fibration category structure on $\FibCat$.
\end{prop}
\begin{proof}
  Check \cite{KS19}.
\end{proof}

We want to switch to the $\infty$-categorical context, so let's introduce the
corresponding notions.

\begin{defn}
  A fibration $\infty$-category is a triple $(\cC,W,Fib)$ where $W\subset\cC$
  is a sub-$\infty$-category with the 2-out-of-3 property and $Fib$ is a class
  of fibrations such that:
  \begin{enumerate}
    \item for any pullback square
      \[\begin{tikzcd}
        {x'} & x \\
        {y'} & y
        \arrow["p", from=1-2, to=2-2]
        \arrow["{p'}"', from=1-1, to=2-1]
        \arrow["v"', from=2-1, to=2-2]
        \arrow["u", from=1-1, to=1-2]
      \end{tikzcd}\]
      in $\cC$ in which $p$ is a fibration between fibrant objects, if $p\in W$
      then $p'\in W$;
    \item given a map $f\colon x\rightarrow y$ in $\cC$ with $y$ fibrant, there
      exists a map $w\colon x\rightarrow x'$ in $W$ and a fibration $p\colon
      x'\rightarrow y$ that $f=p\cdot w$.
  \end{enumerate}
  The maps in $W$ are generally called \emph{weak equivalences}, while
  fibrations which are also in $W$ are \emph{trivial fibrations}.
\end{defn}

\begin{defn}
  Let $\cC,\cD$ be fibration $\infty$-categories. A functor
  $F\colon\cC\rightarrow\cD$ is \emph{left exact} if it has the following
  properties:
  \begin{enumerate}
    \item it preserves terminal objects;
    \item it preserves fibrations and trivial fibrations;
    \item given a pullback square
      \[\begin{tikzcd}
        {x'} & x \\
        {y'} & y
        \arrow["p", from=1-2, to=2-2]
        \arrow["{p'}"', from=1-1, to=2-1]
        \arrow["v"', from=2-1, to=2-2]
        \arrow["u", from=1-1, to=1-2]
      \end{tikzcd}\]
      in $\cC$ where $p$ is a fibration and $y$ and $y'$ are fibrant, the square
      \[\begin{tikzcd}
        {Fx'} & Fx \\
        {Fy'} & Fy
        \arrow["Fp", from=1-2, to=2-2]
        \arrow["{Fp'}"', from=1-1, to=2-1]
        \arrow["Fv"', from=2-1, to=2-2]
        \arrow["Fu", from=1-1, to=1-2]
      \end{tikzcd}\]
      is a pullback in $\cD$.
  \end{enumerate}
\end{defn}

Let's then consider $\FibCat_\infty$, the $\infty$-category of small fibration
$\infty$-categories and left exact functors. We can take the same definition of
fibration for $\FibCat_\infty$ and use the following one for weak equivalences, which then should (CHECK!)
make it a fibration $\infty$-category.

\begin{defn}
    A left exact functor between fibration $\infty$-categories
    $F\colon\cP\rightarrow\cQ$ is a weak equivalence if it induces an
    equivalence $ho(L(\cP))\rightarrow ho(L(\cQ))$. Equivalently, if it
    satisfies the conditions of \cite[Thm. 7.6.15]{Cis19}.
\end{defn}

\begin{prop}
  (DESIDERATA) The $\infty$-category $\FibCat_\infty$ with the specified classes
  of maps is a fibration $\infty$-category.
\end{prop}
\begin{proof}
  The problem is showing the factorization condition \wfd{(SHOW IT)}. Fibrations
  should be stable under pullback (which are computed as in $\sSet$ if we manage
  to describe $\FibCat_\infty$ as a full subcategory of functors with codomain
  in $\Cat_\infty$) and the same applies to trivial fibrations, with the proof
  being the same as the one given in the 1-categorical context. Every map to the
  terminal fibration $\infty$-category is naturally a fibration.
\end{proof}

\section{This is [not] an adjunction}

In this section we provide a right adjoint to the localization functor $\Hoi$.

\begin{rmk}
  Given a finitely complete $\infty$-category, we can provide a fibrational
  structure on it by defining weak equivalences to be the isomorphisms and
  fibrations to be all of the maps. Under this construction we see that, given
  two finitely complete $\infty$-categories, a functor between the induced
  fibration $\infty$-categories is left exact if and only if it is left exact
  when seen as a functor between the underlying $\infty$-categories.
\end{rmk}

\begin{prop}\label{adj}
  There is an adjunction $\Hoi\colon\FibCat_\infty\rightleftarrows\Lexi\colon
  i$, where $i$ is a full embedding, the unit is given by the
  localization maps $\gamma_{\cP}\colon\cP\rightarrow L(\cP)$ and the counit is
  the identity. This exhibits $\Lexi$ as a reflexive sub-$\infty$-category of
  $\FibCat_\infty$.
\end{prop}
\begin{proof}
  Let's consider for any finitely complete $\infty$-category $\cC$ the pullback
  diagram
  \[\begin{tikzcd}
    {\Hoi/\cC} & {\Lexi/\cC} \\
    {\FibCat_\infty} & {\Lexi}
    \arrow["\Hoi"', from=2-1, to=2-2]
    \arrow[from=1-2, to=2-2]
    \arrow[from=1-1, to=2-1]
    \arrow[from=1-1, to=1-2]
  \end{tikzcd}.\]
  We see that $\Hoi/\cC$ has a terminal object, namely the pair $(\cC,\id_\cC)$,
  where $\cC$ has the fibrational structure specified above: indeed, given any
  other object $(\cP,F)$, the space of maps
  $(\Hoi/\cC)((\cP,F),(\cC,\id_\cC))$ is contractible as it corresponds to the
  fiber at $F$ of the equivalence
  $\FibCat_\infty(\cP,\cC)\cong\Lexi(L(\cP),\cC)$ given by \cite[Prop.\
  7.5.11]{Cis19}.

  Unravelling everything, the constructed right adjoint is then the full
  embedding mapping everything in $\Lexi$ to its copy in $\FibCat_\infty$.
\end{proof}

Given this adjunction we present the following result, which reduces the
problem of proving that the functor we constructed initially between $L(\FibCat)$
and $\Lexi$ is an equivalence to showing it for the right derived functor of
$\iota$.

\begin{prop}
  The right derived functor of $\Hoi$ is an equivalence of $\infty$-categories
  $L(\FibCat_\infty)\cong\Lexi$.
\end{prop}
\begin{proof}
  By \ref{adj} we know that $\Hoi$ has a fully faithful right adjoint, so by
  \cite[Prop.\ 7.1.18]{Cis19} it is enough to show that a map in
  $\FibCat_\infty$ is mapped by $\Hoi$ to an isomorphism if and only if it is a
  weak equivalence. By \cite[Prop.\ 7.6.11]{Cis19} a left exact functor between
  fibration $\infty$-categories induces an equivalences on the localizations if
  and only if it does so on the homotopy categories, which is also the condition
  under which it is a weak equivalence.
\end{proof}

\section{This is [not] a $\infty$-category}

Our approach relies on defining a $\infty$-category $\FibCat_\infty$. How to do this?

1- Construct a $\infty$-category where objects are pairs of functors (specifying weak equivalences and fibrations) with the same codomain between $\infty$-categories, that is $\Fun(\{0\rightarrow 01\leftarrow 1\},\Cat_\infty)$. Then the morphisms are triples of functors making the proper diagram commute. We take the full subcategory of objects satisfying the axioms of a $\infty$-category with weak equivalences and fibrations, but how can we ask for pullbacks along fibrations to be preserved by the maps in this $\infty$-category? We also have to show that the hom-spaces are what we want. How to show then that a pullback along a fibration is a fibration? It may not be immediate because we are considering only triples of functors which make the diagrams commute as morphisms!

What if we take pairs of functors $\coprod_W\Delta^1\rightarrow\cD$,
$Fib\rightarrow\cC^{[1]}\rightarrow\cC$, where the composite is a Cartesian
fibration and $Fib\rightarrow\cC^{[1]}$ preserves Cartesian morphisms?
Essentially this is the definition of comprehension category, plus some extra
data given by the first functor. We call fibrations the maps in the image of
$Fib\rightarrow\cC$ and their compositions, while weak equivalences are the
maps identified by the first functor; we then ask for the axioms of fibration
$\infty$-category to be satisfied. This gives us more freedom when specifying
weak equivalences then just providing the definition we give when constructing a
tribe from a comprehension category. Also, in this way weak equivalences are
preserved.

Problem: maps between such objects may not be what we want! Namely, I want the
functor $\cC^{[1]}\rightarrow\cD^{[1]}$ to be induced by post-composing with
$\cC\rightarrow\cD$, but I don't know how to specify this. This (with the
condition of $Fib\rightarrow\cC$ being a cartesian fibration and the
factorization $Fib\rightarrow\cC^{[1]}\rightarrow\cC$ preserving cartesian
morphisms) would guarantee that pullbacks along fibrations are preserved by maps
between fibration categories.

2- Take the 1-category of $\infty$-categories with fibrations and weak
equivalences, then localize at categorical equivalences. What do the hom-spaces
look like? If instead we localized directly at weak equivalences we would get
directly $L(\FibCat_\infty)$, which supposedly is the ``true'' $\infty$-category
of $\infty$-categories with fibrations and weak equivalences.

It's important to relate the hom-spaces $\FibCat_\infty(\cP,\cQ)$ to the ones we
want, that is $k(\Fun_{lex}(\cP,\cQ))$. Also, what are the ones of $\Lexi$ like?
Supposedly $k(\Fun_{lex}(\cC,\cD))$ and we may show both claims using
\cite[Prop.\ 7.10.6]{Cis19}, but I do not think it would be easy.

\section{Other ideas}

1- We could also try to get an indirect comparison by looking at the inclusion of the category of fibration categories and of the category of finitely complete $\infty$-categories into the one of $\infty$-categories with weak equivalences and fibrations. We then give to the latter the structure of a fibrations category by taking the definition above. The axioms should be easy to show because a pullback along a fibration will simply be a pullback in $\sSet$ and then we can give it the desired structure as in the context of fibration categories. Problem: showing the factorization condition.

We have then to show that the inclusion functors are left exact as functors
between fibration categories. This means that the pullback of left exact
functors between finitely complete $\infty$-categories is again a finitely
complete $\infty$-category, which is proven in \cite[Lem.\ 2.11]{Szu17b}.

After this we have to show that the hypothesis of \cite[Thm.\ 7.6.15]{Cis19} hold. This is easy for (i), but for (ii) it's hard when it comes to the inclusion of fibration categories into $\infty$-categories with fibrations and weak equivalences. Namely, what if the domain of the morphism is a $\infty$-category? We have to find a span linking it to a weakly equivalent fibration category!

2- We can work with the 1-category of fibration $\infty$-categories and left
exact functors and give it a fibrational structure. Problems: how do we show
that after localizing a fibration of fibration categories we get an inner
fibration? That is needed for the localization functor to be left exact and
maybe we can use the ideas from \cite[Prop.\ 3.5]{Szu17b}. Also, how to show
condition (ii) of \cite[Thm.\ 7.6.15]{Cis19} for the inclusion of fibration
categories into fibration $\infty$-categories? We need to approximate a map from
a fibration $\infty$-category to a fibration category by one between fibration
categories, the same problem as before. Everything else should be easy.

Idea: given a morphism $\cC\rightarrow L(\cP)$, we consider a finitely complete
category $\sSet_0/\cC$, which can be given a fibrational structure mimicking the
cofibrational one specified in \cite[Def.\ 4.1]{Szu17b}. If the map
$\widetilde{\lim}\colon L(\sSet_0/\cC)\rightarrow\cC$ is a weak equivalence,
then we get the desired commutative square by looking at
\[\begin{tikzcd}
	\cC & {L(\cP)} \\
	{L(\sSet_0/\cC)} & {L(\sSet_0/\cC)}
	\arrow[from=2-2, to=1-2]
	\arrow[from=1-1, to=1-2]
  \arrow["\widetilde{\lim}", from=2-1, to=1-1]
	\arrow[Rightarrow, no head, from=2-1, to=2-2]
\end{tikzcd}\]
and somehow lift the map $L(\sSet_0/\cC)\rightarrow L(\cP)$ through $L$.

How to mimick that cofibrational structure? Weak equivalences are obviously the
maps sent by the functor $\lim$ to isomorphisms, but what about fibrations? We
need something such that the pullback of a trivial fibration is again a trivial
fibration.

We may also specify a weak equivalence $\cC\rightarrow L(\sSet_0/\cC)$ as in
\cite[Def.\ 4.6]{Szu17b} \wfd{(I don't really see how to adapt this)}.
We extend $\cC\rightarrow L(\cP)$ first to a functor $\sSet_0/\cC\rightarrow
L(\cP)$, $D\colon K\rightarrow\cC\mapsto\lim_KD$ \wfd{(doubtful about doing
this)} and then localize to get $L(\sSet_0/\cC)\rightarrow L(\cP)$, granting us a
diagram
\[\begin{tikzcd}
	\cC & {L(P)} \\
	\cC & {L(\sSet_0/\cC)}
	\arrow[from=2-2, to=1-2]
	\arrow[from=1-1, to=1-2]
	\arrow[Rightarrow, no head, from=2-1, to=1-1]
	\arrow[from=2-1, to=2-2]
\end{tikzcd}.\]

\section{Plan B: LCCC}

We can also rewrite the paper by Kapulkin about LCCC arising from TT using the language of localizations of quasi-categories. There they develop the relevant theory showing that under some conditions the frame associated to a fibration category is locally cartesian closed, but using Cisinski's results we can prove the same theorem directly using a more mainstream theory.

What should be included in such an overview?

1- Cisinski's theory of localizations (of fibration $\infty$-categories)

2- an introduction to contextual categories: where do they come from? Why are they useful? Check out Voevodsky's papers about C-systems

\printbibliography

\end{document}
