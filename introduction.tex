\chapter*{Introduction}
\addcontentsline{toc}{chapter}{Introduction}

We can also rewrite the paper by Kapulkin about LCCC arising from TT using the
language of localizations of quasi-categories. There they develop the relevant
theory showing that under some conditions the frame associated to a fibration
category is locally cartesian closed, but using Cisinski's results we can prove
the same theorem directly using a more mainstream theory.

What should be included in such an overview?

1- Cisinski's theory of localizations (of fibration $\infty$-categories)

2- an introduction to contextual categories: where do they come from? Why are
they useful? Check out Voevodsky's papers about C-systems

We explain what dependent type theory is (Martin-Lof's notes from 1984) and why
it's an interesting foundation of mathematics. We mention Homotopy Type Theory
as an effort to provide homotopical foundations which better model how we think
about identities, which explains why intensional identity types are more
interesting to us than extensional ones.

We move on to defining contextual categories (1211.2851, 1406.7413, 1507.02648)
and what
the Pi, Sigma and Id structures are (1406.7413, 1211.2851 Appendix B). To
understand what the link between such structures and syntactically presented
type theories we refer to 1507.02648, Sec.\ 1.1, while the statement of the
conjectured correspondence is in 1304.0680, Sec.\ 2.1.

Where does the link between dependent type theories and $\infty$-categories come
from? We see that $\infty$-categories intuitively model the behavior of type
theories and their type constructions, especially when considering Homotopy Type
Theory, however this relation is known only
partially (references in the intro of 1507.02648). The idea is that the type
theory we are interested in should be the internal language of some class of
$\infty$-categories and a precise statement would require us to provide
homotopical functors in both directions which induce an equivalence on the
associated $\infty$-categories. The idea is to construct the
functor from contextual categories as a localization functor, that is we need to
provide a homotopical structure on contextual categories, as they do in
1507.02648 (there should be an older reference) which then provides an
associated $\infty$-category. This is the object of the Initiality Conjecture,
stated in 1610.00037, in the hope that such a correspondence will extend to
Homotopy Type Theory and some notion of Elementary Higher Toposes, perhaps the
one specified in 1805.03805. At the moment we know that HoTT can be interpreted
in Higher Toposes with some structure. Current progress: 1709.09519, an upcoming
paper by Nguyen-Uemura (HoTTest talk).

Our aim is to show that when taking contextual categories with the structure we
specified earlier we obtain a locally cartesian closed $\infty$-category. To do
so we provide a fibrational structure on contextual categories (1304.0680,
1507.02648), which as we anticipate will imply that their simplicial
localizations are finitely complete. We also prove that the hypothesis of
\cite[Thm.\ 7.6.16]{Cis19} are satisfied, informing that this will be sufficient
to prove Kapulkin's main result from 1507.02648.

We then develop the theory of localizations of $\infty$-categories by Cisinski
and specifically develop the results concerning $\infty$-categories with
fibrations and weak equivalences. Localizations of such $\infty$-categories are
finitely complete. The objective is to show \cite[Thm.\ 7.6.16]{Cis19}. How in
depth should we go?

Why all of this is interesting: we are proving Kapulkin's result internalizing
all of the discussion within the language of $\infty$-category theory and
relying only on its simplicial model.

